%% Generated by Sphinx.
\def\sphinxdocclass{article}
\documentclass[letterpaper,10pt,english]{sphinxhowto}
\ifdefined\pdfpxdimen
   \let\sphinxpxdimen\pdfpxdimen\else\newdimen\sphinxpxdimen
\fi \sphinxpxdimen=.75bp\relax

\PassOptionsToPackage{warn}{textcomp}
\usepackage[utf8]{inputenc}
\ifdefined\DeclareUnicodeCharacter
 \ifdefined\DeclareUnicodeCharacterAsOptional
  \DeclareUnicodeCharacter{"00A0}{\nobreakspace}
  \DeclareUnicodeCharacter{"2500}{\sphinxunichar{2500}}
  \DeclareUnicodeCharacter{"2502}{\sphinxunichar{2502}}
  \DeclareUnicodeCharacter{"2514}{\sphinxunichar{2514}}
  \DeclareUnicodeCharacter{"251C}{\sphinxunichar{251C}}
  \DeclareUnicodeCharacter{"2572}{\textbackslash}
 \else
  \DeclareUnicodeCharacter{00A0}{\nobreakspace}
  \DeclareUnicodeCharacter{2500}{\sphinxunichar{2500}}
  \DeclareUnicodeCharacter{2502}{\sphinxunichar{2502}}
  \DeclareUnicodeCharacter{2514}{\sphinxunichar{2514}}
  \DeclareUnicodeCharacter{251C}{\sphinxunichar{251C}}
  \DeclareUnicodeCharacter{2572}{\textbackslash}
 \fi
\fi
\usepackage{cmap}
\usepackage[T1]{fontenc}
\usepackage{amsmath,amssymb,amstext}
\usepackage{babel}
\usepackage{times}
\usepackage[Sonny]{fncychap}
\ChNameVar{\Large\normalfont\sffamily}
\ChTitleVar{\Large\normalfont\sffamily}
\usepackage{sphinx}

\usepackage{geometry}

% Include hyperref last.
\usepackage{hyperref}
% Fix anchor placement for figures with captions.
\usepackage{hypcap}% it must be loaded after hyperref.
% Set up styles of URL: it should be placed after hyperref.
\urlstyle{same}

\addto\captionsenglish{\renewcommand{\figurename}{图}}
\addto\captionsenglish{\renewcommand{\tablename}{表}}
\addto\captionsenglish{\renewcommand{\literalblockname}{列表}}

\addto\captionsenglish{\renewcommand{\literalblockcontinuedname}{续上页}}
\addto\captionsenglish{\renewcommand{\literalblockcontinuesname}{continues on next page}}

\addto\extrasenglish{\def\pageautorefname{页}}




                \hypersetup{unicode=true}
                \usepackage{CJKutf8}
                \AtBeginDocument{\begin{CJK}{UTF8}{gbsn}}
                \AtEndDocument{\end{CJK}}
                

\title{Matplotlib简明手册 Documentation}
\date{2020 年 02 月 16 日}
\release{1.0}
\author{刘大拿}
\newcommand{\sphinxlogo}{\vbox{}}
\renewcommand{\releasename}{发布}
\makeindex

\begin{document}
\ifdefined\shorthandoff
  \ifnum\catcode`\=\string=\active\shorthandoff{=}\fi
  \ifnum\catcode`\"=\active\shorthandoff{"}\fi
\fi
\maketitle
\sphinxtableofcontents
\phantomsection\label{\detokenize{index::doc}}

\begin{itemize}
\item {} 
本内容旨在提供一个数据科学,人工智能基本入门简明教程

\item {} 
入门简明教程的含义是重在基本操作,没有理论讲解

\item {} 
学习方法:
\begin{itemize}
\item {} 
本文有配套视频课程,详见\sphinxhref{http://www.tulingxueyuan.com}{北京图灵学院官网}

\item {} 
按照本文或者视频,自己使用jupyter notebook 手动敲击代码后推敲结果

\item {} 
因为内容都是使用方式,需要记住即可,深入学习可以需要在人工智能或数据分析领域完成

\item {} 
本文配套Jupyter Notebook格式文档参见\sphinxhref{http://www.mycode.wang/blog/liudana}{麦扣网}

\item {} 
本文严重参考(抄袭)《Python数据科学手册》

\end{itemize}

\end{itemize}


\section{Matplotlib绘图简介}
\label{\detokenize{matplotlib_u7b80_u4ecb:matplotlib}}\label{\detokenize{matplotlib_u7b80_u4ecb::doc}}\begin{itemize}
\item {} 
建立在Numpy数组基础上的多平台数据可视化库

\item {} 
最初设计用来完善SciPy生态

\item {} 
良好的操作系统兼容性和图形显示底层接口兼容性

\item {} 
功能强大,设置灵活

\end{itemize}


\subsection{图形样式}
\label{\detokenize{matplotlib_u7b80_u4ecb:id1}}
利用Matplotlib绘图的图形风格可以设定,所有图形,可以通过\sphinxcode{\sphinxupquote{plt.style.use('classic')}}
设置图形样式,在1.5版本以前只能使用经典风格,新版本会有不同的风格设置。


\subsection{不同环境的使用}
\label{\detokenize{matplotlib_u7b80_u4ecb:id2}}
不同环境中使用Matplotlib可能稍有不同,需要注意:


\subsubsection{脚本中使用}
\label{\detokenize{matplotlib_u7b80_u4ecb:id3}}
在脚本中使用,并不能自动显示图形,需要使用\sphinxcode{\sphinxupquote{plt.show()}}明确让Matplotlib把图形显示出来,否则不会看到图形。


\subsubsection{IPython shell}
\label{\detokenize{matplotlib_u7b80_u4ecb:ipython-shell}}
在IPython shell中使用画图,需要使用魔法命令\sphinxcode{\sphinxupquote{\%matplotlib}}, 在以后的画图命令中,会自动显示出需要画的图形。


\subsubsection{Jupyter Notebook}
\label{\detokenize{matplotlib_u7b80_u4ecb:jupyter-notebook}}
在Jupyter Notebook中一般也需要使用魔法函数,但是相应魔法函数有两个:
\begin{itemize}
\item {} 
\sphinxcode{\sphinxupquote{\%matplotlib inline}}: 在Notebook中启动静态图形

\item {} 
\sphinxcode{\sphinxupquote{\%matplotlib notebook}}:在Notebook中启动交互式图形

\end{itemize}

使用以上两个魔法函数,理论上不需要在使用show函数强制显示图形,但在实践过程中,可能出现不显示的情况,此时建议使用
show显示图像。


\subsection{图像的保存}
\label{\detokenize{matplotlib_u7b80_u4ecb:id4}}
经常需要保存产生的图像以备使用,
如果使用的是交互式显示图形,可以直接点击保存按钮来保存图形,在所有代码中,都可以使用savefig函数来保存函数。


\subsection{两种不同的画图接口}
\label{\detokenize{matplotlib_u7b80_u4ecb:id5}}
Matplotlib支持两类画图接口:
\begin{itemize}
\item {} 
MATLAB风格接口:此类接口的特点是有状态的(Stateful),使用方便,对MATLAB用户友好,此类接口在pyplot中

\item {} 
面向对象接口:面向对象方式,处理复杂图形比较方便,能力相对比较强

\end{itemize}

此两种风格接口比较相似,容易混淆,我们在使用中,经常会混用。


\section{图形简单绘制}
\label{\detokenize{_u56fe_u5f62_u7b80_u5355_u7ed8_u5236:id1}}\label{\detokenize{_u56fe_u5f62_u7b80_u5355_u7ed8_u5236::doc}}
绘制图形基本函数是plot, 这个函数的工作原理大概是分别取x,y的值,然后取到坐标(x,y)后对不同的连续点进行连线,这样就形成了一条连续的线,如果适当调整各个点直接的跨度,就可以看到一条平滑曲线。


\subsection{绘制简单图形}
\label{\detokenize{_u56fe_u5f62_u7b80_u5355_u7ed8_u5236:id2}}
需要画如下简单线形图,我们需要创建一个图形fig和一个坐标轴ax。
\begin{itemize}
\item {} 
fig:figure(plt.Figure)是一个能容纳各种坐标轴,图形,文字和标签的容器。

\item {} 
ax:axes(plt.Axes)是一个带有刻度和标签的矩形,最终会包含各种可视化元素

\end{itemize}

在图形绘制中,y=f(x)类函数的显示可以看做是最简单的图形,我们来尝试.

\fvset{hllines={, ,}}%
\begin{sphinxVerbatim}[commandchars=\\\{\}]
\PYG{c+c1}{\PYGZsh{} 绘制 y=f(x)方程的图形}

\PYG{c+c1}{\PYGZsh{} 图形显示方式}
\PYG{o}{\PYGZpc{}}\PYG{n}{matplotlib} \PYG{n}{inline}
\PYG{k+kn}{import} \PYG{n+nn}{matplotlib.pyplot} \PYG{k+kn}{as} \PYG{n+nn}{plt}
\PYG{k+kn}{import} \PYG{n+nn}{numpy} \PYG{k+kn}{as} \PYG{n+nn}{np}

\PYG{c+c1}{\PYGZsh{} 图形显示风格}
\PYG{n}{plt}\PYG{o}{.}\PYG{n}{style}\PYG{o}{.}\PYG{n}{use}\PYG{p}{(}\PYG{l+s+s1}{\PYGZsq{}}\PYG{l+s+s1}{seaborn\PYGZhy{}whitegrid}\PYG{l+s+s1}{\PYGZsq{}}\PYG{p}{)}
 
\PYG{c+c1}{\PYGZsh{}创建fig和ax}
\PYG{n}{fig} \PYG{o}{=} \PYG{n}{plt}\PYG{o}{.}\PYG{n}{figure}\PYG{p}{(}\PYG{p}{)}
\PYG{n}{ax} \PYG{o}{=} \PYG{n}{plt}\PYG{o}{.}\PYG{n}{axes}\PYG{p}{(}\PYG{p}{)}




\PYG{n}{x} \PYG{o}{=} \PYG{n}{np}\PYG{o}{.}\PYG{n}{linspace}\PYG{p}{(}\PYG{l+m+mi}{0}\PYG{p}{,} \PYG{l+m+mi}{10}\PYG{p}{,} \PYG{l+m+mi}{100}\PYG{p}{)}
\PYG{c+c1}{\PYGZsh{} 显示sin函数图形}
\PYG{n}{plt}\PYG{o}{.}\PYG{n}{plot}\PYG{p}{(}\PYG{n}{x}\PYG{p}{,} \PYG{n}{np}\PYG{o}{.}\PYG{n}{sin}\PYG{p}{(}\PYG{n}{x}\PYG{p}{)}\PYG{p}{)}
\PYG{c+c1}{\PYGZsh{}显示cos函数图形}
\PYG{n}{plt}\PYG{o}{.}\PYG{n}{plot}\PYG{p}{(}\PYG{n}{x}\PYG{p}{,} \PYG{n}{np}\PYG{o}{.}\PYG{n}{cos}\PYG{p}{(}\PYG{n}{x}\PYG{p}{)}\PYG{p}{)}
\end{sphinxVerbatim}

\fvset{hllines={, ,}}%
\begin{sphinxVerbatim}[commandchars=\\\{\}]
\PYG{p}{[}\PYG{o}{\PYGZlt{}}\PYG{n}{matplotlib}\PYG{o}{.}\PYG{n}{lines}\PYG{o}{.}\PYG{n}{Line2D} \PYG{n}{at} \PYG{l+m+mh}{0x7fe669654cc0}\PYG{o}{\PYGZgt{}}\PYG{p}{]}
\end{sphinxVerbatim}

\sphinxincludegraphics{{output_7_1}.png}


\subsection{调整图形风格}
\label{\detokenize{_u56fe_u5f62_u7b80_u5355_u7ed8_u5236:id3}}
通过选用相应的参数,可以调整图形绘制线条的特征,比如颜色,风格等。图形颜色值可以支持多种定义风格,常见的包括:
\begin{itemize}
\item {} 
使用名称: color=’red’

\item {} 
颜色代码: 采用rgbcmyk格式, color=’g’

\item {} 
范围在0-1的灰度值: color=‘0.85’

\item {} 
十六进制: RGB模式, clolor=’\#FFDD45‘

\item {} 
RGB元素,范围0-1:color=(0.5, 0.2, 0.9)

\item {} 
HTML颜色名称:color=’chartreuse‘

\end{itemize}

\fvset{hllines={, ,}}%
\begin{sphinxVerbatim}[commandchars=\\\{\}]
\PYG{c+c1}{\PYGZsh{} 不同颜色的线条}
\PYG{n}{plt}\PYG{o}{.}\PYG{n}{plot}\PYG{p}{(}\PYG{n}{x}\PYG{p}{,} \PYG{n}{np}\PYG{o}{.}\PYG{n}{sin}\PYG{p}{(}\PYG{n}{x}\PYG{o}{\PYGZhy{}}\PYG{l+m+mi}{1}\PYG{p}{)}\PYG{p}{,} \PYG{n}{color}\PYG{o}{=}\PYG{l+s+s1}{\PYGZsq{}}\PYG{l+s+s1}{black}\PYG{l+s+s1}{\PYGZsq{}}\PYG{p}{)}
\PYG{n}{plt}\PYG{o}{.}\PYG{n}{plot}\PYG{p}{(}\PYG{n}{x}\PYG{p}{,} \PYG{n}{np}\PYG{o}{.}\PYG{n}{sin}\PYG{p}{(}\PYG{n}{x}\PYG{o}{\PYGZhy{}}\PYG{l+m+mi}{2}\PYG{p}{)}\PYG{p}{,} \PYG{n}{color}\PYG{o}{=}\PYG{l+s+s1}{\PYGZsq{}}\PYG{l+s+s1}{y}\PYG{l+s+s1}{\PYGZsq{}}\PYG{p}{)}
\PYG{n}{plt}\PYG{o}{.}\PYG{n}{plot}\PYG{p}{(}\PYG{n}{x}\PYG{p}{,} \PYG{n}{np}\PYG{o}{.}\PYG{n}{sin}\PYG{p}{(}\PYG{n}{x}\PYG{o}{\PYGZhy{}}\PYG{l+m+mi}{3}\PYG{p}{)}\PYG{p}{,} \PYG{n}{color}\PYG{o}{=}\PYG{l+s+s1}{\PYGZsq{}}\PYG{l+s+s1}{0.65}\PYG{l+s+s1}{\PYGZsq{}}\PYG{p}{)}
\PYG{n}{plt}\PYG{o}{.}\PYG{n}{plot}\PYG{p}{(}\PYG{n}{x}\PYG{p}{,} \PYG{n}{np}\PYG{o}{.}\PYG{n}{sin}\PYG{p}{(}\PYG{n}{x}\PYG{o}{\PYGZhy{}}\PYG{l+m+mi}{4}\PYG{p}{)}\PYG{p}{,} \PYG{n}{color}\PYG{o}{=}\PYG{l+s+s1}{\PYGZsq{}}\PYG{l+s+s1}{\PYGZsh{}FF0000}\PYG{l+s+s1}{\PYGZsq{}}\PYG{p}{)}
\PYG{n}{plt}\PYG{o}{.}\PYG{n}{plot}\PYG{p}{(}\PYG{n}{x}\PYG{p}{,} \PYG{n}{np}\PYG{o}{.}\PYG{n}{sin}\PYG{p}{(}\PYG{n}{x}\PYG{o}{\PYGZhy{}}\PYG{l+m+mi}{5}\PYG{p}{)}\PYG{p}{,} \PYG{n}{color}\PYG{o}{=}\PYG{p}{(}\PYG{l+m+mf}{0.2}\PYG{p}{,} \PYG{l+m+mf}{1.0}\PYG{p}{,} \PYG{l+m+mf}{1.0}\PYG{p}{)}\PYG{p}{)}
\PYG{n}{plt}\PYG{o}{.}\PYG{n}{plot}\PYG{p}{(}\PYG{n}{x}\PYG{p}{,} \PYG{n}{np}\PYG{o}{.}\PYG{n}{sin}\PYG{p}{(}\PYG{n}{x}\PYG{o}{\PYGZhy{}}\PYG{l+m+mi}{6}\PYG{p}{)}\PYG{p}{,} \PYG{n}{color}\PYG{o}{=}\PYG{l+s+s1}{\PYGZsq{}}\PYG{l+s+s1}{chartreuse}\PYG{l+s+s1}{\PYGZsq{}}\PYG{p}{)}
\end{sphinxVerbatim}

\fvset{hllines={, ,}}%
\begin{sphinxVerbatim}[commandchars=\\\{\}]
\PYG{p}{[}\PYG{o}{\PYGZlt{}}\PYG{n}{matplotlib}\PYG{o}{.}\PYG{n}{lines}\PYG{o}{.}\PYG{n}{Line2D} \PYG{n}{at} \PYG{l+m+mh}{0x7fe66949ff28}\PYG{o}{\PYGZgt{}}\PYG{p}{]}
\end{sphinxVerbatim}

\sphinxincludegraphics{{output_9_1}.png}

同样,使用linestyle参数可以定制图形中线的特征,常用的linestyle值为:
\begin{itemize}
\item {} 
solic: 实现,会用 \sphinxcode{\sphinxupquote{-}}做简写

\item {} 
dashed:虚线,用\sphinxcode{\sphinxupquote{-{-}}}做简写

\item {} 
dashdot:点划线, 用\sphinxcode{\sphinxupquote{-.}}简写

\item {} 
dotted:实心点线, 用\sphinxcode{\sphinxupquote{:}}简写

\end{itemize}

这几个是比较常用的,所有类型可以通过内置文档进行查看。

\fvset{hllines={, ,}}%
\begin{sphinxVerbatim}[commandchars=\\\{\}]
\PYG{c+c1}{\PYGZsh{} 不同颜色的线条}
\PYG{n}{plt}\PYG{o}{.}\PYG{n}{plot}\PYG{p}{(}\PYG{n}{x}\PYG{p}{,} \PYG{n}{np}\PYG{o}{.}\PYG{n}{sin}\PYG{p}{(}\PYG{n}{x}\PYG{o}{\PYGZhy{}}\PYG{l+m+mi}{1}\PYG{p}{)}\PYG{p}{,} \PYG{n}{linestyle}\PYG{o}{=}\PYG{l+s+s1}{\PYGZsq{}}\PYG{l+s+s1}{solid}\PYG{l+s+s1}{\PYGZsq{}}\PYG{p}{,} \PYG{n}{color}\PYG{o}{=}\PYG{l+s+s1}{\PYGZsq{}}\PYG{l+s+s1}{black}\PYG{l+s+s1}{\PYGZsq{}}\PYG{p}{)}
\PYG{n}{plt}\PYG{o}{.}\PYG{n}{plot}\PYG{p}{(}\PYG{n}{x}\PYG{p}{,} \PYG{n}{np}\PYG{o}{.}\PYG{n}{sin}\PYG{p}{(}\PYG{n}{x}\PYG{o}{\PYGZhy{}}\PYG{l+m+mi}{2}\PYG{p}{)}\PYG{p}{,}\PYG{n}{linestyle}\PYG{o}{=}\PYG{l+s+s1}{\PYGZsq{}}\PYG{l+s+s1}{\PYGZhy{}\PYGZhy{}}\PYG{l+s+s1}{\PYGZsq{}}\PYG{p}{,} \PYG{n}{color}\PYG{o}{=}\PYG{l+s+s1}{\PYGZsq{}}\PYG{l+s+s1}{y}\PYG{l+s+s1}{\PYGZsq{}}\PYG{p}{)}
\PYG{n}{plt}\PYG{o}{.}\PYG{n}{plot}\PYG{p}{(}\PYG{n}{x}\PYG{p}{,} \PYG{n}{np}\PYG{o}{.}\PYG{n}{sin}\PYG{p}{(}\PYG{n}{x}\PYG{o}{\PYGZhy{}}\PYG{l+m+mi}{3}\PYG{p}{)}\PYG{p}{,} \PYG{n}{linestyle}\PYG{o}{=}\PYG{l+s+s1}{\PYGZsq{}}\PYG{l+s+s1}{dashdot}\PYG{l+s+s1}{\PYGZsq{}}\PYG{p}{,}\PYG{n}{color}\PYG{o}{=}\PYG{l+s+s1}{\PYGZsq{}}\PYG{l+s+s1}{0.65}\PYG{l+s+s1}{\PYGZsq{}}\PYG{p}{)}
\PYG{n}{plt}\PYG{o}{.}\PYG{n}{plot}\PYG{p}{(}\PYG{n}{x}\PYG{p}{,} \PYG{n}{np}\PYG{o}{.}\PYG{n}{sin}\PYG{p}{(}\PYG{n}{x}\PYG{o}{\PYGZhy{}}\PYG{l+m+mi}{4}\PYG{p}{)}\PYG{p}{,} \PYG{n}{linestyle}\PYG{o}{=}\PYG{l+s+s1}{\PYGZsq{}}\PYG{l+s+s1}{:}\PYG{l+s+s1}{\PYGZsq{}}\PYG{p}{,}\PYG{n}{color}\PYG{o}{=}\PYG{l+s+s1}{\PYGZsq{}}\PYG{l+s+s1}{\PYGZsh{}FF0000}\PYG{l+s+s1}{\PYGZsq{}}\PYG{p}{)}
 
\end{sphinxVerbatim}

\fvset{hllines={, ,}}%
\begin{sphinxVerbatim}[commandchars=\\\{\}]
\PYG{p}{[}\PYG{o}{\PYGZlt{}}\PYG{n}{matplotlib}\PYG{o}{.}\PYG{n}{lines}\PYG{o}{.}\PYG{n}{Line2D} \PYG{n}{at} \PYG{l+m+mh}{0x7fe669478748}\PYG{o}{\PYGZgt{}}\PYG{p}{]}
\end{sphinxVerbatim}

\sphinxincludegraphics{{output_11_1}.png}

linestyle和color可以组合在一起使用更简洁的表示形式,plot函数提供了一个非关键字参数可以让我们使用组合简写:

\fvset{hllines={, ,}}%
\begin{sphinxVerbatim}[commandchars=\\\{\}]
\PYG{c+c1}{\PYGZsh{} 使用组合简写}

\PYG{n}{plt}\PYG{o}{.}\PYG{n}{plot}\PYG{p}{(}\PYG{n}{x}\PYG{p}{,} \PYG{n}{x}\PYG{o}{+}\PYG{l+m+mi}{0}\PYG{p}{,} \PYG{l+s+s1}{\PYGZsq{}}\PYG{l+s+s1}{\PYGZhy{}r}\PYG{l+s+s1}{\PYGZsq{}}\PYG{p}{)}
\PYG{n}{plt}\PYG{o}{.}\PYG{n}{plot}\PYG{p}{(}\PYG{n}{x}\PYG{p}{,} \PYG{n}{x}\PYG{o}{+}\PYG{l+m+mi}{1}\PYG{p}{,} \PYG{l+s+s1}{\PYGZsq{}}\PYG{l+s+s1}{\PYGZhy{}\PYGZhy{}g}\PYG{l+s+s1}{\PYGZsq{}}\PYG{p}{)}
\PYG{n}{plt}\PYG{o}{.}\PYG{n}{plot}\PYG{p}{(}\PYG{n}{x}\PYG{p}{,} \PYG{n}{x}\PYG{o}{+}\PYG{l+m+mi}{2}\PYG{p}{,} \PYG{l+s+s1}{\PYGZsq{}}\PYG{l+s+s1}{\PYGZhy{}.b}\PYG{l+s+s1}{\PYGZsq{}}\PYG{p}{)}
\PYG{n}{plt}\PYG{o}{.}\PYG{n}{plot}\PYG{p}{(}\PYG{n}{x}\PYG{p}{,} \PYG{n}{x}\PYG{o}{+}\PYG{l+m+mi}{3}\PYG{p}{,} \PYG{l+s+s1}{\PYGZsq{}}\PYG{l+s+s1}{:c}\PYG{l+s+s1}{\PYGZsq{}}\PYG{p}{)}
 
\end{sphinxVerbatim}

\fvset{hllines={, ,}}%
\begin{sphinxVerbatim}[commandchars=\\\{\}]
\PYG{p}{[}\PYG{o}{\PYGZlt{}}\PYG{n}{matplotlib}\PYG{o}{.}\PYG{n}{lines}\PYG{o}{.}\PYG{n}{Line2D} \PYG{n}{at} \PYG{l+m+mh}{0x7fe6693a7550}\PYG{o}{\PYGZgt{}}\PYG{p}{]}
\end{sphinxVerbatim}

\sphinxincludegraphics{{output_13_1}.png}


\subsection{调整坐标轴}
\label{\detokenize{_u56fe_u5f62_u7b80_u5355_u7ed8_u5236:id4}}

\subsubsection{xlim和ylim}
\label{\detokenize{_u56fe_u5f62_u7b80_u5355_u7ed8_u5236:xlimylim}}
Matplotlib会自动为图形调整坐标轴上下限,但有时候也需要自定义调整,常用的调整方法是:
\begin{itemize}
\item {} 
plt.xlim: x轴

\item {} 
plt.ylim: y轴

\end{itemize}

通过xlim和ylim可以调整x轴y轴的上下限,需要说明的是,这两个函数只是简单的把坐标轴数据改成两个参数之间的值就可以,它并不关心真正设置的值,比如可以设置逆序。

\fvset{hllines={, ,}}%
\begin{sphinxVerbatim}[commandchars=\\\{\}]
\PYG{c+c1}{\PYGZsh{} 手动设置坐标轴}
\PYG{c+c1}{\PYGZsh{} 考虑到x的取值,我们得到了一个很丑很不协调的sin}
\PYG{n}{plt}\PYG{o}{.}\PYG{n}{plot}\PYG{p}{(}\PYG{n}{x}\PYG{p}{,} \PYG{n}{np}\PYG{o}{.}\PYG{n}{sin}\PYG{p}{(}\PYG{n}{x}\PYG{p}{)}\PYG{p}{)}

\PYG{n}{plt}\PYG{o}{.}\PYG{n}{xlim}\PYG{p}{(}\PYG{o}{\PYGZhy{}}\PYG{l+m+mi}{5}\PYG{p}{,}\PYG{l+m+mi}{10}\PYG{p}{)}
\PYG{n}{plt}\PYG{o}{.}\PYG{n}{ylim}\PYG{p}{(}\PYG{o}{\PYGZhy{}}\PYG{l+m+mi}{3}\PYG{p}{,}\PYG{l+m+mi}{5}\PYG{p}{)}
\end{sphinxVerbatim}

\fvset{hllines={, ,}}%
\begin{sphinxVerbatim}[commandchars=\\\{\}]
\PYG{p}{(}\PYG{o}{\PYGZhy{}}\PYG{l+m+mi}{3}\PYG{p}{,} \PYG{l+m+mi}{5}\PYG{p}{)}
\end{sphinxVerbatim}

\sphinxincludegraphics{{output_15_1}.png}


\subsubsection{axis函数}
\label{\detokenize{_u56fe_u5f62_u7b80_u5355_u7ed8_u5236:axis}}
axis函数可以更灵活更强大的设置坐标轴信息, 可以传入\sphinxcode{\sphinxupquote{{[}xmin, xmax, ymin, ymax{]}}}来设置x,y轴的四个值,也还可以设置坐标轴的风格,比如:
\begin{itemize}
\item {} 
tight:把图形设置成紧凑模式,即坐标按照图形内容自动收紧坐标轴,不留下空白区域。

\item {} 
equal:图形显示分辨率为1:1

\end{itemize}

\fvset{hllines={, ,}}%
\begin{sphinxVerbatim}[commandchars=\\\{\}]
\PYG{c+c1}{\PYGZsh{} 手动设置坐标轴}

\PYG{n}{plt}\PYG{o}{.}\PYG{n}{plot}\PYG{p}{(}\PYG{n}{x}\PYG{p}{,} \PYG{n}{np}\PYG{o}{.}\PYG{n}{cos}\PYG{p}{(}\PYG{n}{x}\PYG{p}{)}\PYG{p}{)}
\PYG{n}{plt}\PYG{o}{.}\PYG{n}{axis}\PYG{p}{(}\PYG{p}{[}\PYG{o}{\PYGZhy{}}\PYG{l+m+mi}{5}\PYG{p}{,}\PYG{l+m+mi}{10}\PYG{p}{,}\PYG{o}{\PYGZhy{}}\PYG{l+m+mi}{3}\PYG{p}{,}\PYG{l+m+mi}{5}\PYG{p}{]}\PYG{p}{)}
\end{sphinxVerbatim}

\fvset{hllines={, ,}}%
\begin{sphinxVerbatim}[commandchars=\\\{\}]
\PYG{p}{[}\PYG{o}{\PYGZhy{}}\PYG{l+m+mi}{5}\PYG{p}{,} \PYG{l+m+mi}{10}\PYG{p}{,} \PYG{o}{\PYGZhy{}}\PYG{l+m+mi}{3}\PYG{p}{,} \PYG{l+m+mi}{5}\PYG{p}{]}
\end{sphinxVerbatim}

\sphinxincludegraphics{{output_17_1}.png}

\fvset{hllines={, ,}}%
\begin{sphinxVerbatim}[commandchars=\\\{\}]
\PYG{c+c1}{\PYGZsh{} 让图形自动紧凑}
\PYG{c+c1}{\PYGZsh{} 图形空白自动去掉}
\PYG{n}{plt}\PYG{o}{.}\PYG{n}{plot}\PYG{p}{(}\PYG{n}{x}\PYG{p}{,} \PYG{n}{np}\PYG{o}{.}\PYG{n}{cos}\PYG{p}{(}\PYG{n}{x}\PYG{p}{)}\PYG{p}{)}
\PYG{n}{plt}\PYG{o}{.}\PYG{n}{axis}\PYG{p}{(}\PYG{l+s+s1}{\PYGZsq{}}\PYG{l+s+s1}{tight}\PYG{l+s+s1}{\PYGZsq{}}\PYG{p}{)}
\end{sphinxVerbatim}

\fvset{hllines={, ,}}%
\begin{sphinxVerbatim}[commandchars=\\\{\}]
\PYG{p}{(}\PYG{l+m+mf}{0.0}\PYG{p}{,} \PYG{l+m+mf}{10.0}\PYG{p}{,} \PYG{o}{\PYGZhy{}}\PYG{l+m+mf}{0.9999471661761239}\PYG{p}{,} \PYG{l+m+mf}{1.0}\PYG{p}{)}
\end{sphinxVerbatim}

\sphinxincludegraphics{{output_18_1}.png}

\fvset{hllines={, ,}}%
\begin{sphinxVerbatim}[commandchars=\\\{\}]
\PYG{c+c1}{\PYGZsh{} 让图形自动紧凑}
\PYG{c+c1}{\PYGZsh{} 显示分辨率1:1}
\PYG{n}{plt}\PYG{o}{.}\PYG{n}{plot}\PYG{p}{(}\PYG{n}{x}\PYG{p}{,} \PYG{n}{np}\PYG{o}{.}\PYG{n}{cos}\PYG{p}{(}\PYG{n}{x}\PYG{p}{)}\PYG{p}{)}
\PYG{n}{plt}\PYG{o}{.}\PYG{n}{axis}\PYG{p}{(}\PYG{l+s+s1}{\PYGZsq{}}\PYG{l+s+s1}{equal}\PYG{l+s+s1}{\PYGZsq{}}\PYG{p}{)}
\end{sphinxVerbatim}

\fvset{hllines={, ,}}%
\begin{sphinxVerbatim}[commandchars=\\\{\}]
\PYG{p}{(}\PYG{l+m+mf}{0.0}\PYG{p}{,} \PYG{l+m+mf}{10.0}\PYG{p}{,} \PYG{o}{\PYGZhy{}}\PYG{l+m+mf}{1.0}\PYG{p}{,} \PYG{l+m+mf}{1.0}\PYG{p}{)}
\end{sphinxVerbatim}

\sphinxincludegraphics{{output_19_1}.png}


\subsection{设置图形标签}
\label{\detokenize{_u56fe_u5f62_u7b80_u5355_u7ed8_u5236:id5}}
在对图形进行设置的时候,可能需要一些文字性信息,此类信息主要包含:
\begin{itemize}
\item {} 
图形标题:plt.title

\item {} 
坐标轴标题: plt.xlabel, plt.ylabel

\item {} 
简易图例: plt.legend

\end{itemize}

对此类信息的设置,中文可能会出现乱码,需要对jupyter Notebook进行单独设置。


\subsubsection{简单标签和title设置}
\label{\detokenize{_u56fe_u5f62_u7b80_u5355_u7ed8_u5236:title}}
\fvset{hllines={, ,}}%
\begin{sphinxVerbatim}[commandchars=\\\{\}]
\PYG{c+c1}{\PYGZsh{} 设置sin图形标签}

\PYG{n}{plt}\PYG{o}{.}\PYG{n}{plot}\PYG{p}{(}\PYG{n}{x}\PYG{p}{,} \PYG{n}{np}\PYG{o}{.}\PYG{n}{sin}\PYG{p}{(}\PYG{n}{x} \PYG{p}{)}\PYG{p}{)}
\PYG{n}{plt}\PYG{o}{.}\PYG{n}{title}\PYG{p}{(}\PYG{l+s+s2}{\PYGZdq{}}\PYG{l+s+s2}{sin Function}\PYG{l+s+s2}{\PYGZdq{}}\PYG{p}{)}
\PYG{n}{plt}\PYG{o}{.}\PYG{n}{xlabel}\PYG{p}{(}\PYG{l+s+s2}{\PYGZdq{}}\PYG{l+s+s2}{X\PYGZhy{}Value}\PYG{l+s+s2}{\PYGZdq{}}\PYG{p}{)}
\PYG{n}{plt}\PYG{o}{.}\PYG{n}{ylabel}\PYG{p}{(}\PYG{l+s+s2}{\PYGZdq{}}\PYG{l+s+s2}{Y\PYGZhy{}Value}\PYG{l+s+s2}{\PYGZdq{}}\PYG{p}{)}
\end{sphinxVerbatim}

\fvset{hllines={, ,}}%
\begin{sphinxVerbatim}[commandchars=\\\{\}]
\PYG{n}{Text}\PYG{p}{(}\PYG{l+m+mi}{0}\PYG{p}{,}\PYG{l+m+mf}{0.5}\PYG{p}{,}\PYG{l+s+s1}{\PYGZsq{}}\PYG{l+s+s1}{Y\PYGZhy{}Value}\PYG{l+s+s1}{\PYGZsq{}}\PYG{p}{)}
\end{sphinxVerbatim}

\sphinxincludegraphics{{output_21_1}.png}


\subsubsection{简单图例设置}
\label{\detokenize{_u56fe_u5f62_u7b80_u5355_u7ed8_u5236:id6}}
通过legend可以设置图例,同时通过参数的调整可以细腻的设置图例的位置,形式等。

常见的参数为:
\begin{itemize}
\item {} 
loc: 图例位置

\item {} 
frameon:是否带边框

\item {} 
framealpha: 颜色透明

\item {} 
shadow: 阴影

\end{itemize}

\fvset{hllines={, ,}}%
\begin{sphinxVerbatim}[commandchars=\\\{\}]
\PYG{c+c1}{\PYGZsh{}使用面向对象的方法画图}

\PYG{n}{x} \PYG{o}{=} \PYG{n}{np}\PYG{o}{.}\PYG{n}{linspace}\PYG{p}{(}\PYG{l+m+mi}{0}\PYG{p}{,} \PYG{l+m+mi}{10}\PYG{p}{,} \PYG{l+m+mi}{50}\PYG{p}{)}

\PYG{n}{fig}\PYG{p}{,} \PYG{n}{ax} \PYG{o}{=} \PYG{n}{plt}\PYG{o}{.}\PYG{n}{subplots}\PYG{p}{(}\PYG{p}{)}

\PYG{n}{ax}\PYG{o}{.}\PYG{n}{plot}\PYG{p}{(}\PYG{n}{x}\PYG{p}{,} \PYG{n}{np}\PYG{o}{.}\PYG{n}{sin}\PYG{p}{(}\PYG{n}{x} \PYG{p}{)}\PYG{p}{,} \PYG{n}{color}\PYG{o}{=}\PYG{l+s+s1}{\PYGZsq{}}\PYG{l+s+s1}{red}\PYG{l+s+s1}{\PYGZsq{}}\PYG{p}{,} \PYG{n}{label}\PYG{o}{=}\PYG{l+s+s2}{\PYGZdq{}}\PYG{l+s+s2}{SIN}\PYG{l+s+s2}{\PYGZdq{}}\PYG{p}{)}
\PYG{n}{ax}\PYG{o}{.}\PYG{n}{plot}\PYG{p}{(}\PYG{n}{x}\PYG{p}{,} \PYG{n}{np}\PYG{o}{.}\PYG{n}{cos}\PYG{p}{(}\PYG{n}{x} \PYG{p}{)}\PYG{p}{,} \PYG{n}{color}\PYG{o}{=}\PYG{l+s+s1}{\PYGZsq{}}\PYG{l+s+s1}{blue}\PYG{l+s+s1}{\PYGZsq{}}\PYG{p}{,} \PYG{n}{label}\PYG{o}{=}\PYG{l+s+s2}{\PYGZdq{}}\PYG{l+s+s2}{COS}\PYG{l+s+s2}{\PYGZdq{}}\PYG{p}{)}

\PYG{c+c1}{\PYGZsh{}全部使用默认}
\PYG{n}{ax}\PYG{o}{.}\PYG{n}{legend}\PYG{p}{(}\PYG{p}{)}
\end{sphinxVerbatim}

\fvset{hllines={, ,}}%
\begin{sphinxVerbatim}[commandchars=\\\{\}]
\PYG{o}{\PYGZlt{}}\PYG{n}{matplotlib}\PYG{o}{.}\PYG{n}{legend}\PYG{o}{.}\PYG{n}{Legend} \PYG{n}{at} \PYG{l+m+mh}{0x7fe65abe54e0}\PYG{o}{\PYGZgt{}}
\end{sphinxVerbatim}

\sphinxincludegraphics{{output_23_1}.png}

\fvset{hllines={, ,}}%
\begin{sphinxVerbatim}[commandchars=\\\{\}]
\PYG{c+c1}{\PYGZsh{}使用面向对象的方法画图}

\PYG{n}{x} \PYG{o}{=} \PYG{n}{np}\PYG{o}{.}\PYG{n}{linspace}\PYG{p}{(}\PYG{l+m+mi}{0}\PYG{p}{,} \PYG{l+m+mi}{10}\PYG{p}{,} \PYG{l+m+mi}{50}\PYG{p}{)}

\PYG{n}{fig}\PYG{p}{,} \PYG{n}{ax} \PYG{o}{=} \PYG{n}{plt}\PYG{o}{.}\PYG{n}{subplots}\PYG{p}{(}\PYG{p}{)}

\PYG{n}{ax}\PYG{o}{.}\PYG{n}{plot}\PYG{p}{(}\PYG{n}{x}\PYG{p}{,} \PYG{n}{np}\PYG{o}{.}\PYG{n}{sin}\PYG{p}{(}\PYG{n}{x} \PYG{p}{)}\PYG{p}{,} \PYG{n}{color}\PYG{o}{=}\PYG{l+s+s1}{\PYGZsq{}}\PYG{l+s+s1}{red}\PYG{l+s+s1}{\PYGZsq{}}\PYG{p}{,} \PYG{n}{label}\PYG{o}{=}\PYG{l+s+s2}{\PYGZdq{}}\PYG{l+s+s2}{SIN}\PYG{l+s+s2}{\PYGZdq{}}\PYG{p}{)}
\PYG{n}{ax}\PYG{o}{.}\PYG{n}{plot}\PYG{p}{(}\PYG{n}{x}\PYG{p}{,} \PYG{n}{np}\PYG{o}{.}\PYG{n}{cos}\PYG{p}{(}\PYG{n}{x} \PYG{p}{)}\PYG{p}{,} \PYG{n}{color}\PYG{o}{=}\PYG{l+s+s1}{\PYGZsq{}}\PYG{l+s+s1}{blue}\PYG{l+s+s1}{\PYGZsq{}}\PYG{p}{,} \PYG{n}{label}\PYG{o}{=}\PYG{l+s+s2}{\PYGZdq{}}\PYG{l+s+s2}{COS}\PYG{l+s+s2}{\PYGZdq{}}\PYG{p}{)}

\PYG{c+c1}{\PYGZsh{}更改位置,添加边框}
\PYG{n}{ax}\PYG{o}{.}\PYG{n}{legend}\PYG{p}{(}\PYG{n}{loc}\PYG{o}{=}\PYG{l+s+s1}{\PYGZsq{}}\PYG{l+s+s1}{upper left}\PYG{l+s+s1}{\PYGZsq{}}\PYG{p}{,} \PYG{n}{frameon}\PYG{o}{=}\PYG{n+nb+bp}{True}\PYG{p}{)}
\end{sphinxVerbatim}

\fvset{hllines={, ,}}%
\begin{sphinxVerbatim}[commandchars=\\\{\}]
\PYG{o}{\PYGZlt{}}\PYG{n}{matplotlib}\PYG{o}{.}\PYG{n}{legend}\PYG{o}{.}\PYG{n}{Legend} \PYG{n}{at} \PYG{l+m+mh}{0x7fe65ab9ca58}\PYG{o}{\PYGZgt{}}
\end{sphinxVerbatim}

\sphinxincludegraphics{{output_24_1}.png}

\fvset{hllines={, ,}}%
\begin{sphinxVerbatim}[commandchars=\\\{\}]
\PYG{c+c1}{\PYGZsh{}使用面向对象的方法画图}

\PYG{n}{x} \PYG{o}{=} \PYG{n}{np}\PYG{o}{.}\PYG{n}{linspace}\PYG{p}{(}\PYG{l+m+mi}{0}\PYG{p}{,} \PYG{l+m+mi}{10}\PYG{p}{,} \PYG{l+m+mi}{50}\PYG{p}{)}

\PYG{n}{fig}\PYG{p}{,} \PYG{n}{ax} \PYG{o}{=} \PYG{n}{plt}\PYG{o}{.}\PYG{n}{subplots}\PYG{p}{(}\PYG{p}{)}

\PYG{n}{ax}\PYG{o}{.}\PYG{n}{plot}\PYG{p}{(}\PYG{n}{x}\PYG{p}{,} \PYG{n}{np}\PYG{o}{.}\PYG{n}{sin}\PYG{p}{(}\PYG{n}{x} \PYG{p}{)}\PYG{p}{,} \PYG{n}{color}\PYG{o}{=}\PYG{l+s+s1}{\PYGZsq{}}\PYG{l+s+s1}{red}\PYG{l+s+s1}{\PYGZsq{}}\PYG{p}{,} \PYG{n}{label}\PYG{o}{=}\PYG{l+s+s2}{\PYGZdq{}}\PYG{l+s+s2}{SIN}\PYG{l+s+s2}{\PYGZdq{}}\PYG{p}{)}
\PYG{n}{ax}\PYG{o}{.}\PYG{n}{plot}\PYG{p}{(}\PYG{n}{x}\PYG{p}{,} \PYG{n}{np}\PYG{o}{.}\PYG{n}{cos}\PYG{p}{(}\PYG{n}{x} \PYG{p}{)}\PYG{p}{,} \PYG{n}{color}\PYG{o}{=}\PYG{l+s+s1}{\PYGZsq{}}\PYG{l+s+s1}{blue}\PYG{l+s+s1}{\PYGZsq{}}\PYG{p}{,} \PYG{n}{label}\PYG{o}{=}\PYG{l+s+s2}{\PYGZdq{}}\PYG{l+s+s2}{COS}\PYG{l+s+s2}{\PYGZdq{}}\PYG{p}{)}

\PYG{c+c1}{\PYGZsh{}更改位置,添加边框}
\PYG{n}{ax}\PYG{o}{.}\PYG{n}{legend}\PYG{p}{(}\PYG{n}{loc}\PYG{o}{=}\PYG{l+s+s1}{\PYGZsq{}}\PYG{l+s+s1}{upper left}\PYG{l+s+s1}{\PYGZsq{}}\PYG{p}{,} \PYG{n}{frameon}\PYG{o}{=}\PYG{n+nb+bp}{True}\PYG{p}{,} \PYG{n}{shadow}\PYG{o}{=}\PYG{n+nb+bp}{True}\PYG{p}{,} \PYG{n}{framealpha}\PYG{o}{=}\PYG{l+m+mf}{0.2}\PYG{p}{)}
\end{sphinxVerbatim}

\fvset{hllines={, ,}}%
\begin{sphinxVerbatim}[commandchars=\\\{\}]
\PYG{o}{\PYGZlt{}}\PYG{n}{matplotlib}\PYG{o}{.}\PYG{n}{legend}\PYG{o}{.}\PYG{n}{Legend} \PYG{n}{at} \PYG{l+m+mh}{0x7fe65a834da0}\PYG{o}{\PYGZgt{}}
\end{sphinxVerbatim}

\sphinxincludegraphics{{output_25_1}.png}


\subsubsection{选择图例显示的元素}
\label{\detokenize{_u56fe_u5f62_u7b80_u5355_u7ed8_u5236:id7}}
选择图例显示元素一般有两种方法:
\begin{itemize}
\item {} 
通过设置图形的label属性,不设置的不显示

\item {} 
通过给legend传递需要显示的图形,不传递的不显示

\end{itemize}

\fvset{hllines={, ,}}%
\begin{sphinxVerbatim}[commandchars=\\\{\}]
\PYG{c+c1}{\PYGZsh{} 有时候可能我们并不想把所有线条的图例都显示出来,此时可以选择图例显示的元素}

\PYG{n}{x} \PYG{o}{=} \PYG{n}{np}\PYG{o}{.}\PYG{n}{linspace}\PYG{p}{(}\PYG{l+m+mi}{0}\PYG{p}{,} \PYG{l+m+mi}{10}\PYG{p}{,} \PYG{l+m+mi}{50}\PYG{p}{)}

\PYG{n}{fig}\PYG{p}{,} \PYG{n}{ax} \PYG{o}{=} \PYG{n}{plt}\PYG{o}{.}\PYG{n}{subplots}\PYG{p}{(}\PYG{p}{)}

\PYG{n}{sin\PYGZus{}line}  \PYG{o}{=} \PYG{n}{ax}\PYG{o}{.}\PYG{n}{plot}\PYG{p}{(}\PYG{n}{x}\PYG{p}{,} \PYG{n}{np}\PYG{o}{.}\PYG{n}{sin}\PYG{p}{(}\PYG{n}{x} \PYG{p}{)}\PYG{p}{,} \PYG{n}{color}\PYG{o}{=}\PYG{l+s+s1}{\PYGZsq{}}\PYG{l+s+s1}{red}\PYG{l+s+s1}{\PYGZsq{}}\PYG{p}{,} \PYG{n}{label}\PYG{o}{=}\PYG{l+s+s2}{\PYGZdq{}}\PYG{l+s+s2}{SIN}\PYG{l+s+s2}{\PYGZdq{}}\PYG{p}{)}
\PYG{n}{cos\PYGZus{}line} \PYG{o}{=} \PYG{n}{ax}\PYG{o}{.}\PYG{n}{plot}\PYG{p}{(}\PYG{n}{x}\PYG{p}{,} \PYG{n}{np}\PYG{o}{.}\PYG{n}{cos}\PYG{p}{(}\PYG{n}{x} \PYG{p}{)}\PYG{p}{,} \PYG{n}{color}\PYG{o}{=}\PYG{l+s+s1}{\PYGZsq{}}\PYG{l+s+s1}{blue}\PYG{l+s+s1}{\PYGZsq{}}\PYG{p}{,} \PYG{n}{label}\PYG{o}{=}\PYG{l+s+s2}{\PYGZdq{}}\PYG{l+s+s2}{COS}\PYG{l+s+s2}{\PYGZdq{}}\PYG{p}{)}

\PYG{c+c1}{\PYGZsh{}明确把需要添加图例的线条放入legend作为参数}
\PYG{c+c1}{\PYGZsh{}没有添加的不显示}
\PYG{n}{ax}\PYG{o}{.}\PYG{n}{legend}\PYG{p}{(}\PYG{n}{cos\PYGZus{}line}\PYG{p}{,} \PYG{n}{loc}\PYG{o}{=}\PYG{l+s+s1}{\PYGZsq{}}\PYG{l+s+s1}{upper left}\PYG{l+s+s1}{\PYGZsq{}}\PYG{p}{,} \PYG{n}{frameon}\PYG{o}{=}\PYG{n+nb+bp}{True}\PYG{p}{,} \PYG{n}{shadow}\PYG{o}{=}\PYG{n+nb+bp}{True}\PYG{p}{,} \PYG{n}{framealpha}\PYG{o}{=}\PYG{l+m+mf}{0.2}\PYG{p}{)}
\end{sphinxVerbatim}

\fvset{hllines={, ,}}%
\begin{sphinxVerbatim}[commandchars=\\\{\}]
\PYG{o}{\PYGZlt{}}\PYG{n}{matplotlib}\PYG{o}{.}\PYG{n}{legend}\PYG{o}{.}\PYG{n}{Legend} \PYG{n}{at} \PYG{l+m+mh}{0x7fe65a6f97f0}\PYG{o}{\PYGZgt{}}
\end{sphinxVerbatim}

\sphinxincludegraphics{{output_27_1}.png}

\fvset{hllines={, ,}}%
\begin{sphinxVerbatim}[commandchars=\\\{\}]
\PYG{c+c1}{\PYGZsh{} 有时候可能我们并不想把所有线条的图例都显示出来,此时可以选择图例显示的元素}

\PYG{n}{x} \PYG{o}{=} \PYG{n}{np}\PYG{o}{.}\PYG{n}{linspace}\PYG{p}{(}\PYG{l+m+mi}{0}\PYG{p}{,} \PYG{l+m+mi}{10}\PYG{p}{,} \PYG{l+m+mi}{50}\PYG{p}{)}

\PYG{n}{fig}\PYG{p}{,} \PYG{n}{ax} \PYG{o}{=} \PYG{n}{plt}\PYG{o}{.}\PYG{n}{subplots}\PYG{p}{(}\PYG{p}{)}

\PYG{c+c1}{\PYGZsh{}默认legend只显示有label的线条,没有的不显示}
\PYG{n}{sin\PYGZus{}line}  \PYG{o}{=} \PYG{n}{ax}\PYG{o}{.}\PYG{n}{plot}\PYG{p}{(}\PYG{n}{x}\PYG{p}{,} \PYG{n}{np}\PYG{o}{.}\PYG{n}{sin}\PYG{p}{(}\PYG{n}{x} \PYG{p}{)}\PYG{p}{,} \PYG{n}{color}\PYG{o}{=}\PYG{l+s+s1}{\PYGZsq{}}\PYG{l+s+s1}{red}\PYG{l+s+s1}{\PYGZsq{}}\PYG{p}{,} \PYG{n}{label}\PYG{o}{=}\PYG{l+s+s2}{\PYGZdq{}}\PYG{l+s+s2}{SIN}\PYG{l+s+s2}{\PYGZdq{}}\PYG{p}{)}
\PYG{n}{cos\PYGZus{}line} \PYG{o}{=} \PYG{n}{ax}\PYG{o}{.}\PYG{n}{plot}\PYG{p}{(}\PYG{n}{x}\PYG{p}{,} \PYG{n}{np}\PYG{o}{.}\PYG{n}{cos}\PYG{p}{(}\PYG{n}{x} \PYG{p}{)}\PYG{p}{,} \PYG{n}{color}\PYG{o}{=}\PYG{l+s+s1}{\PYGZsq{}}\PYG{l+s+s1}{blue}\PYG{l+s+s1}{\PYGZsq{}}\PYG{p}{)}

\PYG{c+c1}{\PYGZsh{}只有一个具有label参数,没有的就不显示}
\PYG{n}{ax}\PYG{o}{.}\PYG{n}{legend}\PYG{p}{(}\PYG{n}{loc}\PYG{o}{=}\PYG{l+s+s1}{\PYGZsq{}}\PYG{l+s+s1}{upper left}\PYG{l+s+s1}{\PYGZsq{}}\PYG{p}{,} \PYG{n}{frameon}\PYG{o}{=}\PYG{n+nb+bp}{True}\PYG{p}{,} \PYG{n}{shadow}\PYG{o}{=}\PYG{n+nb+bp}{True}\PYG{p}{,} \PYG{n}{framealpha}\PYG{o}{=}\PYG{l+m+mf}{0.2}\PYG{p}{)}
\end{sphinxVerbatim}

\fvset{hllines={, ,}}%
\begin{sphinxVerbatim}[commandchars=\\\{\}]
\PYG{o}{\PYGZlt{}}\PYG{n}{matplotlib}\PYG{o}{.}\PYG{n}{legend}\PYG{o}{.}\PYG{n}{Legend} \PYG{n}{at} \PYG{l+m+mh}{0x7fe65a64df98}\PYG{o}{\PYGZgt{}}
\end{sphinxVerbatim}

\sphinxincludegraphics{{output_28_1}.png}


\subsubsection{图例中显示不同尺寸}
\label{\detokenize{_u56fe_u5f62_u7b80_u5355_u7ed8_u5236:id8}}
\fvset{hllines={, ,}}%
\begin{sphinxVerbatim}[commandchars=\\\{\}]
\PYG{c+c1}{\PYGZsh{} scatter用来画散点图}
\PYG{c+c1}{\PYGZsh{} 我们将在后面章节中讲述}

\PYG{n}{rng} \PYG{o}{=} \PYG{n}{np}\PYG{o}{.}\PYG{n}{random}\PYG{o}{.}\PYG{n}{RandomState}\PYG{p}{(}\PYG{l+m+mi}{0}\PYG{p}{)}
\PYG{n}{x} \PYG{o}{=} \PYG{n}{rng}\PYG{o}{.}\PYG{n}{randn}\PYG{p}{(}\PYG{l+m+mi}{100}\PYG{p}{)}
\PYG{n}{y} \PYG{o}{=} \PYG{n}{rng}\PYG{o}{.}\PYG{n}{randn}\PYG{p}{(}\PYG{l+m+mi}{100}\PYG{p}{)}

\PYG{n}{colors} \PYG{o}{=} \PYG{n}{rng}\PYG{o}{.}\PYG{n}{rand}\PYG{p}{(}\PYG{l+m+mi}{100}\PYG{p}{)}
\PYG{n}{sizes} \PYG{o}{=} \PYG{l+m+mi}{1000} \PYG{o}{*} \PYG{n}{rng}\PYG{o}{.}\PYG{n}{rand}\PYG{p}{(}\PYG{l+m+mi}{100}\PYG{p}{)}

\PYG{c+c1}{\PYGZsh{} 设定横轴坐标的范围}
\PYG{n}{plt}\PYG{o}{.}\PYG{n}{xlim}\PYG{p}{(}\PYG{o}{\PYGZhy{}}\PYG{l+m+mi}{3}\PYG{p}{,} \PYG{l+m+mi}{6}\PYG{p}{)}
\PYG{c+c1}{\PYGZsh{}画散点图}
\PYG{n}{plt}\PYG{o}{.}\PYG{n}{scatter}\PYG{p}{(}\PYG{n}{x}\PYG{p}{,} \PYG{n}{y}\PYG{p}{,} \PYG{n}{c}\PYG{o}{=}\PYG{n}{colors}\PYG{p}{,} \PYG{n}{s}\PYG{o}{=}\PYG{n}{sizes}\PYG{p}{,} \PYG{n}{alpha}\PYG{o}{=}\PYG{l+m+mf}{0.4}\PYG{p}{,} \PYG{n}{cmap}\PYG{o}{=}\PYG{l+s+s1}{\PYGZsq{}}\PYG{l+s+s1}{viridis}\PYG{l+s+s1}{\PYGZsq{}}\PYG{p}{)}

\PYG{c+c1}{\PYGZsh{}显示颜色条}
\PYG{n}{plt}\PYG{o}{.}\PYG{n}{colorbar}\PYG{p}{(}\PYG{p}{)}

\PYG{c+c1}{\PYGZsh{}画图例散点图}
\PYG{k}{for} \PYG{n}{a} \PYG{o+ow}{in} \PYG{p}{[}\PYG{l+m+mf}{0.1}\PYG{p}{,} \PYG{l+m+mf}{0.3}\PYG{p}{,} \PYG{l+m+mf}{0.5}\PYG{p}{,} \PYG{l+m+mf}{0.7}\PYG{p}{,} \PYG{l+m+mf}{0.9}\PYG{p}{]}\PYG{p}{:}
    \PYG{n}{plt}\PYG{o}{.}\PYG{n}{scatter}\PYG{p}{(}\PYG{p}{[}\PYG{p}{]}\PYG{p}{,} \PYG{p}{[}\PYG{p}{]}\PYG{p}{,} \PYG{n}{c}\PYG{o}{=}\PYG{l+s+s1}{\PYGZsq{}}\PYG{l+s+s1}{red}\PYG{l+s+s1}{\PYGZsq{}}\PYG{p}{,} \PYG{n}{alpha}\PYG{o}{=}\PYG{l+m+mf}{0.5}\PYG{p}{,} \PYG{n}{s}\PYG{o}{=}\PYG{n}{a}\PYG{o}{*}\PYG{l+m+mi}{300}\PYG{p}{,} \PYG{n}{label}\PYG{o}{=}\PYG{l+s+s2}{\PYGZdq{}}\PYG{l+s+s2}{\PYGZob{}\PYGZcb{} Label}\PYG{l+s+s2}{\PYGZdq{}}\PYG{o}{.}\PYG{n}{format}\PYG{p}{(}\PYG{n}{a}\PYG{p}{)}\PYG{p}{)}

\PYG{c+c1}{\PYGZsh{}画图例文字  }
\PYG{n}{plt}\PYG{o}{.}\PYG{n}{legend}\PYG{p}{(}\PYG{n}{scatterpoints}\PYG{o}{=}\PYG{l+m+mi}{1}\PYG{p}{,} \PYG{n}{frameon}\PYG{o}{=}\PYG{n+nb+bp}{False}\PYG{p}{,} \PYG{n}{labelspacing}\PYG{o}{=}\PYG{l+m+mi}{1}\PYG{p}{,} \PYG{n}{title}\PYG{o}{=}\PYG{l+s+s1}{\PYGZsq{}}\PYG{l+s+s1}{Random Value}\PYG{l+s+s1}{\PYGZsq{}}\PYG{p}{)}
\end{sphinxVerbatim}

\fvset{hllines={, ,}}%
\begin{sphinxVerbatim}[commandchars=\\\{\}]
\PYG{o}{\PYGZlt{}}\PYG{n}{matplotlib}\PYG{o}{.}\PYG{n}{legend}\PYG{o}{.}\PYG{n}{Legend} \PYG{n}{at} \PYG{l+m+mh}{0x7fe659c3dba8}\PYG{o}{\PYGZgt{}}
\end{sphinxVerbatim}

\sphinxincludegraphics{{output_30_1}.png}


\subsection{配置颜色条}
\label{\detokenize{_u56fe_u5f62_u7b80_u5355_u7ed8_u5236:id9}}
使用颜色条来配置图例是一种常规操作, 本章讲述颜色条的配置。

\fvset{hllines={, ,}}%
\begin{sphinxVerbatim}[commandchars=\\\{\}]
\PYG{c+c1}{\PYGZsh{} 准备数据}
\PYG{k+kn}{import} \PYG{n+nn}{matplotlib.pyplot} \PYG{k+kn}{as} \PYG{n+nn}{plt}
\PYG{k+kn}{import} \PYG{n+nn}{numpy} \PYG{k+kn}{as} \PYG{n+nn}{np}

\PYG{n}{plt}\PYG{o}{.}\PYG{n}{style}\PYG{o}{.}\PYG{n}{use}\PYG{p}{(}\PYG{l+s+s1}{\PYGZsq{}}\PYG{l+s+s1}{classic}\PYG{l+s+s1}{\PYGZsq{}}\PYG{p}{)}
\PYG{o}{\PYGZpc{}}\PYG{n}{matplotlib} \PYG{n}{inline}
\end{sphinxVerbatim}


\subsubsection{简单颜色条配置}
\label{\detokenize{_u56fe_u5f62_u7b80_u5355_u7ed8_u5236:id10}}
\fvset{hllines={, ,}}%
\begin{sphinxVerbatim}[commandchars=\\\{\}]
\PYG{n}{x} \PYG{o}{=} \PYG{n}{np}\PYG{o}{.}\PYG{n}{linspace}\PYG{p}{(}\PYG{l+m+mi}{0}\PYG{p}{,}\PYG{l+m+mi}{10}\PYG{p}{,} \PYG{l+m+mi}{1000}\PYG{p}{)}
\PYG{n}{I} \PYG{o}{=} \PYG{n}{np}\PYG{o}{.}\PYG{n}{sin}\PYG{p}{(}\PYG{n}{x}\PYG{p}{)} \PYG{o}{*} \PYG{n}{np}\PYG{o}{.}\PYG{n}{cos}\PYG{p}{(}\PYG{n}{x}\PYG{p}{[}\PYG{p}{:}\PYG{p}{,} \PYG{n}{np}\PYG{o}{.}\PYG{n}{newaxis}\PYG{p}{]}\PYG{p}{)}


\PYG{n}{plt}\PYG{o}{.}\PYG{n}{imshow}\PYG{p}{(}\PYG{n}{I}\PYG{p}{)}
\PYG{n}{plt}\PYG{o}{.}\PYG{n}{colorbar}\PYG{p}{(}\PYG{p}{)}
\end{sphinxVerbatim}

\fvset{hllines={, ,}}%
\begin{sphinxVerbatim}[commandchars=\\\{\}]
\PYG{o}{\PYGZlt{}}\PYG{n}{matplotlib}\PYG{o}{.}\PYG{n}{colorbar}\PYG{o}{.}\PYG{n}{Colorbar} \PYG{n}{at} \PYG{l+m+mh}{0x7f387a03ca58}\PYG{o}{\PYGZgt{}}
\end{sphinxVerbatim}

\sphinxincludegraphics{{output_34_1}.png}


\subsubsection{配置颜色条}
\label{\detokenize{_u56fe_u5f62_u7b80_u5355_u7ed8_u5236:id11}}
可以通过cmap参数为图形设置颜色条的配置方案, 所有配色方案都在
plt.cm命名空间里,可以直接通过 \sphinxcode{\sphinxupquote{plt.cm.\textless{}TAB\textgreater{}}}查看。

选择合理的配色方案能让实行示例清晰明了,常见的配色方案有:
\begin{itemize}
\item {} 
顺序配色方案: 由一组连续的颜色构成的方案,例如binary或者viridis

\item {} 
互逆配色方案: 由两种互补的颜色构成,表示正反两种含义,例如RdBu或者PuOr

\item {} 
定性配色方案:随机顺序的一组颜色,例如rainbow或者jet

\end{itemize}

\fvset{hllines={, ,}}%
\begin{sphinxVerbatim}[commandchars=\\\{\}]
\PYG{c+c1}{\PYGZsh{}使用灰色颜色条}
\PYG{n}{x} \PYG{o}{=} \PYG{n}{np}\PYG{o}{.}\PYG{n}{linspace}\PYG{p}{(}\PYG{l+m+mi}{0}\PYG{p}{,}\PYG{l+m+mi}{10}\PYG{p}{,} \PYG{l+m+mi}{1000}\PYG{p}{)}
\PYG{n}{I} \PYG{o}{=} \PYG{n}{np}\PYG{o}{.}\PYG{n}{sin}\PYG{p}{(}\PYG{n}{x}\PYG{p}{)} \PYG{o}{*} \PYG{n}{np}\PYG{o}{.}\PYG{n}{cos}\PYG{p}{(}\PYG{n}{x}\PYG{p}{[}\PYG{p}{:}\PYG{p}{,} \PYG{n}{np}\PYG{o}{.}\PYG{n}{newaxis}\PYG{p}{]}\PYG{p}{)}

\PYG{n}{plt}\PYG{o}{.}\PYG{n}{imshow}\PYG{p}{(}\PYG{n}{I}\PYG{p}{,} \PYG{n}{cmap}\PYG{o}{=}\PYG{l+s+s1}{\PYGZsq{}}\PYG{l+s+s1}{RdBu}\PYG{l+s+s1}{\PYGZsq{}}\PYG{p}{)}
\PYG{n}{plt}\PYG{o}{.}\PYG{n}{colorbar}\PYG{p}{(}\PYG{p}{)}
\end{sphinxVerbatim}

\fvset{hllines={, ,}}%
\begin{sphinxVerbatim}[commandchars=\\\{\}]
\PYG{o}{\PYGZlt{}}\PYG{n}{matplotlib}\PYG{o}{.}\PYG{n}{colorbar}\PYG{o}{.}\PYG{n}{Colorbar} \PYG{n}{at} \PYG{l+m+mh}{0x7fe659eca5c0}\PYG{o}{\PYGZgt{}}
\end{sphinxVerbatim}

\sphinxincludegraphics{{output_37_1}.png}


\subsubsection{离散型颜色条}
\label{\detokenize{_u56fe_u5f62_u7b80_u5355_u7ed8_u5236:id12}}
颜色条默认是连续的,可以通过设置配色方案和颜色的区间量来显示离散型颜色条。

\fvset{hllines={, ,}}%
\begin{sphinxVerbatim}[commandchars=\\\{\}]
\PYG{c+c1}{\PYGZsh{}使用灰色颜色条}
\PYG{n}{x} \PYG{o}{=} \PYG{n}{np}\PYG{o}{.}\PYG{n}{linspace}\PYG{p}{(}\PYG{l+m+mi}{0}\PYG{p}{,}\PYG{l+m+mi}{10}\PYG{p}{,} \PYG{l+m+mi}{1000}\PYG{p}{)}
\PYG{n}{I} \PYG{o}{=} \PYG{n}{np}\PYG{o}{.}\PYG{n}{sin}\PYG{p}{(}\PYG{n}{x}\PYG{p}{)} \PYG{o}{*} \PYG{n}{np}\PYG{o}{.}\PYG{n}{cos}\PYG{p}{(}\PYG{n}{x}\PYG{p}{[}\PYG{p}{:}\PYG{p}{,} \PYG{n}{np}\PYG{o}{.}\PYG{n}{newaxis}\PYG{p}{]}\PYG{p}{)}

\PYG{c+c1}{\PYGZsh{}get\PYGZus{}cmap两个参数}
\PYG{c+c1}{\PYGZsh{} 1. 配色方案}
\PYG{c+c1}{\PYGZsh{} 2. 颜色多少等分}
\PYG{n}{c} \PYG{o}{=} \PYG{n}{plt}\PYG{o}{.}\PYG{n}{cm}\PYG{o}{.}\PYG{n}{get\PYGZus{}cmap}\PYG{p}{(}\PYG{l+s+s2}{\PYGZdq{}}\PYG{l+s+s2}{PuOr}\PYG{l+s+s2}{\PYGZdq{}}\PYG{p}{,} \PYG{l+m+mi}{10}\PYG{p}{)}
\PYG{n}{plt}\PYG{o}{.}\PYG{n}{imshow}\PYG{p}{(}\PYG{n}{I}\PYG{p}{,} \PYG{n}{cmap}\PYG{o}{=}\PYG{n}{c}\PYG{p}{)}
\PYG{n}{plt}\PYG{o}{.}\PYG{n}{colorbar}\PYG{p}{(}\PYG{p}{)}
\end{sphinxVerbatim}

\fvset{hllines={, ,}}%
\begin{sphinxVerbatim}[commandchars=\\\{\}]
\PYG{o}{\PYGZlt{}}\PYG{n}{matplotlib}\PYG{o}{.}\PYG{n}{colorbar}\PYG{o}{.}\PYG{n}{Colorbar} \PYG{n}{at} \PYG{l+m+mh}{0x7fe6595ac5c0}\PYG{o}{\PYGZgt{}}
\end{sphinxVerbatim}

\sphinxincludegraphics{{output_39_1}.png}


\section{散点图}
\label{\detokenize{_u6563_u70b9_u56fe:id1}}\label{\detokenize{_u6563_u70b9_u56fe::doc}}
散点图(Scatter Plot)主要是以点为主,数据是不连续的数据,通过设置线的型号为圆点来完成。
其余的线的形状为:
\begin{itemize}
\item {} 
\sphinxcode{\sphinxupquote{'.'}}          point marker

\item {} 
\sphinxcode{\sphinxupquote{','}}          pixel marker

\item {} 
\sphinxcode{\sphinxupquote{'o'}}          circle marker

\item {} 
\sphinxcode{\sphinxupquote{'v'}}          triangle\_down marker

\item {} 
\sphinxcode{\sphinxupquote{'\textasciicircum{}'}}          triangle\_up marker

\item {} 
\sphinxcode{\sphinxupquote{'\textless{}'}}          triangle\_left marker

\item {} 
\sphinxcode{\sphinxupquote{'\textgreater{}'}}          triangle\_right marker

\item {} 
\sphinxcode{\sphinxupquote{'1'}}          tri\_down marker

\item {} 
\sphinxcode{\sphinxupquote{'2'}}          tri\_up marker

\item {} 
\sphinxcode{\sphinxupquote{'3'}}          tri\_left marker

\item {} 
\sphinxcode{\sphinxupquote{'4'}}          tri\_right marker

\item {} 
\sphinxcode{\sphinxupquote{'s'}}          square marker

\item {} 
\sphinxcode{\sphinxupquote{'p'}}          pentagon marker

\item {} 
\sphinxcode{\sphinxupquote{'*'}}          star marker

\item {} 
\sphinxcode{\sphinxupquote{'h'}}          hexagon1 marker

\item {} 
\sphinxcode{\sphinxupquote{'H'}}          hexagon2 marker

\item {} 
\sphinxcode{\sphinxupquote{'+'}}          plus marker

\item {} 
\sphinxcode{\sphinxupquote{'x'}}          x marker

\item {} 
\sphinxcode{\sphinxupquote{'D'}}          diamond marker

\item {} 
\sphinxcode{\sphinxupquote{'d'}}          thin\_diamond marker

\item {} 
\sphinxcode{\sphinxupquote{'\textbar{}'}}          vline marker

\item {} 
\sphinxcode{\sphinxupquote{'\_'}}          hline marker

\end{itemize}


\subsection{简单散点图}
\label{\detokenize{_u6563_u70b9_u56fe:id2}}
\fvset{hllines={, ,}}%
\begin{sphinxVerbatim}[commandchars=\\\{\}]
\PYG{c+c1}{\PYGZsh{}准备环境}
\PYG{o}{\PYGZpc{}}\PYG{n}{matplotlib} \PYG{n}{inline}
\PYG{k+kn}{import} \PYG{n+nn}{matplotlib.pyplot} \PYG{k+kn}{as} \PYG{n+nn}{plt}
\PYG{k+kn}{import} \PYG{n+nn}{numpy} \PYG{k+kn}{as} \PYG{n+nn}{np}

\PYG{c+c1}{\PYGZsh{} 设置风格}
\PYG{n}{plt}\PYG{o}{.}\PYG{n}{style}\PYG{o}{.}\PYG{n}{use}\PYG{p}{(}\PYG{l+s+s1}{\PYGZsq{}}\PYG{l+s+s1}{seaborn\PYGZhy{}whitegrid}\PYG{l+s+s1}{\PYGZsq{}}\PYG{p}{)}
\end{sphinxVerbatim}

\fvset{hllines={, ,}}%
\begin{sphinxVerbatim}[commandchars=\\\{\}]
\PYG{n}{x} \PYG{o}{=} \PYG{n}{np}\PYG{o}{.}\PYG{n}{linspace}\PYG{p}{(}\PYG{l+m+mi}{0}\PYG{p}{,}\PYG{l+m+mi}{10}\PYG{p}{,} \PYG{l+m+mi}{30}\PYG{p}{)}
\PYG{n}{y} \PYG{o}{=} \PYG{n}{np}\PYG{o}{.}\PYG{n}{sin}\PYG{p}{(}\PYG{n}{x}\PYG{p}{)}

\PYG{c+c1}{\PYGZsh{} 通过设置线型为点来完成三点图的显示}
\PYG{n}{plt}\PYG{o}{.}\PYG{n}{plot}\PYG{p}{(}\PYG{n}{x}\PYG{p}{,} \PYG{n}{y}\PYG{p}{,} \PYG{l+s+s1}{\PYGZsq{}}\PYG{l+s+s1}{o}\PYG{l+s+s1}{\PYGZsq{}}\PYG{p}{,} \PYG{n}{color}\PYG{o}{=}\PYG{l+s+s1}{\PYGZsq{}}\PYG{l+s+s1}{blue}\PYG{l+s+s1}{\PYGZsq{}}\PYG{p}{)}
\end{sphinxVerbatim}

\fvset{hllines={, ,}}%
\begin{sphinxVerbatim}[commandchars=\\\{\}]
\PYG{p}{[}\PYG{o}{\PYGZlt{}}\PYG{n}{matplotlib}\PYG{o}{.}\PYG{n}{lines}\PYG{o}{.}\PYG{n}{Line2D} \PYG{n}{at} \PYG{l+m+mh}{0x7fe668f962b0}\PYG{o}{\PYGZgt{}}\PYG{p}{]}
\end{sphinxVerbatim}

\sphinxincludegraphics{{output_43_1}.png}


\subsection{其他散点图形状}
\label{\detokenize{_u6563_u70b9_u56fe:id3}}
\fvset{hllines={, ,}}%
\begin{sphinxVerbatim}[commandchars=\\\{\}]
\PYG{c+c1}{\PYGZsh{} 散点图的形状展示}

\PYG{n}{rng} \PYG{o}{=} \PYG{n}{np}\PYG{o}{.}\PYG{n}{random}\PYG{o}{.}\PYG{n}{RandomState}\PYG{p}{(}\PYG{l+m+mi}{0}\PYG{p}{)}

\PYG{k}{for} \PYG{n}{marker} \PYG{o+ow}{in} \PYG{p}{[}\PYG{l+s+s1}{\PYGZsq{}}\PYG{l+s+s1}{o}\PYG{l+s+s1}{\PYGZsq{}}\PYG{p}{,} \PYG{l+s+s1}{\PYGZsq{}}\PYG{l+s+s1}{.}\PYG{l+s+s1}{\PYGZsq{}}\PYG{p}{,} \PYG{l+s+s1}{\PYGZsq{}}\PYG{l+s+s1}{x}\PYG{l+s+s1}{\PYGZsq{}}\PYG{p}{,} \PYG{l+s+s1}{\PYGZsq{}}\PYG{l+s+s1}{+}\PYG{l+s+s1}{\PYGZsq{}}\PYG{p}{,} \PYG{l+s+s1}{\PYGZsq{}}\PYG{l+s+s1}{\PYGZca{}}\PYG{l+s+s1}{\PYGZsq{}}\PYG{p}{,} \PYG{l+s+s1}{\PYGZsq{}}\PYG{l+s+s1}{\PYGZlt{}}\PYG{l+s+s1}{\PYGZsq{}}\PYG{p}{,} \PYG{l+s+s1}{\PYGZsq{}}\PYG{l+s+s1}{s}\PYG{l+s+s1}{\PYGZsq{}}\PYG{p}{,} \PYG{l+s+s1}{\PYGZsq{}}\PYG{l+s+s1}{d}\PYG{l+s+s1}{\PYGZsq{}}\PYG{p}{]}\PYG{p}{:}
    \PYG{n}{plt}\PYG{o}{.}\PYG{n}{plot}\PYG{p}{(}\PYG{n}{rng}\PYG{o}{.}\PYG{n}{rand}\PYG{p}{(}\PYG{l+m+mi}{5}\PYG{p}{)}\PYG{p}{,} \PYG{n}{rng}\PYG{o}{.}\PYG{n}{rand}\PYG{p}{(}\PYG{l+m+mi}{5}\PYG{p}{)}\PYG{p}{,} \PYG{n}{marker}\PYG{p}{,} \PYG{n}{label}\PYG{o}{=}\PYG{l+s+s1}{\PYGZsq{}}\PYG{l+s+s1}{marker=\PYGZob{}\PYGZcb{}}\PYG{l+s+s1}{\PYGZsq{}}\PYG{o}{.}\PYG{n}{format}\PYG{p}{(}\PYG{n}{marker}\PYG{p}{)}\PYG{p}{)}
    \PYG{n}{plt}\PYG{o}{.}\PYG{n}{legend}\PYG{p}{(}\PYG{n}{numpoints}\PYG{o}{=}\PYG{l+m+mi}{1}\PYG{p}{)}
    \PYG{n}{plt}\PYG{o}{.}\PYG{n}{xlim}\PYG{p}{(}\PYG{l+m+mi}{0}\PYG{p}{,} \PYG{l+m+mf}{1.8}\PYG{p}{)}
\end{sphinxVerbatim}

\sphinxincludegraphics{{output_45_0}.png}


\subsection{点线结合的图}
\label{\detokenize{_u6563_u70b9_u56fe:id4}}
在plot的使用中,对线的类型使用直线(-),圆圈(o)可以画出带有点线结合的图形。

\fvset{hllines={, ,}}%
\begin{sphinxVerbatim}[commandchars=\\\{\}]
\PYG{n}{x} \PYG{o}{=} \PYG{n}{np}\PYG{o}{.}\PYG{n}{linspace}\PYG{p}{(}\PYG{l+m+mi}{0}\PYG{p}{,}\PYG{l+m+mi}{10}\PYG{p}{,} \PYG{l+m+mi}{30}\PYG{p}{)}
\PYG{n}{y} \PYG{o}{=} \PYG{n}{np}\PYG{o}{.}\PYG{n}{sin}\PYG{p}{(}\PYG{n}{x}\PYG{p}{)}

\PYG{c+c1}{\PYGZsh{} 通过设置线型为点来完成三点图的显示}
\PYG{n}{plt}\PYG{o}{.}\PYG{n}{plot}\PYG{p}{(}\PYG{n}{x}\PYG{p}{,} \PYG{n}{y}\PYG{p}{,} \PYG{l+s+s1}{\PYGZsq{}}\PYG{l+s+s1}{\PYGZhy{}o}\PYG{l+s+s1}{\PYGZsq{}}\PYG{p}{,} \PYG{n}{color}\PYG{o}{=}\PYG{l+s+s1}{\PYGZsq{}}\PYG{l+s+s1}{blue}\PYG{l+s+s1}{\PYGZsq{}}\PYG{p}{)}
\end{sphinxVerbatim}

\fvset{hllines={, ,}}%
\begin{sphinxVerbatim}[commandchars=\\\{\}]
\PYG{p}{[}\PYG{o}{\PYGZlt{}}\PYG{n}{matplotlib}\PYG{o}{.}\PYG{n}{lines}\PYG{o}{.}\PYG{n}{Line2D} \PYG{n}{at} \PYG{l+m+mh}{0x7fe6692aa748}\PYG{o}{\PYGZgt{}}\PYG{p}{]}
\end{sphinxVerbatim}

\sphinxincludegraphics{{output_47_1}.png}


\subsection{使用plt.scatter画散点图}
\label{\detokenize{_u6563_u70b9_u56fe:plt-scatter}}
另一个画散点图的函数是scatter,用法和plot函数类似。

但scatter更加灵活,甚至可以单独控制每个散点不同的属性,例如大小,颜色,边控等。

相对来讲,对于大量数据的渲染,plot效率要高于scatter。

\fvset{hllines={, ,}}%
\begin{sphinxVerbatim}[commandchars=\\\{\}]
\PYG{c+c1}{\PYGZsh{} scatter案例}

\PYG{n}{rng} \PYG{o}{=} \PYG{n}{np}\PYG{o}{.}\PYG{n}{random}\PYG{o}{.}\PYG{n}{RandomState}\PYG{p}{(}\PYG{l+m+mi}{0}\PYG{p}{)}
\PYG{n}{x} \PYG{o}{=} \PYG{n}{rng}\PYG{o}{.}\PYG{n}{randn}\PYG{p}{(}\PYG{l+m+mi}{100}\PYG{p}{)}
\PYG{n}{y} \PYG{o}{=} \PYG{n}{rng}\PYG{o}{.}\PYG{n}{randn}\PYG{p}{(}\PYG{l+m+mi}{100}\PYG{p}{)}

\PYG{n}{colors} \PYG{o}{=} \PYG{n}{rng}\PYG{o}{.}\PYG{n}{rand}\PYG{p}{(}\PYG{l+m+mi}{100}\PYG{p}{)}
\PYG{n}{sizes} \PYG{o}{=} \PYG{l+m+mi}{1000} \PYG{o}{*} \PYG{n}{rng}\PYG{o}{.}\PYG{n}{rand}\PYG{p}{(}\PYG{l+m+mi}{100}\PYG{p}{)}
\PYG{n}{plt}\PYG{o}{.}\PYG{n}{scatter}\PYG{p}{(}\PYG{n}{x}\PYG{p}{,} \PYG{n}{y}\PYG{p}{,} \PYG{n}{c}\PYG{o}{=}\PYG{n}{colors}\PYG{p}{,} \PYG{n}{s}\PYG{o}{=}\PYG{n}{sizes}\PYG{p}{,} \PYG{n}{alpha}\PYG{o}{=}\PYG{l+m+mf}{0.4}\PYG{p}{,} \PYG{n}{cmap}\PYG{o}{=}\PYG{l+s+s1}{\PYGZsq{}}\PYG{l+s+s1}{viridis}\PYG{l+s+s1}{\PYGZsq{}}\PYG{p}{)}

\PYG{c+c1}{\PYGZsh{}显示颜色条}
\PYG{n}{plt}\PYG{o}{.}\PYG{n}{colorbar}\PYG{p}{(}\PYG{p}{)}
\end{sphinxVerbatim}

\fvset{hllines={, ,}}%
\begin{sphinxVerbatim}[commandchars=\\\{\}]
\PYG{o}{\PYGZlt{}}\PYG{n}{matplotlib}\PYG{o}{.}\PYG{n}{colorbar}\PYG{o}{.}\PYG{n}{Colorbar} \PYG{n}{at} \PYG{l+m+mh}{0x7fe668405438}\PYG{o}{\PYGZgt{}}
\end{sphinxVerbatim}

\sphinxincludegraphics{{output_49_1}.png}


\section{误差线}
\label{\detokenize{_u8bef_u5dee_u7ebf:id1}}\label{\detokenize{_u8bef_u5dee_u7ebf::doc}}
通过对误差线的绘制,可以直观反映出数据的误差大小等。

\fvset{hllines={, ,}}%
\begin{sphinxVerbatim}[commandchars=\\\{\}]
\PYG{c+c1}{\PYGZsh{}准备环境}
\PYG{o}{\PYGZpc{}}\PYG{n}{matplotlib} \PYG{n}{inline}
\PYG{k+kn}{import} \PYG{n+nn}{matplotlib.pyplot} \PYG{k+kn}{as} \PYG{n+nn}{plt}
\PYG{k+kn}{import} \PYG{n+nn}{numpy} \PYG{k+kn}{as} \PYG{n+nn}{np}

\PYG{c+c1}{\PYGZsh{} 设置风格}
\PYG{n}{plt}\PYG{o}{.}\PYG{n}{style}\PYG{o}{.}\PYG{n}{use}\PYG{p}{(}\PYG{l+s+s1}{\PYGZsq{}}\PYG{l+s+s1}{seaborn\PYGZhy{}whitegrid}\PYG{l+s+s1}{\PYGZsq{}}\PYG{p}{)}
\end{sphinxVerbatim}


\subsection{基本误差线}
\label{\detokenize{_u8bef_u5dee_u7ebf:id2}}
误差线使用函数plt.errorbar来创建,可以使用不同的参数进行配置。
\begin{itemize}
\item {} 
ecolor: 控制误差线颜色

\item {} 
fmt:线型,代码与plot线型控制参数一致

\end{itemize}

\fvset{hllines={, ,}}%
\begin{sphinxVerbatim}[commandchars=\\\{\}]
\PYG{c+c1}{\PYGZsh{} 基本误差线}
\PYG{n}{x} \PYG{o}{=} \PYG{n}{np}\PYG{o}{.}\PYG{n}{linspace}\PYG{p}{(}\PYG{l+m+mi}{0}\PYG{p}{,} \PYG{l+m+mi}{10}\PYG{p}{,} \PYG{l+m+mi}{50}\PYG{p}{)}
\PYG{n}{dy} \PYG{o}{=} \PYG{n}{x} \PYG{o}{*} \PYG{l+m+mf}{0.7}

\PYG{n}{y} \PYG{o}{=} \PYG{n}{np}\PYG{o}{.}\PYG{n}{sin}\PYG{p}{(}\PYG{n}{x}\PYG{p}{)} \PYG{o}{+} \PYG{n}{dy}

\PYG{n}{plt}\PYG{o}{.}\PYG{n}{errorbar}\PYG{p}{(}\PYG{n}{x}\PYG{p}{,} \PYG{n}{y}\PYG{p}{,} \PYG{n}{yerr}\PYG{o}{=}\PYG{n}{dy}\PYG{p}{,} \PYG{n}{fmt}\PYG{o}{=}\PYG{l+s+s1}{\PYGZsq{}}\PYG{l+s+s1}{.k}\PYG{l+s+s1}{\PYGZsq{}}\PYG{p}{,} \PYG{n}{ecolor}\PYG{o}{=}\PYG{l+s+s1}{\PYGZsq{}}\PYG{l+s+s1}{blue}\PYG{l+s+s1}{\PYGZsq{}}\PYG{p}{)}
\end{sphinxVerbatim}

\fvset{hllines={, ,}}%
\begin{sphinxVerbatim}[commandchars=\\\{\}]
\PYG{o}{\PYGZlt{}}\PYG{n}{ErrorbarContainer} \PYG{n+nb}{object} \PYG{n}{of} \PYG{l+m+mi}{3} \PYG{n}{artists}\PYG{o}{\PYGZgt{}}
\end{sphinxVerbatim}

\sphinxincludegraphics{{output_53_1}.png}

\fvset{hllines={, ,}}%
\begin{sphinxVerbatim}[commandchars=\\\{\}]
\PYG{n}{x} \PYG{o}{=} \PYG{n}{np}\PYG{o}{.}\PYG{n}{linspace}\PYG{p}{(}\PYG{l+m+mi}{0}\PYG{p}{,} \PYG{l+m+mi}{10}\PYG{p}{,} \PYG{l+m+mi}{50}\PYG{p}{)}
\PYG{n}{dy} \PYG{o}{=}  \PYG{l+m+mf}{0.7}

\PYG{n}{y} \PYG{o}{=} \PYG{n}{np}\PYG{o}{.}\PYG{n}{sin}\PYG{p}{(}\PYG{n}{x}\PYG{p}{)} \PYG{o}{+} \PYG{n}{dy} \PYG{o}{*} \PYG{n}{np}\PYG{o}{.}\PYG{n}{random}\PYG{o}{.}\PYG{n}{rand}\PYG{p}{(}\PYG{l+m+mi}{50}\PYG{p}{)}

\PYG{n}{plt}\PYG{o}{.}\PYG{n}{errorbar}\PYG{p}{(}\PYG{n}{x}\PYG{p}{,} \PYG{n}{y}\PYG{p}{,} \PYG{n}{yerr}\PYG{o}{=}\PYG{n}{dy}\PYG{p}{,} \PYG{n}{fmt}\PYG{o}{=}\PYG{l+s+s1}{\PYGZsq{}}\PYG{l+s+s1}{o}\PYG{l+s+s1}{\PYGZsq{}}\PYG{p}{,} \PYG{n}{ecolor}\PYG{o}{=}\PYG{l+s+s1}{\PYGZsq{}}\PYG{l+s+s1}{blue}\PYG{l+s+s1}{\PYGZsq{}}\PYG{p}{,} \PYG{n}{color}\PYG{o}{=}\PYG{l+s+s1}{\PYGZsq{}}\PYG{l+s+s1}{red}\PYG{l+s+s1}{\PYGZsq{}}\PYG{p}{,} \PYG{n}{elinewidth}\PYG{o}{=}\PYG{l+m+mi}{3}\PYG{p}{,} \PYG{n}{capsize}\PYG{o}{=}\PYG{l+m+mi}{1}\PYG{p}{)}
\end{sphinxVerbatim}

\fvset{hllines={, ,}}%
\begin{sphinxVerbatim}[commandchars=\\\{\}]
\PYG{o}{\PYGZlt{}}\PYG{n}{ErrorbarContainer} \PYG{n+nb}{object} \PYG{n}{of} \PYG{l+m+mi}{3} \PYG{n}{artists}\PYG{o}{\PYGZgt{}}
\end{sphinxVerbatim}

\sphinxincludegraphics{{output_54_1}.png}


\subsection{连续误差}
\label{\detokenize{_u8bef_u5dee_u7ebf:id3}}
连续误差表示的是连续量,没有比较合适的简单方法来绘制此类型图形,我们可以使用plt.plot和plt.fill\_between来解决,即画出两条区间线表示上下限,然后填充中间区域即可。

下面我们对sin和cos进行简单绘制,绘制后填充两个的中间差值。

\fvset{hllines={, ,}}%
\begin{sphinxVerbatim}[commandchars=\\\{\}]
 

\PYG{n}{x} \PYG{o}{=} \PYG{n}{np}\PYG{o}{.}\PYG{n}{linspace}\PYG{p}{(}\PYG{l+m+mi}{0}\PYG{p}{,}\PYG{l+m+mi}{10}\PYG{p}{,} \PYG{l+m+mi}{50}\PYG{p}{)}
\PYG{n}{ysin} \PYG{o}{=} \PYG{n}{np}\PYG{o}{.}\PYG{n}{sin}\PYG{p}{(}\PYG{n}{x}\PYG{p}{)}
\PYG{n}{ycos} \PYG{o}{=} \PYG{n}{np}\PYG{o}{.}\PYG{n}{cos}\PYG{p}{(}\PYG{n}{x}\PYG{p}{)}

\PYG{n}{plt}\PYG{o}{.}\PYG{n}{plot}\PYG{p}{(}\PYG{n}{x}\PYG{p}{,} \PYG{n}{ysin}\PYG{p}{,} \PYG{n}{color}\PYG{o}{=}\PYG{l+s+s1}{\PYGZsq{}}\PYG{l+s+s1}{red}\PYG{l+s+s1}{\PYGZsq{}}\PYG{p}{)}
\PYG{n}{plt}\PYG{o}{.}\PYG{n}{plot}\PYG{p}{(}\PYG{n}{x}\PYG{p}{,} \PYG{n}{ycos}\PYG{p}{,} \PYG{n}{color}\PYG{o}{=}\PYG{l+s+s1}{\PYGZsq{}}\PYG{l+s+s1}{blue}\PYG{l+s+s1}{\PYGZsq{}}\PYG{p}{)}

\PYG{n}{plt}\PYG{o}{.}\PYG{n}{fill\PYGZus{}between}\PYG{p}{(}\PYG{n}{x}\PYG{p}{,} \PYG{n}{ysin}\PYG{p}{,} \PYG{n}{ycos}\PYG{p}{,} \PYG{n}{color}\PYG{o}{=}\PYG{l+s+s1}{\PYGZsq{}}\PYG{l+s+s1}{gray}\PYG{l+s+s1}{\PYGZsq{}}\PYG{p}{,} \PYG{n}{alpha}\PYG{o}{=}\PYG{l+m+mf}{0.2}\PYG{p}{)}
\end{sphinxVerbatim}

\fvset{hllines={, ,}}%
\begin{sphinxVerbatim}[commandchars=\\\{\}]
\PYG{o}{\PYGZlt{}}\PYG{n}{matplotlib}\PYG{o}{.}\PYG{n}{collections}\PYG{o}{.}\PYG{n}{PolyCollection} \PYG{n}{at} \PYG{l+m+mh}{0x7fe65b2eb860}\PYG{o}{\PYGZgt{}}
\end{sphinxVerbatim}

\sphinxincludegraphics{{output_56_1}.png}


\section{密度图和等高线}
\label{\detokenize{_u5bc6_u5ea6_u56fe_u548c_u7b49_u9ad8_u7ebf:id1}}\label{\detokenize{_u5bc6_u5ea6_u56fe_u548c_u7b49_u9ad8_u7ebf::doc}}
等高线或者密度图使我们常用图形, Matplotlib提供三个函数来供我们使用:
\begin{itemize}
\item {} 
plt.contour: 等高线

\item {} 
plt.contourf: 自带填充色

\item {} 
plt.imshow: 显示图形

\end{itemize}

具体使用请参照下面例子:

\fvset{hllines={, ,}}%
\begin{sphinxVerbatim}[commandchars=\\\{\}]
\PYG{c+c1}{\PYGZsh{}准备环境}
\PYG{o}{\PYGZpc{}}\PYG{n}{matplotlib} \PYG{n}{inline}
\PYG{k+kn}{import} \PYG{n+nn}{matplotlib.pyplot} \PYG{k+kn}{as} \PYG{n+nn}{plt}
\PYG{k+kn}{import} \PYG{n+nn}{numpy} \PYG{k+kn}{as} \PYG{n+nn}{np}

\PYG{c+c1}{\PYGZsh{} 设置风格}
\PYG{n}{plt}\PYG{o}{.}\PYG{n}{style}\PYG{o}{.}\PYG{n}{use}\PYG{p}{(}\PYG{l+s+s1}{\PYGZsq{}}\PYG{l+s+s1}{seaborn\PYGZhy{}whitegrid}\PYG{l+s+s1}{\PYGZsq{}}\PYG{p}{)}
\end{sphinxVerbatim}


\subsection{contour}
\label{\detokenize{_u5bc6_u5ea6_u56fe_u548c_u7b49_u9ad8_u7ebf:contour}}
我们需要一个三维函数,z=f(x,y)来演示等高线图,按照下面函数来进行生成.

contour创建需要至少三个参数,x,y和z,其中x,y我们可以用横轴纵轴表示,z用等高线来表示就可以。当只有一个颜色的图形是,虚线表示负值,实现部分表示正值。

我们使用meshgrid来从一维数据构成二维网格数据。

\fvset{hllines={, ,}}%
\begin{sphinxVerbatim}[commandchars=\\\{\}]
\PYG{c+c1}{\PYGZsh{}函数}
\PYG{k}{def} \PYG{n+nf}{f}\PYG{p}{(}\PYG{n}{x}\PYG{p}{,} \PYG{n}{y}\PYG{p}{)}\PYG{p}{:}
    \PYG{k}{return} \PYG{n}{np}\PYG{o}{.}\PYG{n}{sin}\PYG{p}{(}\PYG{n}{x}\PYG{p}{)} \PYG{o}{*}\PYG{o}{*} \PYG{l+m+mi}{10} \PYG{o}{+} \PYG{n}{np}\PYG{o}{.}\PYG{n}{cos}\PYG{p}{(}\PYG{l+m+mi}{10} \PYG{o}{+} \PYG{n}{y}\PYG{o}{*}\PYG{n}{x}\PYG{p}{)} \PYG{o}{*} \PYG{n}{np}\PYG{o}{.}\PYG{n}{cos}\PYG{p}{(}\PYG{n}{x}\PYG{p}{)}

\PYG{n}{x} \PYG{o}{=} \PYG{n}{np}\PYG{o}{.}\PYG{n}{linspace}\PYG{p}{(}\PYG{l+m+mi}{0}\PYG{p}{,} \PYG{l+m+mi}{5}\PYG{p}{,} \PYG{l+m+mi}{50}\PYG{p}{)}
\PYG{n}{y} \PYG{o}{=} \PYG{n}{np}\PYG{o}{.}\PYG{n}{linspace}\PYG{p}{(}\PYG{l+m+mi}{0}\PYG{p}{,} \PYG{l+m+mi}{5}\PYG{p}{,} \PYG{l+m+mi}{40}\PYG{p}{)}

\PYG{c+c1}{\PYGZsh{}得到网格点矩阵}
\PYG{n}{x}\PYG{p}{,} \PYG{n}{y} \PYG{o}{=} \PYG{n}{np}\PYG{o}{.}\PYG{n}{meshgrid}\PYG{p}{(}\PYG{n}{x}\PYG{p}{,} \PYG{n}{y}\PYG{p}{)}

\PYG{c+c1}{\PYGZsh{} 计算z轴的值}
\PYG{n}{z} \PYG{o}{=} \PYG{n}{f}\PYG{p}{(}\PYG{n}{x}\PYG{p}{,}\PYG{n}{y}\PYG{p}{)}

\PYG{c+c1}{\PYGZsh{}绘制图形}
\PYG{n}{plt}\PYG{o}{.}\PYG{n}{contour}\PYG{p}{(}\PYG{n}{x}\PYG{p}{,} \PYG{n}{y}\PYG{p}{,} \PYG{n}{z}\PYG{p}{,} \PYG{n}{colors}\PYG{o}{=}\PYG{l+s+s1}{\PYGZsq{}}\PYG{l+s+s1}{green}\PYG{l+s+s1}{\PYGZsq{}}\PYG{p}{)}
\end{sphinxVerbatim}

\fvset{hllines={, ,}}%
\begin{sphinxVerbatim}[commandchars=\\\{\}]
\PYG{o}{\PYGZlt{}}\PYG{n}{matplotlib}\PYG{o}{.}\PYG{n}{contour}\PYG{o}{.}\PYG{n}{QuadContourSet} \PYG{n}{at} \PYG{l+m+mh}{0x7f3879e8cdd8}\PYG{o}{\PYGZgt{}}
\end{sphinxVerbatim}

\sphinxincludegraphics{{output_60_1}.png}

\fvset{hllines={, ,}}%
\begin{sphinxVerbatim}[commandchars=\\\{\}]
\PYG{c+c1}{\PYGZsh{}绘制图形}
\PYG{c+c1}{\PYGZsh{}使用红灰色配色方案}
\PYG{c+c1}{\PYGZsh{}把值范围50等分}
\PYG{n}{plt}\PYG{o}{.}\PYG{n}{contour}\PYG{p}{(}\PYG{n}{x}\PYG{p}{,} \PYG{n}{y}\PYG{p}{,} \PYG{n}{z}\PYG{p}{,} \PYG{l+m+mi}{50}\PYG{p}{,} \PYG{n}{cmap}\PYG{o}{=}\PYG{l+s+s1}{\PYGZsq{}}\PYG{l+s+s1}{RdGy}\PYG{l+s+s1}{\PYGZsq{}}\PYG{p}{)}
\end{sphinxVerbatim}

\fvset{hllines={, ,}}%
\begin{sphinxVerbatim}[commandchars=\\\{\}]
\PYG{o}{\PYGZlt{}}\PYG{n}{matplotlib}\PYG{o}{.}\PYG{n}{contour}\PYG{o}{.}\PYG{n}{QuadContourSet} \PYG{n}{at} \PYG{l+m+mh}{0x7fe65b190550}\PYG{o}{\PYGZgt{}}
\end{sphinxVerbatim}

\sphinxincludegraphics{{output_61_1}.png}


\subsection{plt.contourf}
\label{\detokenize{_u5bc6_u5ea6_u56fe_u548c_u7b49_u9ad8_u7ebf:plt-contourf}}
以上绘图还是存在比如间隙过大的问题,我们可以用连续的颜色来填充图形,让它变的平滑起来。

plt.contourf可以满足我们的需求,其余填充参数基本同plt.contour一致。

\fvset{hllines={, ,}}%
\begin{sphinxVerbatim}[commandchars=\\\{\}]
\PYG{c+c1}{\PYGZsh{}绘制图形}
\PYG{c+c1}{\PYGZsh{}平滑过度色彩}
\PYG{n}{plt}\PYG{o}{.}\PYG{n}{contourf}\PYG{p}{(}\PYG{n}{x}\PYG{p}{,} \PYG{n}{y}\PYG{p}{,} \PYG{n}{z}\PYG{p}{,} \PYG{l+m+mi}{50}\PYG{p}{,} \PYG{n}{cmap}\PYG{o}{=}\PYG{l+s+s1}{\PYGZsq{}}\PYG{l+s+s1}{RdGy}\PYG{l+s+s1}{\PYGZsq{}}\PYG{p}{)}
\end{sphinxVerbatim}

\fvset{hllines={, ,}}%
\begin{sphinxVerbatim}[commandchars=\\\{\}]
\PYG{o}{\PYGZlt{}}\PYG{n}{matplotlib}\PYG{o}{.}\PYG{n}{contour}\PYG{o}{.}\PYG{n}{QuadContourSet} \PYG{n}{at} \PYG{l+m+mh}{0x7fe65b017908}\PYG{o}{\PYGZgt{}}
\end{sphinxVerbatim}

\sphinxincludegraphics{{output_63_1}.png}


\subsection{plt.imshow}
\label{\detokenize{_u5bc6_u5ea6_u56fe_u548c_u7b49_u9ad8_u7ebf:plt-imshow}}
上述图形的显示色彩过度还是不够细腻,因为画上面的图的时候使用的是一条一条的线来绘制,虽然可以通过缩小间隙来让图形更加细腻,但是这样会造成计算资源的过度浪费,Matplotlib为我们提供了imshow来完成渐变图的渲染。
\begin{itemize}
\item {} 
plt.imshow函数不支持x,y轴的设置,必须通过extent参数来完成设置,extent={[}xmin, xmax, ymin, ymax{]}

\item {} 
plt.imshow默认以右上角为坐标原点,一般我们使用左下角为坐标原点

\item {} 
plt.imshow自动调整坐标轴精度来适配数据显示,可以通过plt.axis(aspect=’image’)来设置x,y的单位

\end{itemize}

\fvset{hllines={, ,}}%
\begin{sphinxVerbatim}[commandchars=\\\{\}]
\PYG{c+c1}{\PYGZsh{} imshow}

\PYG{n}{plt}\PYG{o}{.}\PYG{n}{imshow}\PYG{p}{(}\PYG{n}{z}\PYG{p}{,} \PYG{n}{extent}\PYG{o}{=}\PYG{p}{[}\PYG{l+m+mi}{0}\PYG{p}{,} \PYG{l+m+mi}{5}\PYG{p}{,} \PYG{l+m+mi}{0}\PYG{p}{,} \PYG{l+m+mi}{5}\PYG{p}{]}\PYG{p}{,} \PYG{n}{origin}\PYG{o}{=}\PYG{l+s+s1}{\PYGZsq{}}\PYG{l+s+s1}{lower}\PYG{l+s+s1}{\PYGZsq{}}\PYG{p}{,} \PYG{n}{cmap}\PYG{o}{=}\PYG{l+s+s1}{\PYGZsq{}}\PYG{l+s+s1}{RdGy}\PYG{l+s+s1}{\PYGZsq{}}\PYG{p}{)}

\PYG{n}{plt}\PYG{o}{.}\PYG{n}{colorbar}\PYG{p}{(}\PYG{p}{)}
\PYG{n}{plt}\PYG{o}{.}\PYG{n}{axis}\PYG{p}{(}\PYG{n}{aspect}\PYG{o}{=}\PYG{l+s+s1}{\PYGZsq{}}\PYG{l+s+s1}{image}\PYG{l+s+s1}{\PYGZsq{}}\PYG{p}{)}
\end{sphinxVerbatim}

\fvset{hllines={, ,}}%
\begin{sphinxVerbatim}[commandchars=\\\{\}]
\PYG{p}{(}\PYG{l+m+mf}{0.0}\PYG{p}{,} \PYG{l+m+mf}{5.0}\PYG{p}{,} \PYG{l+m+mf}{0.0}\PYG{p}{,} \PYG{l+m+mf}{5.0}\PYG{p}{)}
\end{sphinxVerbatim}

\sphinxincludegraphics{{output_65_1}.png}

\fvset{hllines={, ,}}%
\begin{sphinxVerbatim}[commandchars=\\\{\}]
\PYG{c+c1}{\PYGZsh{} 显示等高线的同时通过颜色显示内容}

\PYG{n}{contours} \PYG{o}{=} \PYG{n}{plt}\PYG{o}{.}\PYG{n}{contour}\PYG{p}{(}\PYG{n}{x}\PYG{p}{,} \PYG{n}{y}\PYG{p}{,} \PYG{n}{z}\PYG{p}{,} \PYG{l+m+mi}{3}\PYG{p}{,} \PYG{n}{colors}\PYG{o}{=}\PYG{l+s+s2}{\PYGZdq{}}\PYG{l+s+s2}{green}\PYG{l+s+s2}{\PYGZdq{}}\PYG{p}{)}
\PYG{n}{plt}\PYG{o}{.}\PYG{n}{clabel}\PYG{p}{(}\PYG{n}{contours}\PYG{p}{,} \PYG{n}{inline}\PYG{o}{=}\PYG{n+nb+bp}{True}\PYG{p}{,} \PYG{n}{fontsize}\PYG{o}{=}\PYG{l+m+mi}{8}\PYG{p}{)}


\PYG{n}{plt}\PYG{o}{.}\PYG{n}{imshow}\PYG{p}{(}\PYG{n}{z}\PYG{p}{,} \PYG{n}{extent}\PYG{o}{=}\PYG{p}{[}\PYG{l+m+mi}{0}\PYG{p}{,} \PYG{l+m+mi}{5}\PYG{p}{,} \PYG{l+m+mi}{0}\PYG{p}{,} \PYG{l+m+mi}{5}\PYG{p}{]}\PYG{p}{,} \PYG{n}{origin}\PYG{o}{=}\PYG{l+s+s1}{\PYGZsq{}}\PYG{l+s+s1}{lower}\PYG{l+s+s1}{\PYGZsq{}}\PYG{p}{,} \PYG{n}{cmap}\PYG{o}{=}\PYG{l+s+s1}{\PYGZsq{}}\PYG{l+s+s1}{RdGy}\PYG{l+s+s1}{\PYGZsq{}}\PYG{p}{,} \PYG{n}{alpha}\PYG{o}{=}\PYG{l+m+mf}{0.2}\PYG{p}{)}

\PYG{n}{plt}\PYG{o}{.}\PYG{n}{colorbar}\PYG{p}{(}\PYG{p}{)}
\PYG{n}{plt}\PYG{o}{.}\PYG{n}{axis}\PYG{p}{(}\PYG{n}{aspect}\PYG{o}{=}\PYG{l+s+s1}{\PYGZsq{}}\PYG{l+s+s1}{image}\PYG{l+s+s1}{\PYGZsq{}}\PYG{p}{)}
\end{sphinxVerbatim}

\fvset{hllines={, ,}}%
\begin{sphinxVerbatim}[commandchars=\\\{\}]
\PYG{p}{(}\PYG{l+m+mf}{0.0}\PYG{p}{,} \PYG{l+m+mf}{5.0}\PYG{p}{,} \PYG{l+m+mf}{0.0}\PYG{p}{,} \PYG{l+m+mf}{5.0}\PYG{p}{)}
\end{sphinxVerbatim}

\sphinxincludegraphics{{output_66_1}.png}


\section{频次直方图,数据区间划分和分布密度}
\label{\detokenize{_u76f4_u65b9_u56fe:id1}}\label{\detokenize{_u76f4_u65b9_u56fe::doc}}
\fvset{hllines={, ,}}%
\begin{sphinxVerbatim}[commandchars=\\\{\}]
\PYG{c+c1}{\PYGZsh{}准备环境}
\PYG{o}{\PYGZpc{}}\PYG{n}{matplotlib} \PYG{n}{inline}
\PYG{k+kn}{import} \PYG{n+nn}{matplotlib.pyplot} \PYG{k+kn}{as} \PYG{n+nn}{plt}
\PYG{k+kn}{import} \PYG{n+nn}{numpy} \PYG{k+kn}{as} \PYG{n+nn}{np}

\PYG{c+c1}{\PYGZsh{} 设置风格}
\PYG{n}{plt}\PYG{o}{.}\PYG{n}{style}\PYG{o}{.}\PYG{n}{use}\PYG{p}{(}\PYG{l+s+s1}{\PYGZsq{}}\PYG{l+s+s1}{seaborn\PYGZhy{}whitegrid}\PYG{l+s+s1}{\PYGZsq{}}\PYG{p}{)}
\end{sphinxVerbatim}


\subsection{频次直方图}
\label{\detokenize{_u76f4_u65b9_u56fe:id2}}
使用plt.hist可以画直方图,重要的参数有:
\begin{itemize}
\item {} 
bins: 画几条方图

\item {} 
color: 颜色

\item {} 
alpha: 透明度

\item {} 
histtype: 图类型

\end{itemize}

\fvset{hllines={, ,}}%
\begin{sphinxVerbatim}[commandchars=\\\{\}]
\PYG{n}{data} \PYG{o}{=} \PYG{n}{np}\PYG{o}{.}\PYG{n}{random}\PYG{o}{.}\PYG{n}{randn}\PYG{p}{(}\PYG{l+m+mi}{1000}\PYG{p}{)}
\PYG{n}{plt}\PYG{o}{.}\PYG{n}{hist}\PYG{p}{(}\PYG{n}{data}\PYG{p}{)}
\end{sphinxVerbatim}

\fvset{hllines={, ,}}%
\begin{sphinxVerbatim}[commandchars=\\\{\}]
\PYG{p}{(}\PYG{n}{array}\PYG{p}{(}\PYG{p}{[}  \PYG{l+m+mf}{9.}\PYG{p}{,}  \PYG{l+m+mf}{44.}\PYG{p}{,} \PYG{l+m+mf}{108.}\PYG{p}{,} \PYG{l+m+mf}{178.}\PYG{p}{,} \PYG{l+m+mf}{241.}\PYG{p}{,} \PYG{l+m+mf}{194.}\PYG{p}{,} \PYG{l+m+mf}{134.}\PYG{p}{,}  \PYG{l+m+mf}{67.}\PYG{p}{,}  \PYG{l+m+mf}{20.}\PYG{p}{,}   \PYG{l+m+mf}{5.}\PYG{p}{]}\PYG{p}{)}\PYG{p}{,}
 \PYG{n}{array}\PYG{p}{(}\PYG{p}{[}\PYG{o}{\PYGZhy{}}\PYG{l+m+mf}{2.83511303}\PYG{p}{,} \PYG{o}{\PYGZhy{}}\PYG{l+m+mf}{2.23919796}\PYG{p}{,} \PYG{o}{\PYGZhy{}}\PYG{l+m+mf}{1.64328288}\PYG{p}{,} \PYG{o}{\PYGZhy{}}\PYG{l+m+mf}{1.04736781}\PYG{p}{,} \PYG{o}{\PYGZhy{}}\PYG{l+m+mf}{0.45145274}\PYG{p}{,}
         \PYG{l+m+mf}{0.14446234}\PYG{p}{,}  \PYG{l+m+mf}{0.74037741}\PYG{p}{,}  \PYG{l+m+mf}{1.33629248}\PYG{p}{,}  \PYG{l+m+mf}{1.93220755}\PYG{p}{,}  \PYG{l+m+mf}{2.52812263}\PYG{p}{,}
         \PYG{l+m+mf}{3.1240377} \PYG{p}{]}\PYG{p}{)}\PYG{p}{,}
 \PYG{o}{\PYGZlt{}}\PYG{n}{a} \PYG{n+nb}{list} \PYG{n}{of} \PYG{l+m+mi}{10} \PYG{n}{Patch} \PYG{n}{objects}\PYG{o}{\PYGZgt{}}\PYG{p}{)}
\end{sphinxVerbatim}

\sphinxincludegraphics{{output_70_1}.png}

\fvset{hllines={, ,}}%
\begin{sphinxVerbatim}[commandchars=\\\{\}]
\PYG{n}{plt}\PYG{o}{.}\PYG{n}{hist}\PYG{p}{(}\PYG{n}{data}\PYG{p}{,} \PYG{n}{bins}\PYG{o}{=}\PYG{l+m+mi}{30}\PYG{p}{,} \PYG{n}{alpha}\PYG{o}{=}\PYG{l+m+mf}{0.3}\PYG{p}{,} \PYG{n}{histtype}\PYG{o}{=}\PYG{l+s+s1}{\PYGZsq{}}\PYG{l+s+s1}{stepfilled}\PYG{l+s+s1}{\PYGZsq{}}\PYG{p}{,} \PYG{n}{color}\PYG{o}{=}\PYG{l+s+s1}{\PYGZsq{}}\PYG{l+s+s1}{steelblue}\PYG{l+s+s1}{\PYGZsq{}}\PYG{p}{,} \PYG{n}{edgecolor}\PYG{o}{=}\PYG{l+s+s1}{\PYGZsq{}}\PYG{l+s+s1}{none}\PYG{l+s+s1}{\PYGZsq{}}\PYG{p}{)}
\end{sphinxVerbatim}

\fvset{hllines={, ,}}%
\begin{sphinxVerbatim}[commandchars=\\\{\}]
\PYG{p}{(}\PYG{n}{array}\PYG{p}{(}\PYG{p}{[} \PYG{l+m+mf}{2.}\PYG{p}{,}  \PYG{l+m+mf}{4.}\PYG{p}{,}  \PYG{l+m+mf}{3.}\PYG{p}{,}  \PYG{l+m+mf}{9.}\PYG{p}{,} \PYG{l+m+mf}{19.}\PYG{p}{,} \PYG{l+m+mf}{16.}\PYG{p}{,} \PYG{l+m+mf}{30.}\PYG{p}{,} \PYG{l+m+mf}{32.}\PYG{p}{,} \PYG{l+m+mf}{46.}\PYG{p}{,} \PYG{l+m+mf}{52.}\PYG{p}{,} \PYG{l+m+mf}{52.}\PYG{p}{,} \PYG{l+m+mf}{74.}\PYG{p}{,} \PYG{l+m+mf}{77.}\PYG{p}{,}
        \PYG{l+m+mf}{87.}\PYG{p}{,} \PYG{l+m+mf}{77.}\PYG{p}{,} \PYG{l+m+mf}{83.}\PYG{p}{,} \PYG{l+m+mf}{66.}\PYG{p}{,} \PYG{l+m+mf}{45.}\PYG{p}{,} \PYG{l+m+mf}{56.}\PYG{p}{,} \PYG{l+m+mf}{41.}\PYG{p}{,} \PYG{l+m+mf}{37.}\PYG{p}{,} \PYG{l+m+mf}{29.}\PYG{p}{,} \PYG{l+m+mf}{18.}\PYG{p}{,} \PYG{l+m+mf}{20.}\PYG{p}{,}  \PYG{l+m+mf}{8.}\PYG{p}{,}  \PYG{l+m+mf}{7.}\PYG{p}{,}
         \PYG{l+m+mf}{5.}\PYG{p}{,}  \PYG{l+m+mf}{3.}\PYG{p}{,}  \PYG{l+m+mf}{0.}\PYG{p}{,}  \PYG{l+m+mf}{2.}\PYG{p}{]}\PYG{p}{)}\PYG{p}{,}
 \PYG{n}{array}\PYG{p}{(}\PYG{p}{[}\PYG{o}{\PYGZhy{}}\PYG{l+m+mf}{2.83511303}\PYG{p}{,} \PYG{o}{\PYGZhy{}}\PYG{l+m+mf}{2.63647467}\PYG{p}{,} \PYG{o}{\PYGZhy{}}\PYG{l+m+mf}{2.43783631}\PYG{p}{,} \PYG{o}{\PYGZhy{}}\PYG{l+m+mf}{2.23919796}\PYG{p}{,} \PYG{o}{\PYGZhy{}}\PYG{l+m+mf}{2.0405596} \PYG{p}{,}
        \PYG{o}{\PYGZhy{}}\PYG{l+m+mf}{1.84192124}\PYG{p}{,} \PYG{o}{\PYGZhy{}}\PYG{l+m+mf}{1.64328288}\PYG{p}{,} \PYG{o}{\PYGZhy{}}\PYG{l+m+mf}{1.44464453}\PYG{p}{,} \PYG{o}{\PYGZhy{}}\PYG{l+m+mf}{1.24600617}\PYG{p}{,} \PYG{o}{\PYGZhy{}}\PYG{l+m+mf}{1.04736781}\PYG{p}{,}
        \PYG{o}{\PYGZhy{}}\PYG{l+m+mf}{0.84872945}\PYG{p}{,} \PYG{o}{\PYGZhy{}}\PYG{l+m+mf}{0.6500911} \PYG{p}{,} \PYG{o}{\PYGZhy{}}\PYG{l+m+mf}{0.45145274}\PYG{p}{,} \PYG{o}{\PYGZhy{}}\PYG{l+m+mf}{0.25281438}\PYG{p}{,} \PYG{o}{\PYGZhy{}}\PYG{l+m+mf}{0.05417602}\PYG{p}{,}
         \PYG{l+m+mf}{0.14446234}\PYG{p}{,}  \PYG{l+m+mf}{0.34310069}\PYG{p}{,}  \PYG{l+m+mf}{0.54173905}\PYG{p}{,}  \PYG{l+m+mf}{0.74037741}\PYG{p}{,}  \PYG{l+m+mf}{0.93901577}\PYG{p}{,}
         \PYG{l+m+mf}{1.13765412}\PYG{p}{,}  \PYG{l+m+mf}{1.33629248}\PYG{p}{,}  \PYG{l+m+mf}{1.53493084}\PYG{p}{,}  \PYG{l+m+mf}{1.7335692} \PYG{p}{,}  \PYG{l+m+mf}{1.93220755}\PYG{p}{,}
         \PYG{l+m+mf}{2.13084591}\PYG{p}{,}  \PYG{l+m+mf}{2.32948427}\PYG{p}{,}  \PYG{l+m+mf}{2.52812263}\PYG{p}{,}  \PYG{l+m+mf}{2.72676099}\PYG{p}{,}  \PYG{l+m+mf}{2.92539934}\PYG{p}{,}
         \PYG{l+m+mf}{3.1240377} \PYG{p}{]}\PYG{p}{)}\PYG{p}{,}
 \PYG{o}{\PYGZlt{}}\PYG{n}{a} \PYG{n+nb}{list} \PYG{n}{of} \PYG{l+m+mi}{1} \PYG{n}{Patch} \PYG{n}{objects}\PYG{o}{\PYGZgt{}}\PYG{p}{)}
\end{sphinxVerbatim}

\sphinxincludegraphics{{output_71_1}.png}

\fvset{hllines={, ,}}%
\begin{sphinxVerbatim}[commandchars=\\\{\}]
\PYG{c+c1}{\PYGZsh{}更复杂的案例}

\PYG{n}{x1} \PYG{o}{=} \PYG{n}{np}\PYG{o}{.}\PYG{n}{random}\PYG{o}{.}\PYG{n}{normal}\PYG{p}{(}\PYG{l+m+mi}{0}\PYG{p}{,} \PYG{l+m+mf}{0.8}\PYG{p}{,} \PYG{l+m+mi}{1000}\PYG{p}{)}
\PYG{n}{x2} \PYG{o}{=} \PYG{n}{np}\PYG{o}{.}\PYG{n}{random}\PYG{o}{.}\PYG{n}{normal}\PYG{p}{(}\PYG{o}{\PYGZhy{}}\PYG{l+m+mi}{2}\PYG{p}{,} \PYG{l+m+mi}{1}\PYG{p}{,} \PYG{l+m+mi}{1000}\PYG{p}{)}
\PYG{n}{x3} \PYG{o}{=} \PYG{n}{np}\PYG{o}{.}\PYG{n}{random}\PYG{o}{.}\PYG{n}{normal}\PYG{p}{(}\PYG{l+m+mi}{3}\PYG{p}{,}\PYG{l+m+mi}{2}\PYG{p}{,} \PYG{l+m+mi}{1000}\PYG{p}{)}

\PYG{n}{kwargs} \PYG{o}{=} \PYG{n+nb}{dict}\PYG{p}{(}\PYG{n}{histtype}\PYG{o}{=}\PYG{l+s+s1}{\PYGZsq{}}\PYG{l+s+s1}{stepfilled}\PYG{l+s+s1}{\PYGZsq{}}\PYG{p}{,} \PYG{n}{alpha}\PYG{o}{=}\PYG{l+m+mf}{0.3}\PYG{p}{,} \PYG{n}{bins}\PYG{o}{=}\PYG{l+m+mi}{40}\PYG{p}{)}

\PYG{n}{plt}\PYG{o}{.}\PYG{n}{hist}\PYG{p}{(}\PYG{n}{x1}\PYG{p}{,} \PYG{o}{*}\PYG{o}{*}\PYG{n}{kwargs}\PYG{p}{)}
\PYG{n}{plt}\PYG{o}{.}\PYG{n}{hist}\PYG{p}{(}\PYG{n}{x2}\PYG{p}{,} \PYG{o}{*}\PYG{o}{*}\PYG{n}{kwargs}\PYG{p}{)}
\PYG{n}{plt}\PYG{o}{.}\PYG{n}{hist}\PYG{p}{(}\PYG{n}{x3}\PYG{p}{,} \PYG{o}{*}\PYG{o}{*}\PYG{n}{kwargs}\PYG{p}{)}
\end{sphinxVerbatim}

\fvset{hllines={, ,}}%
\begin{sphinxVerbatim}[commandchars=\\\{\}]
\PYG{p}{(}\PYG{n}{array}\PYG{p}{(}\PYG{p}{[} \PYG{l+m+mf}{1.}\PYG{p}{,}  \PYG{l+m+mf}{0.}\PYG{p}{,}  \PYG{l+m+mf}{1.}\PYG{p}{,}  \PYG{l+m+mf}{2.}\PYG{p}{,}  \PYG{l+m+mf}{5.}\PYG{p}{,}  \PYG{l+m+mf}{2.}\PYG{p}{,}  \PYG{l+m+mf}{6.}\PYG{p}{,}  \PYG{l+m+mf}{5.}\PYG{p}{,}  \PYG{l+m+mf}{8.}\PYG{p}{,} \PYG{l+m+mf}{11.}\PYG{p}{,} \PYG{l+m+mf}{17.}\PYG{p}{,} \PYG{l+m+mf}{23.}\PYG{p}{,} \PYG{l+m+mf}{26.}\PYG{p}{,}
        \PYG{l+m+mf}{32.}\PYG{p}{,} \PYG{l+m+mf}{35.}\PYG{p}{,} \PYG{l+m+mf}{53.}\PYG{p}{,} \PYG{l+m+mf}{67.}\PYG{p}{,} \PYG{l+m+mf}{49.}\PYG{p}{,} \PYG{l+m+mf}{50.}\PYG{p}{,} \PYG{l+m+mf}{53.}\PYG{p}{,} \PYG{l+m+mf}{61.}\PYG{p}{,} \PYG{l+m+mf}{58.}\PYG{p}{,} \PYG{l+m+mf}{67.}\PYG{p}{,} \PYG{l+m+mf}{67.}\PYG{p}{,} \PYG{l+m+mf}{53.}\PYG{p}{,} \PYG{l+m+mf}{45.}\PYG{p}{,}
        \PYG{l+m+mf}{53.}\PYG{p}{,} \PYG{l+m+mf}{32.}\PYG{p}{,} \PYG{l+m+mf}{34.}\PYG{p}{,} \PYG{l+m+mf}{19.}\PYG{p}{,} \PYG{l+m+mf}{25.}\PYG{p}{,} \PYG{l+m+mf}{22.}\PYG{p}{,}  \PYG{l+m+mf}{9.}\PYG{p}{,}  \PYG{l+m+mf}{2.}\PYG{p}{,}  \PYG{l+m+mf}{4.}\PYG{p}{,}  \PYG{l+m+mf}{1.}\PYG{p}{,}  \PYG{l+m+mf}{0.}\PYG{p}{,}  \PYG{l+m+mf}{1.}\PYG{p}{,}  \PYG{l+m+mf}{0.}\PYG{p}{,}
         \PYG{l+m+mf}{1.}\PYG{p}{]}\PYG{p}{)}\PYG{p}{,}
 \PYG{n}{array}\PYG{p}{(}\PYG{p}{[}\PYG{o}{\PYGZhy{}}\PYG{l+m+mf}{3.90483644}\PYG{p}{,} \PYG{o}{\PYGZhy{}}\PYG{l+m+mf}{3.575889}  \PYG{p}{,} \PYG{o}{\PYGZhy{}}\PYG{l+m+mf}{3.24694156}\PYG{p}{,} \PYG{o}{\PYGZhy{}}\PYG{l+m+mf}{2.91799412}\PYG{p}{,} \PYG{o}{\PYGZhy{}}\PYG{l+m+mf}{2.58904668}\PYG{p}{,}
        \PYG{o}{\PYGZhy{}}\PYG{l+m+mf}{2.26009924}\PYG{p}{,} \PYG{o}{\PYGZhy{}}\PYG{l+m+mf}{1.9311518} \PYG{p}{,} \PYG{o}{\PYGZhy{}}\PYG{l+m+mf}{1.60220436}\PYG{p}{,} \PYG{o}{\PYGZhy{}}\PYG{l+m+mf}{1.27325692}\PYG{p}{,} \PYG{o}{\PYGZhy{}}\PYG{l+m+mf}{0.94430948}\PYG{p}{,}
        \PYG{o}{\PYGZhy{}}\PYG{l+m+mf}{0.61536204}\PYG{p}{,} \PYG{o}{\PYGZhy{}}\PYG{l+m+mf}{0.2864146} \PYG{p}{,}  \PYG{l+m+mf}{0.04253285}\PYG{p}{,}  \PYG{l+m+mf}{0.37148029}\PYG{p}{,}  \PYG{l+m+mf}{0.70042773}\PYG{p}{,}
         \PYG{l+m+mf}{1.02937517}\PYG{p}{,}  \PYG{l+m+mf}{1.35832261}\PYG{p}{,}  \PYG{l+m+mf}{1.68727005}\PYG{p}{,}  \PYG{l+m+mf}{2.01621749}\PYG{p}{,}  \PYG{l+m+mf}{2.34516493}\PYG{p}{,}
         \PYG{l+m+mf}{2.67411237}\PYG{p}{,}  \PYG{l+m+mf}{3.00305981}\PYG{p}{,}  \PYG{l+m+mf}{3.33200725}\PYG{p}{,}  \PYG{l+m+mf}{3.66095469}\PYG{p}{,}  \PYG{l+m+mf}{3.98990213}\PYG{p}{,}
         \PYG{l+m+mf}{4.31884957}\PYG{p}{,}  \PYG{l+m+mf}{4.64779701}\PYG{p}{,}  \PYG{l+m+mf}{4.97674445}\PYG{p}{,}  \PYG{l+m+mf}{5.30569189}\PYG{p}{,}  \PYG{l+m+mf}{5.63463934}\PYG{p}{,}
         \PYG{l+m+mf}{5.96358678}\PYG{p}{,}  \PYG{l+m+mf}{6.29253422}\PYG{p}{,}  \PYG{l+m+mf}{6.62148166}\PYG{p}{,}  \PYG{l+m+mf}{6.9504291} \PYG{p}{,}  \PYG{l+m+mf}{7.27937654}\PYG{p}{,}
         \PYG{l+m+mf}{7.60832398}\PYG{p}{,}  \PYG{l+m+mf}{7.93727142}\PYG{p}{,}  \PYG{l+m+mf}{8.26621886}\PYG{p}{,}  \PYG{l+m+mf}{8.5951663} \PYG{p}{,}  \PYG{l+m+mf}{8.92411374}\PYG{p}{,}
         \PYG{l+m+mf}{9.25306118}\PYG{p}{]}\PYG{p}{)}\PYG{p}{,}
 \PYG{o}{\PYGZlt{}}\PYG{n}{a} \PYG{n+nb}{list} \PYG{n}{of} \PYG{l+m+mi}{1} \PYG{n}{Patch} \PYG{n}{objects}\PYG{o}{\PYGZgt{}}\PYG{p}{)}
\end{sphinxVerbatim}

\sphinxincludegraphics{{output_72_1}.png}


\subsection{二维频次直方图和数据区间划分}
\label{\detokenize{_u76f4_u65b9_u56fe:id3}}
\fvset{hllines={, ,}}%
\begin{sphinxVerbatim}[commandchars=\\\{\}]
\PYG{c+c1}{\PYGZsh{} 准备数据}

\PYG{n}{mean} \PYG{o}{=} \PYG{p}{[}\PYG{l+m+mi}{0}\PYG{p}{,} \PYG{l+m+mi}{0}\PYG{p}{]}
\PYG{n}{cov} \PYG{o}{=} \PYG{p}{[}\PYG{p}{[}\PYG{l+m+mi}{1}\PYG{p}{,}\PYG{l+m+mi}{1}\PYG{p}{]}\PYG{p}{,} \PYG{p}{[}\PYG{l+m+mi}{1}\PYG{p}{,}\PYG{l+m+mi}{2}\PYG{p}{]}\PYG{p}{]}

\PYG{c+c1}{\PYGZsh{} 多元正太分布随机样本抽取}
\PYG{c+c1}{\PYGZsh{} mean:多元正态分布的维度}
\PYG{c+c1}{\PYGZsh{} cov:多元正态分布的协方差矩阵,且协方差矩阵必须是对称矩阵和半正定矩阵(形状为(N,N)的二维数组)。}
\PYG{c+c1}{\PYGZsh{} size: 数组的形状(整数或者由整数构成的元组)。如果该值未给定,则返回单个N维的样本(N恰恰是上面mean的长度)。}
\PYG{n}{x}\PYG{p}{,} \PYG{n}{y} \PYG{o}{=} \PYG{n}{np}\PYG{o}{.}\PYG{n}{random}\PYG{o}{.}\PYG{n}{multivariate\PYGZus{}normal}\PYG{p}{(}\PYG{n}{mean}\PYG{p}{,} \PYG{n}{cov}\PYG{p}{,} \PYG{l+m+mi}{10000}\PYG{p}{)}\PYG{o}{.}\PYG{n}{T}
\end{sphinxVerbatim}


\subsection{plt.hist2d}
\label{\detokenize{_u76f4_u65b9_u56fe:plt-hist2d}}
使用plt.hist2d可以画二维直方图。

\fvset{hllines={, ,}}%
\begin{sphinxVerbatim}[commandchars=\\\{\}]
\PYG{n}{plt}\PYG{o}{.}\PYG{n}{hist2d}\PYG{p}{(}\PYG{n}{x}\PYG{p}{,} \PYG{n}{y}\PYG{p}{,} \PYG{n}{bins}\PYG{o}{=}\PYG{l+m+mi}{30}\PYG{p}{,} \PYG{n}{cmap}\PYG{o}{=}\PYG{l+s+s1}{\PYGZsq{}}\PYG{l+s+s1}{Blues}\PYG{l+s+s1}{\PYGZsq{}}\PYG{p}{)}
\PYG{n}{cb} \PYG{o}{=} \PYG{n}{plt}\PYG{o}{.}\PYG{n}{colorbar}\PYG{p}{(}\PYG{p}{)}
\PYG{n}{cb}\PYG{o}{.}\PYG{n}{set\PYGZus{}label}\PYG{p}{(}\PYG{l+s+s2}{\PYGZdq{}}\PYG{l+s+s2}{counts in bin}\PYG{l+s+s2}{\PYGZdq{}}\PYG{p}{)}
\end{sphinxVerbatim}

\sphinxincludegraphics{{output_76_0}.png}


\subsection{plt.hexbin}
\label{\detokenize{_u76f4_u65b9_u56fe:plt-hexbin}}
hist2d是用的方块组成的图形,还可以使用六边形进行图形分割,需要使用plt.hexbin来完成,用来将图形化成六角形的蜂窝。

\fvset{hllines={, ,}}%
\begin{sphinxVerbatim}[commandchars=\\\{\}]
\PYG{n}{plt}\PYG{o}{.}\PYG{n}{hexbin}\PYG{p}{(}\PYG{n}{x}\PYG{p}{,} \PYG{n}{y}\PYG{p}{,} \PYG{n}{gridsize}\PYG{o}{=}\PYG{l+m+mi}{30}\PYG{p}{,} \PYG{n}{color}\PYG{o}{=}\PYG{l+s+s1}{\PYGZsq{}}\PYG{l+s+s1}{red}\PYG{l+s+s1}{\PYGZsq{}}\PYG{p}{)}
\PYG{n}{cb} \PYG{o}{=} \PYG{n}{plt}\PYG{o}{.}\PYG{n}{colorbar}\PYG{p}{(}\PYG{n}{label}\PYG{o}{=}\PYG{l+s+s2}{\PYGZdq{}}\PYG{l+s+s2}{count in bin}\PYG{l+s+s2}{\PYGZdq{}}\PYG{p}{)}
\end{sphinxVerbatim}

\sphinxincludegraphics{{output_78_0}.png}


\subsection{核密度估计}
\label{\detokenize{_u76f4_u65b9_u56fe:id4}}
核密度估计(KernelDensityEstimation)是一种常用的评估多维度分布密度的方法,本节主要是对画图函数做一个展示,不详细讲述kde算法。

kde方法通过不同的平滑带宽长度在拟合函数的准确性和平滑性之间做出一种权衡。

\fvset{hllines={, ,}}%
\begin{sphinxVerbatim}[commandchars=\\\{\}]
\PYG{k+kn}{from} \PYG{n+nn}{scipy.stats} \PYG{k+kn}{import} \PYG{n}{gaussian\PYGZus{}kde}

\PYG{n}{data} \PYG{o}{=} \PYG{n}{np}\PYG{o}{.}\PYG{n}{vstack}\PYG{p}{(}\PYG{p}{[}\PYG{n}{x}\PYG{p}{,} \PYG{n}{y}\PYG{p}{]}\PYG{p}{)}
\PYG{n}{kde} \PYG{o}{=} \PYG{n}{gaussian\PYGZus{}kde}\PYG{p}{(}\PYG{n}{data}\PYG{p}{)}

\PYG{n}{x} \PYG{o}{=} \PYG{n}{np}\PYG{o}{.}\PYG{n}{linspace}\PYG{p}{(}\PYG{o}{\PYGZhy{}}\PYG{l+m+mf}{3.5}\PYG{p}{,} \PYG{l+m+mf}{3.5}\PYG{p}{,} \PYG{l+m+mi}{40}\PYG{p}{)}
\PYG{n}{y} \PYG{o}{=} \PYG{n}{np}\PYG{o}{.}\PYG{n}{linspace}\PYG{p}{(}\PYG{o}{\PYGZhy{}}\PYG{l+m+mi}{6}\PYG{p}{,} \PYG{l+m+mi}{6}\PYG{p}{,} \PYG{l+m+mi}{40}\PYG{p}{)}

\PYG{n}{x}\PYG{p}{,} \PYG{n}{y} \PYG{o}{=} \PYG{n}{np}\PYG{o}{.}\PYG{n}{meshgrid}\PYG{p}{(}\PYG{n}{x}\PYG{p}{,} \PYG{n}{y}\PYG{p}{)}

\PYG{n}{z} \PYG{o}{=} \PYG{n}{kde}\PYG{o}{.}\PYG{n}{evaluate}\PYG{p}{(}\PYG{n}{np}\PYG{o}{.}\PYG{n}{vstack}\PYG{p}{(}\PYG{p}{[}\PYG{n}{x}\PYG{o}{.}\PYG{n}{ravel}\PYG{p}{(}\PYG{p}{)}\PYG{p}{,} \PYG{n}{y}\PYG{o}{.}\PYG{n}{ravel}\PYG{p}{(}\PYG{p}{)}\PYG{p}{]}\PYG{p}{)}\PYG{p}{)}

\PYG{n}{plt}\PYG{o}{.}\PYG{n}{imshow}\PYG{p}{(}\PYG{n}{z}\PYG{o}{.}\PYG{n}{reshape}\PYG{p}{(}\PYG{n}{x}\PYG{o}{.}\PYG{n}{shape}\PYG{p}{)}\PYG{p}{,} \PYG{n}{origin}\PYG{o}{=}\PYG{l+s+s1}{\PYGZsq{}}\PYG{l+s+s1}{lower}\PYG{l+s+s1}{\PYGZsq{}}\PYG{p}{,} \PYG{n}{aspect}\PYG{o}{=}\PYG{l+s+s1}{\PYGZsq{}}\PYG{l+s+s1}{auto}\PYG{l+s+s1}{\PYGZsq{}}\PYG{p}{,} \PYG{n}{extent}\PYG{o}{=}\PYG{p}{[}\PYG{o}{\PYGZhy{}}\PYG{l+m+mf}{3.5}\PYG{p}{,} \PYG{l+m+mf}{3.5}\PYG{p}{,} \PYG{o}{\PYGZhy{}}\PYG{l+m+mi}{6}\PYG{p}{,} \PYG{l+m+mi}{6}\PYG{p}{]}\PYG{p}{,} \PYG{n}{cmap}\PYG{o}{=}\PYG{l+s+s1}{\PYGZsq{}}\PYG{l+s+s1}{Blues}\PYG{l+s+s1}{\PYGZsq{}}\PYG{p}{)}

\PYG{n}{cb} \PYG{o}{=} \PYG{n}{plt}\PYG{o}{.}\PYG{n}{colorbar}\PYG{p}{(}\PYG{p}{)}
\PYG{n}{cb}\PYG{o}{.}\PYG{n}{set\PYGZus{}label}\PYG{p}{(}\PYG{l+s+s2}{\PYGZdq{}}\PYG{l+s+s2}{density}\PYG{l+s+s2}{\PYGZdq{}}\PYG{p}{)}
\end{sphinxVerbatim}

\sphinxincludegraphics{{output_80_0}.png}


\section{多子图}
\label{\detokenize{_u591a_u5b50_u56fe:id1}}\label{\detokenize{_u591a_u5b50_u56fe::doc}}
一个界面上有时候需要出现多张图形,这就是多子图。

Matplotlib提供了subplot的概念,用来在较大的图形中同时放置较小的一组坐标轴,这些子图可能是画中画,网格图,或者更复杂的布局形式。

\fvset{hllines={, ,}}%
\begin{sphinxVerbatim}[commandchars=\\\{\}]
\PYG{c+c1}{\PYGZsh{} 准备数据}
\PYG{o}{\PYGZpc{}}\PYG{n}{matplotlib} \PYG{n}{inline}
\PYG{k+kn}{import} \PYG{n+nn}{matplotlib.pyplot} \PYG{k+kn}{as} \PYG{n+nn}{plt}
\PYG{k+kn}{import} \PYG{n+nn}{numpy} \PYG{k+kn}{as} \PYG{n+nn}{np}

\PYG{n}{plt}\PYG{o}{.}\PYG{n}{style}\PYG{o}{.}\PYG{n}{use}\PYG{p}{(}\PYG{l+s+s2}{\PYGZdq{}}\PYG{l+s+s2}{seaborn\PYGZhy{}white}\PYG{l+s+s2}{\PYGZdq{}}\PYG{p}{)}
\end{sphinxVerbatim}


\subsection{手动创建多子图}
\label{\detokenize{_u591a_u5b50_u56fe:id2}}
创建坐标轴最基本的方法是用plt.axes函数, 通过设置不同的坐标参数,可以用来手工创建多子图。

多子图坐标系统可以使用四个元素列表,{[}left, button, width, height{]},分别是:
\begin{itemize}
\item {} 
bottom: 底部坐标

\item {} 
left:左侧坐标

\item {} 
width: 宽度

\item {} 
height: 高度

\end{itemize}

其中数值的取值范围是0-1, 左下角为0.

{[}0.5, 0.2, 0.3, 0.4{]}代表从下面开始50\%,从左开始20\%的地方是图形的左下角,图形宽高占用30\%和40\%。

\fvset{hllines={, ,}}%
\begin{sphinxVerbatim}[commandchars=\\\{\}]
\PYG{c+c1}{\PYGZsh{}Matlab风格}
\PYG{n}{ax1} \PYG{o}{=} \PYG{n}{plt}\PYG{o}{.}\PYG{n}{axes}\PYG{p}{(}\PYG{p}{)}
\PYG{n}{ax2} \PYG{o}{=} \PYG{n}{plt}\PYG{o}{.}\PYG{n}{axes}\PYG{p}{(}\PYG{p}{[}\PYG{l+m+mf}{0.4}\PYG{p}{,} \PYG{l+m+mf}{0.2}\PYG{p}{,} \PYG{l+m+mf}{0.3}\PYG{p}{,} \PYG{l+m+mf}{0.6}\PYG{p}{]}\PYG{p}{)}
\end{sphinxVerbatim}

\sphinxincludegraphics{{output_84_0}.png}

\fvset{hllines={, ,}}%
\begin{sphinxVerbatim}[commandchars=\\\{\}]
\PYG{c+c1}{\PYGZsh{}面向对象格式}
\PYG{n}{fig} \PYG{o}{=} \PYG{n}{plt}\PYG{o}{.}\PYG{n}{figure}\PYG{p}{(}\PYG{p}{)}

\PYG{n}{ax1} \PYG{o}{=} \PYG{n}{fig}\PYG{o}{.}\PYG{n}{add\PYGZus{}axes}\PYG{p}{(}\PYG{p}{[}\PYG{l+m+mf}{0.1}\PYG{p}{,} \PYG{l+m+mf}{0.5}\PYG{p}{,} \PYG{l+m+mf}{0.8}\PYG{p}{,} \PYG{l+m+mf}{0.4}\PYG{p}{]}\PYG{p}{,} \PYG{n}{xticklabels}\PYG{o}{=}\PYG{p}{[}\PYG{p}{]}\PYG{p}{,} \PYG{n}{ylim}\PYG{o}{=}\PYG{p}{(}\PYG{o}{\PYGZhy{}}\PYG{l+m+mf}{1.2}\PYG{p}{,} \PYG{l+m+mf}{1.2}\PYG{p}{)}\PYG{p}{)}
\PYG{n}{ax2} \PYG{o}{=} \PYG{n}{fig}\PYG{o}{.}\PYG{n}{add\PYGZus{}axes}\PYG{p}{(}\PYG{p}{[}\PYG{l+m+mf}{0.1}\PYG{p}{,} \PYG{l+m+mf}{0.1}\PYG{p}{,} \PYG{l+m+mf}{0.8}\PYG{p}{,} \PYG{l+m+mf}{0.4}\PYG{p}{]}\PYG{p}{,} \PYG{n}{ylim}\PYG{o}{=}\PYG{p}{(}\PYG{o}{\PYGZhy{}}\PYG{l+m+mf}{1.2}\PYG{p}{,} \PYG{l+m+mf}{1.2}\PYG{p}{)}\PYG{p}{)}

\PYG{n}{x} \PYG{o}{=} \PYG{n}{np}\PYG{o}{.}\PYG{n}{linspace}\PYG{p}{(}\PYG{l+m+mi}{0}\PYG{p}{,} \PYG{l+m+mi}{10}\PYG{p}{)}
\PYG{n}{ax1}\PYG{o}{.}\PYG{n}{plot}\PYG{p}{(}\PYG{n}{np}\PYG{o}{.}\PYG{n}{sin}\PYG{p}{(}\PYG{n}{x}\PYG{p}{)}\PYG{p}{)}
\PYG{n}{ax2}\PYG{o}{.}\PYG{n}{plot}\PYG{p}{(}\PYG{n}{np}\PYG{o}{.}\PYG{n}{cos}\PYG{p}{(}\PYG{n}{x}\PYG{p}{)}\PYG{p}{)}

\end{sphinxVerbatim}

\fvset{hllines={, ,}}%
\begin{sphinxVerbatim}[commandchars=\\\{\}]
\PYG{p}{[}\PYG{o}{\PYGZlt{}}\PYG{n}{matplotlib}\PYG{o}{.}\PYG{n}{lines}\PYG{o}{.}\PYG{n}{Line2D} \PYG{n}{at} \PYG{l+m+mh}{0x7fc8b5147748}\PYG{o}{\PYGZgt{}}\PYG{p}{]}
\end{sphinxVerbatim}

\sphinxincludegraphics{{output_85_1}.png}


\subsection{plt.subplot 建议网格子图}
\label{\detokenize{_u591a_u5b50_u56fe:plt-subplot}}
plt.subplot可以创建整齐排列的子图,这个函数有三个整形参数:
\begin{itemize}
\item {} 
子图行数

\item {} 
子图列数

\item {} 
索引值: 索引值从1开始,左上角到右下角逐渐增大

\end{itemize}

\fvset{hllines={, ,}}%
\begin{sphinxVerbatim}[commandchars=\\\{\}]
\PYG{c+c1}{\PYGZsh{} matlab风格}
\PYG{k}{for} \PYG{n}{i} \PYG{o+ow}{in} \PYG{n+nb}{range}\PYG{p}{(}\PYG{l+m+mi}{1}\PYG{p}{,}\PYG{l+m+mi}{7}\PYG{p}{)}\PYG{p}{:}
    \PYG{n}{plt}\PYG{o}{.}\PYG{n}{subplot}\PYG{p}{(}\PYG{l+m+mi}{2}\PYG{p}{,}\PYG{l+m+mi}{3}\PYG{p}{,}\PYG{n}{i}\PYG{p}{)}
    \PYG{n}{plt}\PYG{o}{.}\PYG{n}{text}\PYG{p}{(}\PYG{l+m+mf}{0.5}\PYG{p}{,} \PYG{l+m+mf}{0.5}\PYG{p}{,} \PYG{n+nb}{str}\PYG{p}{(}\PYG{p}{(}\PYG{l+m+mi}{2}\PYG{p}{,}\PYG{l+m+mi}{3}\PYG{p}{,}\PYG{n}{i}\PYG{p}{)}\PYG{p}{)}\PYG{p}{,} \PYG{n}{fontsize}\PYG{o}{=}\PYG{l+m+mi}{18}\PYG{p}{,} \PYG{n}{ha}\PYG{o}{=}\PYG{l+s+s1}{\PYGZsq{}}\PYG{l+s+s1}{center}\PYG{l+s+s1}{\PYGZsq{}}\PYG{p}{)}
\end{sphinxVerbatim}

\sphinxincludegraphics{{output_87_0}.png}

\fvset{hllines={, ,}}%
\begin{sphinxVerbatim}[commandchars=\\\{\}]
\PYG{c+c1}{\PYGZsh{}面向对象风格}
\PYG{c+c1}{\PYGZsh{} plt.subplots\PYGZus{}adjust调整子图间隔}
\PYG{c+c1}{\PYGZsh{} plt.add\PYGZus{}subplot}

\PYG{n}{fig} \PYG{o}{=} \PYG{n}{plt}\PYG{o}{.}\PYG{n}{figure}\PYG{p}{(}\PYG{p}{)}
\PYG{n}{fig}\PYG{o}{.}\PYG{n}{subplots\PYGZus{}adjust}\PYG{p}{(}\PYG{n}{hspace}\PYG{o}{=}\PYG{l+m+mf}{0.4}\PYG{p}{,} \PYG{n}{wspace}\PYG{o}{=}\PYG{l+m+mf}{0.4}\PYG{p}{)}

\PYG{k}{for} \PYG{n}{i} \PYG{o+ow}{in} \PYG{n+nb}{range}\PYG{p}{(}\PYG{l+m+mi}{1}\PYG{p}{,}\PYG{l+m+mi}{7}\PYG{p}{)}\PYG{p}{:}
    \PYG{n}{ax} \PYG{o}{=} \PYG{n}{fig}\PYG{o}{.}\PYG{n}{add\PYGZus{}subplot}\PYG{p}{(}\PYG{l+m+mi}{2}\PYG{p}{,}\PYG{l+m+mi}{3}\PYG{p}{,}\PYG{n}{i}\PYG{p}{)}
    \PYG{n}{ax}\PYG{o}{.}\PYG{n}{text}\PYG{p}{(}\PYG{l+m+mf}{0.5}\PYG{p}{,} \PYG{l+m+mf}{0.5}\PYG{p}{,} \PYG{n+nb}{str}\PYG{p}{(}\PYG{p}{(}\PYG{l+m+mi}{2}\PYG{p}{,}\PYG{l+m+mi}{3}\PYG{p}{,}\PYG{n}{i}\PYG{p}{)}\PYG{p}{)}\PYG{p}{,} \PYG{n}{fontsize}\PYG{o}{=}\PYG{l+m+mi}{18}\PYG{p}{,} \PYG{n}{ha}\PYG{o}{=}\PYG{l+s+s1}{\PYGZsq{}}\PYG{l+s+s1}{center}\PYG{l+s+s1}{\PYGZsq{}}\PYG{p}{)}
   
\end{sphinxVerbatim}

\sphinxincludegraphics{{output_88_0}.png}


\subsection{7.3 plt.subplots}
\label{\detokenize{_u591a_u5b50_u56fe:plt-subplots}}
一次性创建多个子图,主要四个参数:
\begin{itemize}
\item {} 
行数

\item {} 
列数

\item {} 
sharex:是否共享x轴

\item {} 
sharey:是否共享y轴

\end{itemize}

下图使用subplots创建2x3个子图,每一行共享y轴,每一列共享x轴。

函数返回值是一个NumPy数组,可以通过下表访问返回的每一个子图。

\fvset{hllines={, ,}}%
\begin{sphinxVerbatim}[commandchars=\\\{\}]
\PYG{n}{fig}\PYG{p}{,} \PYG{n}{ax} \PYG{o}{=} \PYG{n}{plt}\PYG{o}{.}\PYG{n}{subplots}\PYG{p}{(}\PYG{l+m+mi}{2}\PYG{p}{,} \PYG{l+m+mi}{3}\PYG{p}{,} \PYG{n}{sharex}\PYG{o}{=}\PYG{l+s+s1}{\PYGZsq{}}\PYG{l+s+s1}{col}\PYG{l+s+s1}{\PYGZsq{}}\PYG{p}{,} \PYG{n}{sharey}\PYG{o}{=}\PYG{l+s+s1}{\PYGZsq{}}\PYG{l+s+s1}{row}\PYG{l+s+s1}{\PYGZsq{}}\PYG{p}{)}
\end{sphinxVerbatim}

\sphinxincludegraphics{{output_90_0}.png}

\fvset{hllines={, ,}}%
\begin{sphinxVerbatim}[commandchars=\\\{\}]
\PYG{k}{for} \PYG{n}{i} \PYG{o+ow}{in} \PYG{n+nb}{range}\PYG{p}{(}\PYG{l+m+mi}{2}\PYG{p}{)}\PYG{p}{:}
    \PYG{k}{for} \PYG{n}{j} \PYG{o+ow}{in} \PYG{n+nb}{range}\PYG{p}{(}\PYG{l+m+mi}{3}\PYG{p}{)}\PYG{p}{:}
        \PYG{n}{ax}\PYG{p}{[}\PYG{n}{i}\PYG{p}{,}\PYG{n}{j}\PYG{p}{]}\PYG{o}{.}\PYG{n}{text}\PYG{p}{(}\PYG{l+m+mf}{0.5}\PYG{p}{,} \PYG{l+m+mf}{0.5}\PYG{p}{,} \PYG{n+nb}{str}\PYG{p}{(}\PYG{p}{(}\PYG{n}{i}\PYG{p}{,}\PYG{n}{j}\PYG{p}{)}\PYG{p}{)}\PYG{p}{,} \PYG{n}{fontsize}\PYG{o}{=}\PYG{l+m+mi}{18}\PYG{p}{,} \PYG{n}{ha}\PYG{o}{=}\PYG{l+s+s1}{\PYGZsq{}}\PYG{l+s+s1}{center}\PYG{l+s+s1}{\PYGZsq{}}\PYG{p}{)}

\PYG{n}{fig}
\end{sphinxVerbatim}

\sphinxincludegraphics{{output_91_0}.png}


\subsection{plt.GridSpec}
\label{\detokenize{_u591a_u5b50_u56fe:plt-gridspec}}
利用plt.GridSpec可以用来实现更复杂的多行多列子图网格。这种画图方式跟前端网页技术的网格思想类似,需先用
plt.GridSpec指出需要总共划分的行和列,然后在具体的画相应子图的时候指出一个子图需要占用的网格。

\fvset{hllines={, ,}}%
\begin{sphinxVerbatim}[commandchars=\\\{\}]
\PYG{c+c1}{\PYGZsh{} 画图方式}
\PYG{c+c1}{\PYGZsh{} 界面总共分成了2行3列}
\PYG{n}{grid} \PYG{o}{=} \PYG{n}{plt}\PYG{o}{.}\PYG{n}{GridSpec}\PYG{p}{(}\PYG{l+m+mi}{2}\PYG{p}{,} \PYG{l+m+mi}{3}\PYG{p}{,} \PYG{n}{wspace}\PYG{o}{=}\PYG{l+m+mf}{0.4}\PYG{p}{,} \PYG{n}{hspace}\PYG{o}{=}\PYG{l+m+mf}{0.3}\PYG{p}{)}

\PYG{c+c1}{\PYGZsh{} 第一个子图占用了第一个方格}
\PYG{n}{plt}\PYG{o}{.}\PYG{n}{subplot}\PYG{p}{(}\PYG{n}{grid}\PYG{p}{[}\PYG{l+m+mi}{0}\PYG{p}{,}\PYG{l+m+mi}{0}\PYG{p}{]}\PYG{p}{)}
\PYG{c+c1}{\PYGZsh{} 第二个子图占用了第一行从第二个后面所有的方格}
\PYG{n}{plt}\PYG{o}{.}\PYG{n}{subplot}\PYG{p}{(}\PYG{n}{grid}\PYG{p}{[}\PYG{l+m+mi}{0}\PYG{p}{,}\PYG{l+m+mi}{1}\PYG{p}{:}\PYG{p}{]}\PYG{p}{)}
\PYG{c+c1}{\PYGZsh{} 第三个子图占用了第二行到下标2前的方格}
\PYG{n}{plt}\PYG{o}{.}\PYG{n}{subplot}\PYG{p}{(}\PYG{n}{grid}\PYG{p}{[}\PYG{l+m+mi}{1}\PYG{p}{,}\PYG{p}{:}\PYG{l+m+mi}{2}\PYG{p}{]}\PYG{p}{)}
\PYG{c+c1}{\PYGZsh{} 第四个子图占用了第二行第三个方格}
\PYG{n}{plt}\PYG{o}{.}\PYG{n}{subplot}\PYG{p}{(}\PYG{n}{grid}\PYG{p}{[}\PYG{l+m+mi}{1}\PYG{p}{,}\PYG{l+m+mi}{2}\PYG{p}{]}\PYG{p}{)}
\end{sphinxVerbatim}

\fvset{hllines={, ,}}%
\begin{sphinxVerbatim}[commandchars=\\\{\}]
\PYG{o}{\PYGZlt{}}\PYG{n}{matplotlib}\PYG{o}{.}\PYG{n}{axes}\PYG{o}{.}\PYG{n}{\PYGZus{}subplots}\PYG{o}{.}\PYG{n}{AxesSubplot} \PYG{n}{at} \PYG{l+m+mh}{0x7fc8b4d7db00}\PYG{o}{\PYGZgt{}}
\end{sphinxVerbatim}

\sphinxincludegraphics{{output_93_1}.png}

\fvset{hllines={, ,}}%
\begin{sphinxVerbatim}[commandchars=\\\{\}]
\PYG{c+c1}{\PYGZsh{} 正态分布数据的多子图显示}

\PYG{n}{mean} \PYG{o}{=} \PYG{p}{[}\PYG{l+m+mi}{0}\PYG{p}{,}\PYG{l+m+mi}{0}\PYG{p}{]}
\PYG{n}{cov} \PYG{o}{=} \PYG{p}{[}\PYG{p}{[}\PYG{l+m+mi}{1}\PYG{p}{,}\PYG{l+m+mi}{1}\PYG{p}{]}\PYG{p}{,} \PYG{p}{[}\PYG{l+m+mi}{1}\PYG{p}{,}\PYG{l+m+mi}{2}\PYG{p}{]}\PYG{p}{]}
\PYG{n}{x}\PYG{p}{,} \PYG{n}{y} \PYG{o}{=} \PYG{n}{np}\PYG{o}{.}\PYG{n}{random}\PYG{o}{.}\PYG{n}{multivariate\PYGZus{}normal}\PYG{p}{(}\PYG{n}{mean}\PYG{p}{,} \PYG{n}{cov}\PYG{p}{,} \PYG{l+m+mi}{3000}\PYG{p}{)}\PYG{o}{.}\PYG{n}{T}

\PYG{c+c1}{\PYGZsh{}设置坐标轴和网格配置}
\PYG{n}{fig} \PYG{o}{=} \PYG{n}{plt}\PYG{o}{.}\PYG{n}{figure}\PYG{p}{(}\PYG{n}{figsize}\PYG{o}{=}\PYG{p}{(}\PYG{l+m+mi}{6}\PYG{p}{,}\PYG{l+m+mi}{6}\PYG{p}{)}\PYG{p}{)}
\PYG{n}{grid} \PYG{o}{=} \PYG{n}{plt}\PYG{o}{.}\PYG{n}{GridSpec}\PYG{p}{(}\PYG{l+m+mi}{4}\PYG{p}{,}\PYG{l+m+mi}{4}\PYG{p}{,} \PYG{n}{hspace}\PYG{o}{=}\PYG{l+m+mf}{0.2}\PYG{p}{,} \PYG{n}{wspace}\PYG{o}{=}\PYG{l+m+mf}{0.2}\PYG{p}{)}

\PYG{n}{main\PYGZus{}ax} \PYG{o}{=} \PYG{n}{fig}\PYG{o}{.}\PYG{n}{add\PYGZus{}subplot}\PYG{p}{(}\PYG{n}{grid}\PYG{p}{[}\PYG{p}{:}\PYG{o}{\PYGZhy{}}\PYG{l+m+mi}{1}\PYG{p}{,} \PYG{l+m+mi}{1}\PYG{p}{:}\PYG{p}{]}\PYG{p}{)}
\PYG{n}{y\PYGZus{}hist} \PYG{o}{=} \PYG{n}{fig}\PYG{o}{.}\PYG{n}{add\PYGZus{}subplot}\PYG{p}{(}\PYG{n}{grid}\PYG{p}{[}\PYG{p}{:}\PYG{o}{\PYGZhy{}}\PYG{l+m+mi}{1}\PYG{p}{,} \PYG{l+m+mi}{0}\PYG{p}{]}\PYG{p}{,} \PYG{n}{xticklabels}\PYG{o}{=}\PYG{p}{[}\PYG{p}{]}\PYG{p}{,} \PYG{n}{sharey}\PYG{o}{=}\PYG{n}{main\PYGZus{}ax}\PYG{p}{)}
\PYG{n}{x\PYGZus{}hist} \PYG{o}{=} \PYG{n}{fig}\PYG{o}{.}\PYG{n}{add\PYGZus{}subplot}\PYG{p}{(}\PYG{n}{grid}\PYG{p}{[}\PYG{o}{\PYGZhy{}}\PYG{l+m+mi}{1}\PYG{p}{,} \PYG{l+m+mi}{1}\PYG{p}{:}\PYG{p}{]}\PYG{p}{,} \PYG{n}{yticklabels}\PYG{o}{=}\PYG{p}{[}\PYG{p}{]}\PYG{p}{,} \PYG{n}{sharex}\PYG{o}{=}\PYG{n}{main\PYGZus{}ax}\PYG{p}{)}

\PYG{c+c1}{\PYGZsh{} 主轴坐标画散点图}
\PYG{n}{main\PYGZus{}ax}\PYG{o}{.}\PYG{n}{plot}\PYG{p}{(}\PYG{n}{x}\PYG{p}{,} \PYG{n}{y}\PYG{p}{,} \PYG{l+s+s1}{\PYGZsq{}}\PYG{l+s+s1}{ok}\PYG{l+s+s1}{\PYGZsq{}}\PYG{p}{,} \PYG{n}{markersize}\PYG{o}{=}\PYG{l+m+mi}{3}\PYG{p}{,} \PYG{n}{alpha}\PYG{o}{=}\PYG{l+m+mf}{0.2}\PYG{p}{)}

\PYG{c+c1}{\PYGZsh{} 次轴坐标画直方图}
\PYG{n}{x\PYGZus{}hist}\PYG{o}{.}\PYG{n}{hist}\PYG{p}{(}\PYG{n}{x}\PYG{p}{,} \PYG{l+m+mi}{40}\PYG{p}{,} \PYG{n}{histtype}\PYG{o}{=}\PYG{l+s+s1}{\PYGZsq{}}\PYG{l+s+s1}{stepfilled}\PYG{l+s+s1}{\PYGZsq{}}\PYG{p}{,} \PYG{n}{orientation}\PYG{o}{=}\PYG{l+s+s1}{\PYGZsq{}}\PYG{l+s+s1}{vertical}\PYG{l+s+s1}{\PYGZsq{}}\PYG{p}{,} \PYG{n}{color}\PYG{o}{=}\PYG{l+s+s1}{\PYGZsq{}}\PYG{l+s+s1}{red}\PYG{l+s+s1}{\PYGZsq{}}\PYG{p}{)}
\PYG{n}{x\PYGZus{}hist}\PYG{o}{.}\PYG{n}{invert\PYGZus{}yaxis}\PYG{p}{(}\PYG{p}{)}


\PYG{n}{y\PYGZus{}hist}\PYG{o}{.}\PYG{n}{hist}\PYG{p}{(}\PYG{n}{x}\PYG{p}{,} \PYG{l+m+mi}{40}\PYG{p}{,} \PYG{n}{histtype}\PYG{o}{=}\PYG{l+s+s1}{\PYGZsq{}}\PYG{l+s+s1}{stepfilled}\PYG{l+s+s1}{\PYGZsq{}}\PYG{p}{,} \PYG{n}{orientation}\PYG{o}{=}\PYG{l+s+s1}{\PYGZsq{}}\PYG{l+s+s1}{horizontal}\PYG{l+s+s1}{\PYGZsq{}}\PYG{p}{,} \PYG{n}{color}\PYG{o}{=}\PYG{l+s+s1}{\PYGZsq{}}\PYG{l+s+s1}{blue}\PYG{l+s+s1}{\PYGZsq{}}\PYG{p}{)}
\PYG{n}{x\PYGZus{}hist}\PYG{o}{.}\PYG{n}{invert\PYGZus{}xaxis}\PYG{p}{(}\PYG{p}{)}
\end{sphinxVerbatim}

\sphinxincludegraphics{{output_94_0}.png}


\section{文字与注释}
\label{\detokenize{_u6587_u5b57_u4e0e_u6ce8_u91ca:id1}}\label{\detokenize{_u6587_u5b57_u4e0e_u6ce8_u91ca::doc}}

\subsection{坐标变换和文字位置}
\label{\detokenize{_u6587_u5b57_u4e0e_u6ce8_u91ca:id2}}
通过不同的坐标变换,可以把文字放在不同的位置,文字的坐标变换方法有:
\begin{itemize}
\item {} 
ax.transData:以数据为基准的坐标变换,直接用数字把坐标表示出来,绝对数据

\item {} 
ax.transAxes: 以轴为基准,坐标数据表示轴中的相对位置

\end{itemize}

\fvset{hllines={, ,}}%
\begin{sphinxVerbatim}[commandchars=\\\{\}]
\PYG{c+c1}{\PYGZsh{}环境准备}

\PYG{o}{\PYGZpc{}}\PYG{n}{matplotlib} \PYG{n}{inline}
\PYG{k+kn}{import} \PYG{n+nn}{matplotlib.pyplot} \PYG{k+kn}{as} \PYG{n+nn}{plt}
\PYG{k+kn}{import} \PYG{n+nn}{numpy} \PYG{k+kn}{as} \PYG{n+nn}{np}

\PYG{n}{plt}\PYG{o}{.}\PYG{n}{style}\PYG{o}{.}\PYG{n}{use}\PYG{p}{(}\PYG{l+s+s2}{\PYGZdq{}}\PYG{l+s+s2}{seaborn\PYGZhy{}whitegrid}\PYG{l+s+s2}{\PYGZdq{}}\PYG{p}{)}
\end{sphinxVerbatim}

\fvset{hllines={, ,}}%
\begin{sphinxVerbatim}[commandchars=\\\{\}]
\PYG{n}{fig}\PYG{p}{,} \PYG{n}{ax} \PYG{o}{=} \PYG{n}{plt}\PYG{o}{.}\PYG{n}{subplots}\PYG{p}{(}\PYG{n}{facecolor}\PYG{o}{=}\PYG{l+s+s1}{\PYGZsq{}}\PYG{l+s+s1}{lightgray}\PYG{l+s+s1}{\PYGZsq{}}\PYG{p}{)}
\PYG{n}{ax}\PYG{o}{.}\PYG{n}{axis}\PYG{p}{(}\PYG{p}{[}\PYG{l+m+mi}{0}\PYG{p}{,} \PYG{l+m+mi}{10}\PYG{p}{,} \PYG{l+m+mi}{0}\PYG{p}{,} \PYG{l+m+mi}{10}\PYG{p}{]}\PYG{p}{)}

\PYG{n}{ax}\PYG{o}{.}\PYG{n}{text}\PYG{p}{(}\PYG{l+m+mi}{1}\PYG{p}{,} \PYG{l+m+mi}{5}\PYG{p}{,} \PYG{l+s+s2}{\PYGZdq{}}\PYG{l+s+s2}{Data:(1,5)}\PYG{l+s+s2}{\PYGZdq{}}\PYG{p}{,} \PYG{n}{transform}\PYG{o}{=}\PYG{n}{ax}\PYG{o}{.}\PYG{n}{transData}\PYG{p}{)}
\PYG{n}{ax}\PYG{o}{.}\PYG{n}{text}\PYG{p}{(}\PYG{l+m+mf}{0.5}\PYG{p}{,} \PYG{l+m+mf}{0.1}\PYG{p}{,} \PYG{l+s+s2}{\PYGZdq{}}\PYG{l+s+s2}{Axes:(0.5, 0.1)}\PYG{l+s+s2}{\PYGZdq{}}\PYG{p}{,} \PYG{n}{transform}\PYG{o}{=}\PYG{n}{ax}\PYG{o}{.}\PYG{n}{transAxes}\PYG{p}{)}
\end{sphinxVerbatim}

\fvset{hllines={, ,}}%
\begin{sphinxVerbatim}[commandchars=\\\{\}]
\PYG{n}{Text}\PYG{p}{(}\PYG{l+m+mf}{0.5}\PYG{p}{,}\PYG{l+m+mf}{0.1}\PYG{p}{,}\PYG{l+s+s1}{\PYGZsq{}}\PYG{l+s+s1}{Axes:(0.5, 0.1)}\PYG{l+s+s1}{\PYGZsq{}}\PYG{p}{)}
\end{sphinxVerbatim}

\sphinxincludegraphics{{output_97_1}.png}


\subsection{箭头和注释}
\label{\detokenize{_u6587_u5b57_u4e0e_u6ce8_u91ca:id3}}
带箭头的注释一般可以使用两个函数实现:
\begin{itemize}
\item {} 
plt.arrow: 产生SVG向量图形式的箭头,会随着分辨率改变而变换,不推荐

\item {} 
plt.annotate: 可以创建文字和箭头

\end{itemize}

在annotate中,箭头的风格通过arrowprops参数控制,具体参数含义使用的时候可以参考官方文档。

\fvset{hllines={, ,}}%
\begin{sphinxVerbatim}[commandchars=\\\{\}]
\PYG{n}{fig}\PYG{p}{,} \PYG{n}{ax} \PYG{o}{=} \PYG{n}{plt}\PYG{o}{.}\PYG{n}{subplots}\PYG{p}{(}\PYG{p}{)}

\PYG{n}{x} \PYG{o}{=} \PYG{n}{np}\PYG{o}{.}\PYG{n}{linspace}\PYG{p}{(}\PYG{l+m+mi}{0}\PYG{p}{,} \PYG{l+m+mi}{20}\PYG{p}{,} \PYG{l+m+mi}{1000}\PYG{p}{)}
\PYG{n}{ax}\PYG{o}{.}\PYG{n}{plot}\PYG{p}{(}\PYG{n}{x}\PYG{p}{,} \PYG{n}{np}\PYG{o}{.}\PYG{n}{cos}\PYG{p}{(}\PYG{n}{x}\PYG{p}{)}\PYG{p}{)}
\PYG{n}{ax}\PYG{o}{.}\PYG{n}{axis}\PYG{p}{(}\PYG{l+s+s1}{\PYGZsq{}}\PYG{l+s+s1}{equal}\PYG{l+s+s1}{\PYGZsq{}}\PYG{p}{)}

\PYG{n}{ax}\PYG{o}{.}\PYG{n}{annotate}\PYG{p}{(}\PYG{l+s+s2}{\PYGZdq{}}\PYG{l+s+s2}{local maximum}\PYG{l+s+s2}{\PYGZdq{}}\PYG{p}{,} \PYG{n}{xy}\PYG{o}{=}\PYG{p}{(}\PYG{l+m+mf}{6.28}\PYG{p}{,} \PYG{l+m+mi}{1}\PYG{p}{)}\PYG{p}{,} \PYG{n}{xytext}\PYG{o}{=}\PYG{p}{(}\PYG{l+m+mi}{10}\PYG{p}{,} \PYG{l+m+mi}{4}\PYG{p}{)}\PYG{p}{,} \PYGZbs{}
            \PYG{n}{arrowprops}\PYG{o}{=}\PYG{n+nb}{dict}\PYG{p}{(}\PYG{n}{facecolor}\PYG{o}{=}\PYG{l+s+s1}{\PYGZsq{}}\PYG{l+s+s1}{black}\PYG{l+s+s1}{\PYGZsq{}}\PYG{p}{,}\PYG{n}{shrink}\PYG{o}{=}\PYG{l+m+mf}{0.05} \PYG{p}{)}\PYG{p}{)}

\PYG{n}{ax}\PYG{o}{.}\PYG{n}{annotate}\PYG{p}{(}\PYG{l+s+s1}{\PYGZsq{}}\PYG{l+s+s1}{local minimum}\PYG{l+s+s1}{\PYGZsq{}}\PYG{p}{,} \PYG{n}{xy}\PYG{o}{=}\PYG{p}{(}\PYG{l+m+mi}{5} \PYG{o}{*} \PYG{n}{np}\PYG{o}{.}\PYG{n}{pi}\PYG{p}{,} \PYG{o}{\PYGZhy{}}\PYG{l+m+mi}{1}\PYG{p}{)}\PYG{p}{,} \PYG{n}{xytext}\PYG{o}{=}\PYG{p}{(}\PYG{l+m+mi}{2}\PYG{p}{,} \PYG{o}{\PYGZhy{}}\PYG{l+m+mi}{6}\PYG{p}{)}\PYG{p}{,}\PYGZbs{}
           \PYG{n}{arrowprops}\PYG{o}{=}\PYG{n+nb}{dict}\PYG{p}{(}\PYG{n}{arrowstyle}\PYG{o}{=}\PYG{l+s+s2}{\PYGZdq{}}\PYG{l+s+s2}{\PYGZhy{}\PYGZgt{}}\PYG{l+s+s2}{\PYGZdq{}}\PYG{p}{,} \PYG{n}{connectionstyle}\PYG{o}{=}\PYG{l+s+s1}{\PYGZsq{}}\PYG{l+s+s1}{angle3, angleA=0, angleB=\PYGZhy{}90}\PYG{l+s+s1}{\PYGZsq{}}\PYG{p}{)}\PYG{p}{)}

\end{sphinxVerbatim}

\fvset{hllines={, ,}}%
\begin{sphinxVerbatim}[commandchars=\\\{\}]
\PYG{n}{Text}\PYG{p}{(}\PYG{l+m+mi}{2}\PYG{p}{,}\PYG{o}{\PYGZhy{}}\PYG{l+m+mi}{6}\PYG{p}{,}\PYG{l+s+s1}{\PYGZsq{}}\PYG{l+s+s1}{local minimum}\PYG{l+s+s1}{\PYGZsq{}}\PYG{p}{)}
\end{sphinxVerbatim}

\sphinxincludegraphics{{output_99_1}.png}


\subsection{自定义坐标轴刻度}
\label{\detokenize{_u6587_u5b57_u4e0e_u6ce8_u91ca:id4}}
Matplotlib有默认的坐标轴定位器(locator)和格式生成器(formatter),基本需求可以满足自定义坐标轴的需求,但是如果需要定制更细腻的表现,需要用到其他的东西。

Matplotlib画图的基本原理是:
\begin{itemize}
\item {} 
figure对象可以看做是一个图形的总的容器,里面可以包含几个子图

\item {} 
axes:每个figure包含一个或者多个axes,每个axes有包含其他表示图形内容的对象

\item {} 
每个axes有xaxis和yaxis属性,每个属性包含坐标轴的线条,刻度,标签等属性

\end{itemize}


\subsubsection{主要刻度和次要刻度}
\label{\detokenize{_u6587_u5b57_u4e0e_u6ce8_u91ca:id5}}
通过一下案例,我们发现主要和次要刻度标签都是通过LogLocater对象设置的, 同样格式生成器都是LogFormatterSciNotaion对象。

\fvset{hllines={, ,}}%
\begin{sphinxVerbatim}[commandchars=\\\{\}]
\PYG{n}{ax} \PYG{o}{=} \PYG{n}{plt}\PYG{o}{.}\PYG{n}{axes}\PYG{p}{(}\PYG{n}{xscale}\PYG{o}{=}\PYG{l+s+s1}{\PYGZsq{}}\PYG{l+s+s1}{log}\PYG{l+s+s1}{\PYGZsq{}}\PYG{p}{,} \PYG{n}{yscale}\PYG{o}{=}\PYG{l+s+s1}{\PYGZsq{}}\PYG{l+s+s1}{log}\PYG{l+s+s1}{\PYGZsq{}}\PYG{p}{)}

\PYG{k}{print}\PYG{p}{(}\PYG{n}{ax}\PYG{o}{.}\PYG{n}{xaxis}\PYG{o}{.}\PYG{n}{get\PYGZus{}major\PYGZus{}locator}\PYG{p}{(}\PYG{p}{)}\PYG{p}{)}
\PYG{k}{print}\PYG{p}{(}\PYG{n}{ax}\PYG{o}{.}\PYG{n}{xaxis}\PYG{o}{.}\PYG{n}{get\PYGZus{}minor\PYGZus{}locator}\PYG{p}{(}\PYG{p}{)}\PYG{p}{)}

\PYG{k}{print}\PYG{p}{(}\PYG{n}{ax}\PYG{o}{.}\PYG{n}{xaxis}\PYG{o}{.}\PYG{n}{get\PYGZus{}major\PYGZus{}formatter}\PYG{p}{(}\PYG{p}{)}\PYG{p}{)}
\PYG{k}{print}\PYG{p}{(}\PYG{n}{ax}\PYG{o}{.}\PYG{n}{xaxis}\PYG{o}{.}\PYG{n}{get\PYGZus{}minor\PYGZus{}formatter}\PYG{p}{(}\PYG{p}{)}\PYG{p}{)}
\end{sphinxVerbatim}

\fvset{hllines={, ,}}%
\begin{sphinxVerbatim}[commandchars=\\\{\}]
\PYG{o}{\PYGZlt{}}\PYG{n}{matplotlib}\PYG{o}{.}\PYG{n}{ticker}\PYG{o}{.}\PYG{n}{LogLocator} \PYG{n+nb}{object} \PYG{n}{at} \PYG{l+m+mh}{0x7fd697b69c50}\PYG{o}{\PYGZgt{}}
\PYG{o}{\PYGZlt{}}\PYG{n}{matplotlib}\PYG{o}{.}\PYG{n}{ticker}\PYG{o}{.}\PYG{n}{LogLocator} \PYG{n+nb}{object} \PYG{n}{at} \PYG{l+m+mh}{0x7fd697b69a58}\PYG{o}{\PYGZgt{}}
\PYG{o}{\PYGZlt{}}\PYG{n}{matplotlib}\PYG{o}{.}\PYG{n}{ticker}\PYG{o}{.}\PYG{n}{LogFormatterSciNotation} \PYG{n+nb}{object} \PYG{n}{at} \PYG{l+m+mh}{0x7fd697b69a90}\PYG{o}{\PYGZgt{}}
\PYG{o}{\PYGZlt{}}\PYG{n}{matplotlib}\PYG{o}{.}\PYG{n}{ticker}\PYG{o}{.}\PYG{n}{LogFormatterSciNotation} \PYG{n+nb}{object} \PYG{n}{at} \PYG{l+m+mh}{0x7fd697be4a58}\PYG{o}{\PYGZgt{}}
\end{sphinxVerbatim}

\sphinxincludegraphics{{output_101_1}.png}


\subsubsection{隐藏刻度和标签}
\label{\detokenize{_u6587_u5b57_u4e0e_u6ce8_u91ca:id6}}
有时候我们不需要总显示刻度和标签,可以通过设置空的刻度标签和格式化生成器完成。

\fvset{hllines={, ,}}%
\begin{sphinxVerbatim}[commandchars=\\\{\}]
\PYG{c+c1}{\PYGZsh{} 删除locator和formmater}
\PYG{n}{ax} \PYG{o}{=} \PYG{n}{plt}\PYG{o}{.}\PYG{n}{axes}\PYG{p}{(}\PYG{p}{)}

\PYG{n}{ax}\PYG{o}{.}\PYG{n}{plot}\PYG{p}{(}\PYG{n}{np}\PYG{o}{.}\PYG{n}{random}\PYG{o}{.}\PYG{n}{rand}\PYG{p}{(}\PYG{l+m+mi}{50}\PYG{p}{)}\PYG{p}{)}

\PYG{n}{ax}\PYG{o}{.}\PYG{n}{yaxis}\PYG{o}{.}\PYG{n}{set\PYGZus{}major\PYGZus{}locator}\PYG{p}{(}\PYG{n}{plt}\PYG{o}{.}\PYG{n}{NullLocator}\PYG{p}{(}\PYG{p}{)}\PYG{p}{)}
\PYG{n}{ax}\PYG{o}{.}\PYG{n}{xaxis}\PYG{o}{.}\PYG{n}{set\PYGZus{}major\PYGZus{}formatter}\PYG{p}{(}\PYG{n}{plt}\PYG{o}{.}\PYG{n}{NullFormatter}\PYG{p}{(}\PYG{p}{)}\PYG{p}{)}
\end{sphinxVerbatim}

\sphinxincludegraphics{{output_103_0}.png}


\subsubsection{增减刻度数量}
\label{\detokenize{_u6587_u5b57_u4e0e_u6ce8_u91ca:id7}}
\fvset{hllines={, ,}}%
\begin{sphinxVerbatim}[commandchars=\\\{\}]
\PYG{c+c1}{\PYGZsh{} 一下图例使用默认刻度,但显得过于拥挤}
\PYG{n}{fig}\PYG{p}{,} \PYG{n}{ax} \PYG{o}{=} \PYG{n}{plt}\PYG{o}{.}\PYG{n}{subplots}\PYG{p}{(}\PYG{l+m+mi}{4}\PYG{p}{,} \PYG{l+m+mi}{4}\PYG{p}{,} \PYG{n}{sharex}\PYG{o}{=}\PYG{n+nb+bp}{True}\PYG{p}{,} \PYG{n}{sharey}\PYG{o}{=}\PYG{n+nb+bp}{True}\PYG{p}{)}
\end{sphinxVerbatim}

\sphinxincludegraphics{{output_105_0}.png}

使用plt.MaxNLocator可以设置最多需要显示多少刻度,根据设置的刻度数量,Matplotlib会自动为刻度安排恰当的位置。

\fvset{hllines={, ,}}%
\begin{sphinxVerbatim}[commandchars=\\\{\}]
\PYG{k}{for} \PYG{n}{axi} \PYG{o+ow}{in} \PYG{n}{ax}\PYG{o}{.}\PYG{n}{flat}\PYG{p}{:}
    \PYG{n}{axi}\PYG{o}{.}\PYG{n}{xaxis}\PYG{o}{.}\PYG{n}{set\PYGZus{}major\PYGZus{}locator}\PYG{p}{(}\PYG{n}{plt}\PYG{o}{.}\PYG{n}{MaxNLocator}\PYG{p}{(}\PYG{l+m+mi}{3}\PYG{p}{)}\PYG{p}{)}
    \PYG{n}{axi}\PYG{o}{.}\PYG{n}{yaxis}\PYG{o}{.}\PYG{n}{set\PYGZus{}major\PYGZus{}locator}\PYG{p}{(}\PYG{n}{plt}\PYG{o}{.}\PYG{n}{MaxNLocator}\PYG{p}{(}\PYG{l+m+mi}{3}\PYG{p}{)}\PYG{p}{)}
    
\PYG{n}{fig}
\end{sphinxVerbatim}

\sphinxincludegraphics{{output_107_0}.png}


\subsubsection{花哨的刻度格式}
\label{\detokenize{_u6587_u5b57_u4e0e_u6ce8_u91ca:id8}}
使用MultipleLocator可以实现把刻度放在你提供的数值的倍数上。

\fvset{hllines={, ,}}%
\begin{sphinxVerbatim}[commandchars=\\\{\}]
\PYG{n}{fig}\PYG{p}{,} \PYG{n}{ax} \PYG{o}{=} \PYG{n}{plt}\PYG{o}{.}\PYG{n}{subplots}\PYG{p}{(}\PYG{p}{)}

\PYG{n}{x} \PYG{o}{=} \PYG{n}{np}\PYG{o}{.}\PYG{n}{linspace}\PYG{p}{(}\PYG{l+m+mi}{0}\PYG{p}{,} \PYG{l+m+mi}{3}\PYG{o}{*}\PYG{n}{np}\PYG{o}{.}\PYG{n}{pi}\PYG{p}{,} \PYG{l+m+mi}{100}\PYG{p}{)}

\PYG{n}{ax}\PYG{o}{.}\PYG{n}{plot}\PYG{p}{(}\PYG{n}{x}\PYG{p}{,} \PYG{n}{np}\PYG{o}{.}\PYG{n}{sin}\PYG{p}{(}\PYG{n}{x}\PYG{p}{)}\PYG{p}{,} \PYG{n}{lw}\PYG{o}{=}\PYG{l+m+mi}{3}\PYG{p}{,} \PYG{n}{label}\PYG{o}{=}\PYG{l+s+s1}{\PYGZsq{}}\PYG{l+s+s1}{SIN}\PYG{l+s+s1}{\PYGZsq{}}\PYG{p}{)}
\PYG{n}{ax}\PYG{o}{.}\PYG{n}{plot}\PYG{p}{(}\PYG{n}{x}\PYG{p}{,} \PYG{n}{np}\PYG{o}{.}\PYG{n}{cos}\PYG{p}{(}\PYG{n}{x}\PYG{p}{)}\PYG{p}{,} \PYG{n}{lw}\PYG{o}{=}\PYG{l+m+mi}{3}\PYG{p}{,} \PYG{n}{label}\PYG{o}{=}\PYG{l+s+s1}{\PYGZsq{}}\PYG{l+s+s1}{COS}\PYG{l+s+s1}{\PYGZsq{}}\PYG{p}{)}

\PYG{c+c1}{\PYGZsh{}设置网格,图例和坐标轴上下限}
\PYG{n}{ax}\PYG{o}{.}\PYG{n}{grid}\PYG{p}{(}\PYG{n+nb+bp}{True}\PYG{p}{)}
\PYG{n}{ax}\PYG{o}{.}\PYG{n}{legend}\PYG{p}{(}\PYG{n}{frameon}\PYG{o}{=}\PYG{n+nb+bp}{False}\PYG{p}{)}
\PYG{n}{ax}\PYG{o}{.}\PYG{n}{axis}\PYG{p}{(}\PYG{l+s+s1}{\PYGZsq{}}\PYG{l+s+s1}{equal}\PYG{l+s+s1}{\PYGZsq{}}\PYG{p}{)}
\PYG{n}{ax}\PYG{o}{.}\PYG{n}{set\PYGZus{}xlim}\PYG{p}{(}\PYG{l+m+mi}{0}\PYG{p}{,} \PYG{l+m+mi}{3}\PYG{o}{*}\PYG{n}{np}\PYG{o}{.}\PYG{n}{pi}\PYG{p}{)}
\end{sphinxVerbatim}

\fvset{hllines={, ,}}%
\begin{sphinxVerbatim}[commandchars=\\\{\}]
\PYG{p}{(}\PYG{l+m+mi}{0}\PYG{p}{,} \PYG{l+m+mf}{9.42477796076938}\PYG{p}{)}
\end{sphinxVerbatim}

\sphinxincludegraphics{{output_109_1}.png}

\fvset{hllines={, ,}}%
\begin{sphinxVerbatim}[commandchars=\\\{\}]
\PYG{n}{ax}\PYG{o}{.}\PYG{n}{xaxis}\PYG{o}{.}\PYG{n}{set\PYGZus{}major\PYGZus{}locator}\PYG{p}{(}\PYG{n}{plt}\PYG{o}{.}\PYG{n}{MultipleLocator}\PYG{p}{(}\PYG{n}{np}\PYG{o}{.}\PYG{n}{pi}\PYG{o}{/}\PYG{l+m+mi}{2}\PYG{p}{)}\PYG{p}{)}
\PYG{n}{ax}\PYG{o}{.}\PYG{n}{xaxis}\PYG{o}{.}\PYG{n}{set\PYGZus{}minor\PYGZus{}locator}\PYG{p}{(}\PYG{n}{plt}\PYG{o}{.}\PYG{n}{MultipleLocator}\PYG{p}{(}\PYG{n}{np}\PYG{o}{.}\PYG{n}{pi}\PYG{o}{/}\PYG{l+m+mi}{4}\PYG{p}{)}\PYG{p}{)}
\PYG{n}{fig}
\end{sphinxVerbatim}

\sphinxincludegraphics{{output_110_0}.png}


\subsubsection{定位器和格式生成器常用值}
\label{\detokenize{_u6587_u5b57_u4e0e_u6ce8_u91ca:id9}}
定位器和格式生成器常用的取值在plt命名空间内可以找到,下面列出来:
\begin{itemize}
\item {} 
NullLocator: 无刻度

\item {} 
FixedLocator:刻度位置固定

\item {} 
IndexLocator:用索引做定位器,例如x=range(10)

\item {} 
LinearLocator: 从min到max均匀分布

\item {} 
LogLocator: 从min到max对数分布

\item {} 
MultipleLocator: 刻度和范围是基数的倍数

\item {} 
MaxNLocator: 为最大刻度找到最优位置

\item {} 
AutoLocator: 以MaxNlocator进行简单配置

\item {} 
AutoMinorLocator:次要刻度的定位器

\end{itemize}

格式生成器的取值:
\begin{itemize}
\item {} 
NullFormatter: 刻度上无标签

\item {} 
IndexFormatter: 将一组标签设置为字符串

\item {} 
FixedFormatter: 手动设置标签

\item {} 
FuncFormatter:自定义函数设置标签

\item {} 
FormatStrFormatter:为每个刻度设置字符串格式

\item {} 
ScalarFormatter: 为标量值设置标签

\item {} 
LogFormatter: 对数坐标轴的默认格式生成器

\end{itemize}


\section{配置文件和样式表}
\label{\detokenize{_u914d_u7f6e_u6587_u4ef6_u548c_u6837_u5f0f_u8868:id1}}\label{\detokenize{_u914d_u7f6e_u6587_u4ef6_u548c_u6837_u5f0f_u8868::doc}}
Matplotlib允许手动调整默认样式,如果默认拍照不能满足的情况下,可以手动调整样式。

而对于样式表,每个程序都有一套完整的配色方案,我们可以对起进行修改或者替换,系统给我们提供了很多固定风格的
搭配,如果需要,不建议过多修改配置内容,必要的时候直接替换样式表就好,必要的时候做一些微小改动即可满足需求。


\subsection{手动配置图形}
\label{\detokenize{_u914d_u7f6e_u6587_u4ef6_u548c_u6837_u5f0f_u8868:id2}}
通过手动配置图形,可以改变图形的刻度,背景等内容,下面例子是对图形配置的一个简单示例。

\fvset{hllines={, ,}}%
\begin{sphinxVerbatim}[commandchars=\\\{\}]
\PYG{c+c1}{\PYGZsh{}设置环境}
\PYG{k+kn}{import} \PYG{n+nn}{matplotlib.pyplot} \PYG{k+kn}{as} \PYG{n+nn}{plt}
\PYG{k+kn}{import} \PYG{n+nn}{numpy} \PYG{k+kn}{as} \PYG{n+nn}{np}

\PYG{n}{plt}\PYG{o}{.}\PYG{n}{style}\PYG{o}{.}\PYG{n}{use}\PYG{p}{(}\PYG{l+s+s1}{\PYGZsq{}}\PYG{l+s+s1}{classic}\PYG{l+s+s1}{\PYGZsq{}}\PYG{p}{)}

\PYG{o}{\PYGZpc{}}\PYG{n}{matplotlib} \PYG{n}{inline}
\end{sphinxVerbatim}

\fvset{hllines={, ,}}%
\begin{sphinxVerbatim}[commandchars=\\\{\}]
\PYG{c+c1}{\PYGZsh{}使用默认配置显示图形}
\PYG{n}{x} \PYG{o}{=} \PYG{n}{np}\PYG{o}{.}\PYG{n}{random}\PYG{o}{.}\PYG{n}{randn}\PYG{p}{(}\PYG{l+m+mi}{1000}\PYG{p}{)}
\PYG{n}{plt}\PYG{o}{.}\PYG{n}{hist}\PYG{p}{(}\PYG{n}{x}\PYG{p}{)}
\end{sphinxVerbatim}

\fvset{hllines={, ,}}%
\begin{sphinxVerbatim}[commandchars=\\\{\}]
\PYG{p}{(}\PYG{n}{array}\PYG{p}{(}\PYG{p}{[}  \PYG{l+m+mf}{1.}\PYG{p}{,}   \PYG{l+m+mf}{3.}\PYG{p}{,}  \PYG{l+m+mf}{43.}\PYG{p}{,} \PYG{l+m+mf}{119.}\PYG{p}{,} \PYG{l+m+mf}{237.}\PYG{p}{,} \PYG{l+m+mf}{285.}\PYG{p}{,} \PYG{l+m+mf}{195.}\PYG{p}{,}  \PYG{l+m+mf}{92.}\PYG{p}{,}  \PYG{l+m+mf}{22.}\PYG{p}{,}   \PYG{l+m+mf}{3.}\PYG{p}{]}\PYG{p}{)}\PYG{p}{,}
 \PYG{n}{array}\PYG{p}{(}\PYG{p}{[}\PYG{o}{\PYGZhy{}}\PYG{l+m+mf}{3.90095153}\PYG{p}{,} \PYG{o}{\PYGZhy{}}\PYG{l+m+mf}{3.1706347} \PYG{p}{,} \PYG{o}{\PYGZhy{}}\PYG{l+m+mf}{2.44031787}\PYG{p}{,} \PYG{o}{\PYGZhy{}}\PYG{l+m+mf}{1.71000103}\PYG{p}{,} \PYG{o}{\PYGZhy{}}\PYG{l+m+mf}{0.9796842} \PYG{p}{,}
        \PYG{o}{\PYGZhy{}}\PYG{l+m+mf}{0.24936737}\PYG{p}{,}  \PYG{l+m+mf}{0.48094946}\PYG{p}{,}  \PYG{l+m+mf}{1.2112663} \PYG{p}{,}  \PYG{l+m+mf}{1.94158313}\PYG{p}{,}  \PYG{l+m+mf}{2.67189996}\PYG{p}{,}
         \PYG{l+m+mf}{3.40221679}\PYG{p}{]}\PYG{p}{)}\PYG{p}{,}
 \PYG{o}{\PYGZlt{}}\PYG{n}{a} \PYG{n+nb}{list} \PYG{n}{of} \PYG{l+m+mi}{10} \PYG{n}{Patch} \PYG{n}{objects}\PYG{o}{\PYGZgt{}}\PYG{p}{)}
\end{sphinxVerbatim}

\sphinxincludegraphics{{output_115_1}.png}

\fvset{hllines={, ,}}%
\begin{sphinxVerbatim}[commandchars=\\\{\}]
\PYG{c+c1}{\PYGZsh{} 对图形进行各种配置}

\PYG{n}{ax} \PYG{o}{=} \PYG{n}{plt}\PYG{o}{.}\PYG{n}{axes}\PYG{p}{(}\PYG{p}{)}
\PYG{n}{ax}\PYG{o}{.}\PYG{n}{set\PYGZus{}axisbelow}\PYG{p}{(}\PYG{n+nb+bp}{True}\PYG{p}{)}

\PYG{c+c1}{\PYGZsh{}被色网格线}
\PYG{n}{plt}\PYG{o}{.}\PYG{n}{grid}\PYG{p}{(}\PYG{n}{color}\PYG{o}{=}\PYG{l+s+s1}{\PYGZsq{}}\PYG{l+s+s1}{g}\PYG{l+s+s1}{\PYGZsq{}}\PYG{p}{,} \PYG{n}{linestyle}\PYG{o}{=}\PYG{l+s+s1}{\PYGZsq{}}\PYG{l+s+s1}{solid}\PYG{l+s+s1}{\PYGZsq{}}\PYG{p}{)}

\PYG{c+c1}{\PYGZsh{}隐藏坐标的线条}
\PYG{k}{for} \PYG{n}{spine} \PYG{o+ow}{in} \PYG{n}{ax}\PYG{o}{.}\PYG{n}{spines}\PYG{o}{.}\PYG{n}{values}\PYG{p}{(}\PYG{p}{)}\PYG{p}{:}
    \PYG{n}{spine}\PYG{o}{.}\PYG{n}{set\PYGZus{}visible}\PYG{p}{(}\PYG{n+nb+bp}{False}\PYG{p}{)}

\PYG{c+c1}{\PYGZsh{}隐藏上边和右边的刻度}
\PYG{n}{ax}\PYG{o}{.}\PYG{n}{xaxis}\PYG{o}{.}\PYG{n}{tick\PYGZus{}bottom}\PYG{p}{(}\PYG{p}{)}
\PYG{n}{ax}\PYG{o}{.}\PYG{n}{yaxis}\PYG{o}{.}\PYG{n}{tick\PYGZus{}left}\PYG{p}{(}\PYG{p}{)}

\PYG{c+c1}{\PYGZsh{}弱化刻度和标签}
\PYG{n}{ax}\PYG{o}{.}\PYG{n}{tick\PYGZus{}params}\PYG{p}{(}\PYG{n}{colors}\PYG{o}{=}\PYG{l+s+s1}{\PYGZsq{}}\PYG{l+s+s1}{green}\PYG{l+s+s1}{\PYGZsq{}}\PYG{p}{,} \PYG{n}{direction}\PYG{o}{=}\PYG{l+s+s1}{\PYGZsq{}}\PYG{l+s+s1}{out}\PYG{l+s+s1}{\PYGZsq{}}\PYG{p}{)}
\PYG{k}{for} \PYG{n}{tick} \PYG{o+ow}{in} \PYG{n}{ax}\PYG{o}{.}\PYG{n}{get\PYGZus{}xticklabels}\PYG{p}{(}\PYG{p}{)}\PYG{p}{:}
    \PYG{n}{tick}\PYG{o}{.}\PYG{n}{set\PYGZus{}color}\PYG{p}{(}\PYG{l+s+s1}{\PYGZsq{}}\PYG{l+s+s1}{orange}\PYG{l+s+s1}{\PYGZsq{}}\PYG{p}{)}

\PYG{k}{for} \PYG{n}{tick} \PYG{o+ow}{in} \PYG{n}{ax}\PYG{o}{.}\PYG{n}{get\PYGZus{}yticklabels}\PYG{p}{(}\PYG{p}{)}\PYG{p}{:}
    \PYG{n}{tick}\PYG{o}{.}\PYG{n}{set\PYGZus{}color}\PYG{p}{(}\PYG{l+s+s1}{\PYGZsq{}}\PYG{l+s+s1}{orange}\PYG{l+s+s1}{\PYGZsq{}}\PYG{p}{)}

\PYG{c+c1}{\PYGZsh{}设置频次直方图轮廓色和填充色}
\PYG{n}{ax}\PYG{o}{.}\PYG{n}{hist}\PYG{p}{(}\PYG{n}{x}\PYG{p}{,} \PYG{n}{edgecolor}\PYG{o}{=}\PYG{l+s+s2}{\PYGZdq{}}\PYG{l+s+s2}{\PYGZsh{}1122FF}\PYG{l+s+s2}{\PYGZdq{}}\PYG{p}{,} \PYG{n}{color}\PYG{o}{=}\PYG{l+s+s1}{\PYGZsq{}}\PYG{l+s+s1}{\PYGZsh{}998877}\PYG{l+s+s1}{\PYGZsq{}}\PYG{p}{)}
\end{sphinxVerbatim}

\fvset{hllines={, ,}}%
\begin{sphinxVerbatim}[commandchars=\\\{\}]
\PYG{p}{(}\PYG{n}{array}\PYG{p}{(}\PYG{p}{[}  \PYG{l+m+mf}{1.}\PYG{p}{,}   \PYG{l+m+mf}{3.}\PYG{p}{,}  \PYG{l+m+mf}{43.}\PYG{p}{,} \PYG{l+m+mf}{119.}\PYG{p}{,} \PYG{l+m+mf}{237.}\PYG{p}{,} \PYG{l+m+mf}{285.}\PYG{p}{,} \PYG{l+m+mf}{195.}\PYG{p}{,}  \PYG{l+m+mf}{92.}\PYG{p}{,}  \PYG{l+m+mf}{22.}\PYG{p}{,}   \PYG{l+m+mf}{3.}\PYG{p}{]}\PYG{p}{)}\PYG{p}{,}
 \PYG{n}{array}\PYG{p}{(}\PYG{p}{[}\PYG{o}{\PYGZhy{}}\PYG{l+m+mf}{3.90095153}\PYG{p}{,} \PYG{o}{\PYGZhy{}}\PYG{l+m+mf}{3.1706347} \PYG{p}{,} \PYG{o}{\PYGZhy{}}\PYG{l+m+mf}{2.44031787}\PYG{p}{,} \PYG{o}{\PYGZhy{}}\PYG{l+m+mf}{1.71000103}\PYG{p}{,} \PYG{o}{\PYGZhy{}}\PYG{l+m+mf}{0.9796842} \PYG{p}{,}
        \PYG{o}{\PYGZhy{}}\PYG{l+m+mf}{0.24936737}\PYG{p}{,}  \PYG{l+m+mf}{0.48094946}\PYG{p}{,}  \PYG{l+m+mf}{1.2112663} \PYG{p}{,}  \PYG{l+m+mf}{1.94158313}\PYG{p}{,}  \PYG{l+m+mf}{2.67189996}\PYG{p}{,}
         \PYG{l+m+mf}{3.40221679}\PYG{p}{]}\PYG{p}{)}\PYG{p}{,}
 \PYG{o}{\PYGZlt{}}\PYG{n}{a} \PYG{n+nb}{list} \PYG{n}{of} \PYG{l+m+mi}{10} \PYG{n}{Patch} \PYG{n}{objects}\PYG{o}{\PYGZgt{}}\PYG{p}{)}
\end{sphinxVerbatim}

\sphinxincludegraphics{{output_116_1}.png}


\subsection{修改默认配置}
\label{\detokenize{_u914d_u7f6e_u6587_u4ef6_u548c_u6837_u5f0f_u8868:id3}}
默认配置在修改的时候需要先把系统默认配置保存,使用完毕后需要还原配置。

\fvset{hllines={, ,}}%
\begin{sphinxVerbatim}[commandchars=\\\{\}]
\PYG{c+c1}{\PYGZsh{}保存默认的配置,修改后需要还原}
\PYG{n}{rc\PYGZus{}default} \PYG{o}{=} \PYG{n}{plt}\PYG{o}{.}\PYG{n}{rcParams}\PYG{o}{.}\PYG{n}{copy}\PYG{p}{(}\PYG{p}{)}

\PYG{k+kn}{from} \PYG{n+nn}{matplotlib} \PYG{k+kn}{import} \PYG{n}{cycler}
\PYG{n}{colors} \PYG{o}{=} \PYG{n}{cycler}\PYG{p}{(}\PYG{l+s+s1}{\PYGZsq{}}\PYG{l+s+s1}{color}\PYG{l+s+s1}{\PYGZsq{}}\PYG{p}{,} \PYG{p}{[}\PYG{l+s+s1}{\PYGZsq{}}\PYG{l+s+s1}{\PYGZsh{}777777}\PYG{l+s+s1}{\PYGZsq{}}\PYG{p}{,} \PYG{l+s+s1}{\PYGZsq{}}\PYG{l+s+s1}{\PYGZsh{}888888}\PYG{l+s+s1}{\PYGZsq{}}\PYG{p}{,} \PYG{l+s+s1}{\PYGZsq{}}\PYG{l+s+s1}{\PYGZsh{}999999}\PYG{l+s+s1}{\PYGZsq{}}\PYG{p}{,} \PYG{l+s+s1}{\PYGZsq{}}\PYG{l+s+s1}{\PYGZsh{}AAAAAA}\PYG{l+s+s1}{\PYGZsq{}}\PYG{p}{,} \PYG{l+s+s1}{\PYGZsq{}}\PYG{l+s+s1}{\PYGZsh{}BBBBBB}\PYG{l+s+s1}{\PYGZsq{}}\PYG{p}{,} \PYG{l+s+s1}{\PYGZsq{}}\PYG{l+s+s1}{\PYGZsh{}CCCCCC}\PYG{l+s+s1}{\PYGZsq{}}\PYG{p}{]}\PYG{p}{)}

\PYG{n}{plt}\PYG{o}{.}\PYG{n}{rc}\PYG{p}{(}\PYG{l+s+s1}{\PYGZsq{}}\PYG{l+s+s1}{axes}\PYG{l+s+s1}{\PYGZsq{}}\PYG{p}{,} \PYG{n}{facecolor}\PYG{o}{=}\PYG{l+s+s1}{\PYGZsq{}}\PYG{l+s+s1}{\PYGZsh{}EEEEEE}\PYG{l+s+s1}{\PYGZsq{}}\PYG{p}{,} \PYG{n}{edgecolor}\PYG{o}{=}\PYG{l+s+s1}{\PYGZsq{}}\PYG{l+s+s1}{none}\PYG{l+s+s1}{\PYGZsq{}}\PYG{p}{,} \PYGZbs{}
       \PYG{n}{axisbelow}\PYG{o}{=}\PYG{n+nb+bp}{True}\PYG{p}{,} \PYG{n}{grid}\PYG{o}{=}\PYG{n+nb+bp}{True}\PYG{p}{,} \PYG{n}{prop\PYGZus{}cycle}\PYG{o}{=}\PYG{n}{colors}\PYG{p}{)}

\PYG{n}{plt}\PYG{o}{.}\PYG{n}{rc}\PYG{p}{(}\PYG{l+s+s1}{\PYGZsq{}}\PYG{l+s+s1}{grid}\PYG{l+s+s1}{\PYGZsq{}}\PYG{p}{,} \PYG{n}{color}\PYG{o}{=}\PYG{l+s+s1}{\PYGZsq{}}\PYG{l+s+s1}{w}\PYG{l+s+s1}{\PYGZsq{}}\PYG{p}{,} \PYG{n}{linestyle}\PYG{o}{=}\PYG{l+s+s1}{\PYGZsq{}}\PYG{l+s+s1}{solid}\PYG{l+s+s1}{\PYGZsq{}}\PYG{p}{)}
\PYG{n}{plt}\PYG{o}{.}\PYG{n}{rc}\PYG{p}{(}\PYG{l+s+s1}{\PYGZsq{}}\PYG{l+s+s1}{xtick}\PYG{l+s+s1}{\PYGZsq{}}\PYG{p}{,} \PYG{n}{direction}\PYG{o}{=}\PYG{l+s+s1}{\PYGZsq{}}\PYG{l+s+s1}{out}\PYG{l+s+s1}{\PYGZsq{}}\PYG{p}{,} \PYG{n}{color}\PYG{o}{=}\PYG{l+s+s1}{\PYGZsq{}}\PYG{l+s+s1}{gray}\PYG{l+s+s1}{\PYGZsq{}}\PYG{p}{)}
\PYG{n}{plt}\PYG{o}{.}\PYG{n}{rc}\PYG{p}{(}\PYG{l+s+s1}{\PYGZsq{}}\PYG{l+s+s1}{ytick}\PYG{l+s+s1}{\PYGZsq{}}\PYG{p}{,} \PYG{n}{direction}\PYG{o}{=}\PYG{l+s+s1}{\PYGZsq{}}\PYG{l+s+s1}{out}\PYG{l+s+s1}{\PYGZsq{}}\PYG{p}{,} \PYG{n}{color}\PYG{o}{=}\PYG{l+s+s1}{\PYGZsq{}}\PYG{l+s+s1}{gray}\PYG{l+s+s1}{\PYGZsq{}}\PYG{p}{)}
\PYG{n}{plt}\PYG{o}{.}\PYG{n}{rc}\PYG{p}{(}\PYG{l+s+s1}{\PYGZsq{}}\PYG{l+s+s1}{patch}\PYG{l+s+s1}{\PYGZsq{}}\PYG{p}{,} \PYG{n}{edgecolor}\PYG{o}{=}\PYG{l+s+s1}{\PYGZsq{}}\PYG{l+s+s1}{green}\PYG{l+s+s1}{\PYGZsq{}}\PYG{p}{)}
\PYG{n}{plt}\PYG{o}{.}\PYG{n}{rc}\PYG{p}{(}\PYG{l+s+s1}{\PYGZsq{}}\PYG{l+s+s1}{lines}\PYG{l+s+s1}{\PYGZsq{}}\PYG{p}{,} \PYG{n}{linewidth}\PYG{o}{=}\PYG{l+m+mi}{2}\PYG{p}{)}

\PYG{n}{plt}\PYG{o}{.}\PYG{n}{hist}\PYG{p}{(}\PYG{n}{x}\PYG{p}{)}
\end{sphinxVerbatim}

\fvset{hllines={, ,}}%
\begin{sphinxVerbatim}[commandchars=\\\{\}]
\PYG{p}{(}\PYG{n}{array}\PYG{p}{(}\PYG{p}{[}  \PYG{l+m+mf}{1.}\PYG{p}{,}   \PYG{l+m+mf}{3.}\PYG{p}{,}  \PYG{l+m+mf}{43.}\PYG{p}{,} \PYG{l+m+mf}{119.}\PYG{p}{,} \PYG{l+m+mf}{237.}\PYG{p}{,} \PYG{l+m+mf}{285.}\PYG{p}{,} \PYG{l+m+mf}{195.}\PYG{p}{,}  \PYG{l+m+mf}{92.}\PYG{p}{,}  \PYG{l+m+mf}{22.}\PYG{p}{,}   \PYG{l+m+mf}{3.}\PYG{p}{]}\PYG{p}{)}\PYG{p}{,}
 \PYG{n}{array}\PYG{p}{(}\PYG{p}{[}\PYG{o}{\PYGZhy{}}\PYG{l+m+mf}{3.90095153}\PYG{p}{,} \PYG{o}{\PYGZhy{}}\PYG{l+m+mf}{3.1706347} \PYG{p}{,} \PYG{o}{\PYGZhy{}}\PYG{l+m+mf}{2.44031787}\PYG{p}{,} \PYG{o}{\PYGZhy{}}\PYG{l+m+mf}{1.71000103}\PYG{p}{,} \PYG{o}{\PYGZhy{}}\PYG{l+m+mf}{0.9796842} \PYG{p}{,}
        \PYG{o}{\PYGZhy{}}\PYG{l+m+mf}{0.24936737}\PYG{p}{,}  \PYG{l+m+mf}{0.48094946}\PYG{p}{,}  \PYG{l+m+mf}{1.2112663} \PYG{p}{,}  \PYG{l+m+mf}{1.94158313}\PYG{p}{,}  \PYG{l+m+mf}{2.67189996}\PYG{p}{,}
         \PYG{l+m+mf}{3.40221679}\PYG{p}{]}\PYG{p}{)}\PYG{p}{,}
 \PYG{o}{\PYGZlt{}}\PYG{n}{a} \PYG{n+nb}{list} \PYG{n}{of} \PYG{l+m+mi}{10} \PYG{n}{Patch} \PYG{n}{objects}\PYG{o}{\PYGZgt{}}\PYG{p}{)}
\end{sphinxVerbatim}

\sphinxincludegraphics{{output_118_1}.png}

\fvset{hllines={, ,}}%
\begin{sphinxVerbatim}[commandchars=\\\{\}]
\PYG{k}{for} \PYG{n}{i} \PYG{o+ow}{in} \PYG{n+nb}{range}\PYG{p}{(}\PYG{l+m+mi}{4}\PYG{p}{)}\PYG{p}{:}
    \PYG{n}{plt}\PYG{o}{.}\PYG{n}{plot}\PYG{p}{(}\PYG{n}{np}\PYG{o}{.}\PYG{n}{random}\PYG{o}{.}\PYG{n}{rand}\PYG{p}{(}\PYG{l+m+mi}{10}\PYG{p}{)}\PYG{p}{)}
    
\PYG{n}{plt}\PYG{o}{.}\PYG{n}{rcParams}\PYG{o}{.}\PYG{n}{update}\PYG{p}{(}\PYG{n}{rc\PYGZus{}default}\PYG{p}{)}
\end{sphinxVerbatim}

\sphinxincludegraphics{{output_119_0}.png}


\subsection{样式表}
\label{\detokenize{_u914d_u7f6e_u6587_u4ef6_u548c_u6837_u5f0f_u8868:id4}}
样式表就是系统给提供的完整配置方案。

在style模块里,包含大量样式表可以使用。

使用\sphinxcode{\sphinxupquote{plt.style.available}}可以得到所有可用的样式:

\fvset{hllines={, ,}}%
\begin{sphinxVerbatim}[commandchars=\\\{\}]
\PYG{p}{[}\PYG{l+s+s1}{\PYGZsq{}}\PYG{l+s+s1}{seaborn\PYGZhy{}dark}\PYG{l+s+s1}{\PYGZsq{}}\PYG{p}{,} \PYG{l+s+s1}{\PYGZsq{}}\PYG{l+s+s1}{tableau\PYGZhy{}colorblind10}\PYG{l+s+s1}{\PYGZsq{}}\PYG{p}{,} \PYG{l+s+s1}{\PYGZsq{}}\PYG{l+s+s1}{fivethirtyeight}\PYG{l+s+s1}{\PYGZsq{}}\PYG{p}{,} \PYG{l+s+s1}{\PYGZsq{}}\PYG{l+s+s1}{seaborn\PYGZhy{}white}\PYG{l+s+s1}{\PYGZsq{}}\PYG{p}{,} \PYG{l+s+s1}{\PYGZsq{}}\PYG{l+s+s1}{seaborn\PYGZhy{}bright}\PYG{l+s+s1}{\PYGZsq{}}\PYG{p}{,} \PYG{l+s+s1}{\PYGZsq{}}\PYG{l+s+s1}{seaborn\PYGZhy{}deep}\PYG{l+s+s1}{\PYGZsq{}}\PYG{p}{,} \PYG{l+s+s1}{\PYGZsq{}}\PYG{l+s+s1}{ggplot}\PYG{l+s+s1}{\PYGZsq{}}\PYG{p}{,} \PYG{l+s+s1}{\PYGZsq{}}\PYG{l+s+s1}{Solarize\PYGZus{}Light2}\PYG{l+s+s1}{\PYGZsq{}}\PYG{p}{,} \PYG{l+s+s1}{\PYGZsq{}}\PYG{l+s+s1}{seaborn\PYGZhy{}colorblind}\PYG{l+s+s1}{\PYGZsq{}}\PYG{p}{,} \PYG{l+s+s1}{\PYGZsq{}}\PYG{l+s+s1}{seaborn\PYGZhy{}darkgrid}\PYG{l+s+s1}{\PYGZsq{}}\PYG{p}{,} \PYG{l+s+s1}{\PYGZsq{}}\PYG{l+s+s1}{seaborn\PYGZhy{}pastel}\PYG{l+s+s1}{\PYGZsq{}}\PYG{p}{,} \PYG{l+s+s1}{\PYGZsq{}}\PYG{l+s+s1}{seaborn}\PYG{l+s+s1}{\PYGZsq{}}\PYG{p}{,} \PYG{l+s+s1}{\PYGZsq{}}\PYG{l+s+s1}{seaborn\PYGZhy{}talk}\PYG{l+s+s1}{\PYGZsq{}}\PYG{p}{,} \PYG{l+s+s1}{\PYGZsq{}}\PYG{l+s+s1}{\PYGZus{}classic\PYGZus{}test}\PYG{l+s+s1}{\PYGZsq{}}\PYG{p}{,} \PYG{l+s+s1}{\PYGZsq{}}\PYG{l+s+s1}{seaborn\PYGZhy{}notebook}\PYG{l+s+s1}{\PYGZsq{}}\PYG{p}{,} \PYG{l+s+s1}{\PYGZsq{}}\PYG{l+s+s1}{dark\PYGZus{}background}\PYG{l+s+s1}{\PYGZsq{}}\PYG{p}{,} \PYG{l+s+s1}{\PYGZsq{}}\PYG{l+s+s1}{fast}\PYG{l+s+s1}{\PYGZsq{}}\PYG{p}{,} \PYG{l+s+s1}{\PYGZsq{}}\PYG{l+s+s1}{seaborn\PYGZhy{}dark\PYGZhy{}palette}\PYG{l+s+s1}{\PYGZsq{}}\PYG{p}{,} \PYG{l+s+s1}{\PYGZsq{}}\PYG{l+s+s1}{classic}\PYG{l+s+s1}{\PYGZsq{}}\PYG{p}{,} \PYG{l+s+s1}{\PYGZsq{}}\PYG{l+s+s1}{grayscale}\PYG{l+s+s1}{\PYGZsq{}}\PYG{p}{,} \PYG{l+s+s1}{\PYGZsq{}}\PYG{l+s+s1}{seaborn\PYGZhy{}poster}\PYG{l+s+s1}{\PYGZsq{}}\PYG{p}{,} \PYG{l+s+s1}{\PYGZsq{}}\PYG{l+s+s1}{bmh}\PYG{l+s+s1}{\PYGZsq{}}\PYG{p}{,} \PYG{l+s+s1}{\PYGZsq{}}\PYG{l+s+s1}{seaborn\PYGZhy{}ticks}\PYG{l+s+s1}{\PYGZsq{}}\PYG{p}{,} \PYG{l+s+s1}{\PYGZsq{}}\PYG{l+s+s1}{seaborn\PYGZhy{}whitegrid}\PYG{l+s+s1}{\PYGZsq{}}\PYG{p}{,} \PYG{l+s+s1}{\PYGZsq{}}\PYG{l+s+s1}{seaborn\PYGZhy{}paper}\PYG{l+s+s1}{\PYGZsq{}}\PYG{p}{,} \PYG{l+s+s1}{\PYGZsq{}}\PYG{l+s+s1}{seaborn\PYGZhy{}muted}\PYG{l+s+s1}{\PYGZsq{}}\PYG{p}{]}
\end{sphinxVerbatim}

对样式的使用,可以使用代码\sphinxcode{\sphinxupquote{plt.style.use('stylename')}}来处理。但这个会改变以后所有的风格,如果需要,建议使用风格上下文管理器来临时更换:

\sphinxcode{\sphinxupquote{plt.style.context('stylename')}}

通过风格上下文管理器,我们可以临时更换配置方案而不必操心还原等操作,一旦离开上下文管理器的作用范围,则临时上下文管理器就失效。

\fvset{hllines={, ,}}%
\begin{sphinxVerbatim}[commandchars=\\\{\}]
\PYG{c+c1}{\PYGZsh{} 准备数据}
\PYG{k}{def} \PYG{n+nf}{hist\PYGZus{}and\PYGZus{}lines}\PYG{p}{(}\PYG{p}{)}\PYG{p}{:}
    \PYG{n}{np}\PYG{o}{.}\PYG{n}{random}\PYG{o}{.}\PYG{n}{seed}\PYG{p}{(}\PYG{l+m+mi}{0}\PYG{p}{)}
    \PYG{n}{fig}\PYG{p}{,} \PYG{n}{ax} \PYG{o}{=} \PYG{n}{plt}\PYG{o}{.}\PYG{n}{subplots}\PYG{p}{(}\PYG{l+m+mi}{1}\PYG{p}{,}\PYG{l+m+mi}{2}\PYG{p}{,}\PYG{n}{figsize}\PYG{o}{=}\PYG{p}{(}\PYG{l+m+mi}{11}\PYG{p}{,}\PYG{l+m+mi}{4}\PYG{p}{)}\PYG{p}{)}
    \PYG{n}{ax}\PYG{p}{[}\PYG{l+m+mi}{0}\PYG{p}{]}\PYG{o}{.}\PYG{n}{hist}\PYG{p}{(}\PYG{n}{np}\PYG{o}{.}\PYG{n}{random}\PYG{o}{.}\PYG{n}{randn}\PYG{p}{(}\PYG{l+m+mi}{1000}\PYG{p}{)}\PYG{p}{)}
    \PYG{k}{for} \PYG{n}{i} \PYG{o+ow}{in} \PYG{n+nb}{range}\PYG{p}{(}\PYG{l+m+mi}{3}\PYG{p}{)}\PYG{p}{:}
        \PYG{n}{ax}\PYG{p}{[}\PYG{l+m+mi}{1}\PYG{p}{]}\PYG{o}{.}\PYG{n}{plot}\PYG{p}{(}\PYG{n}{np}\PYG{o}{.}\PYG{n}{random}\PYG{o}{.}\PYG{n}{rand}\PYG{p}{(}\PYG{l+m+mi}{10}\PYG{p}{)}\PYG{p}{)}
    \PYG{n}{ax}\PYG{p}{[}\PYG{l+m+mi}{1}\PYG{p}{]}\PYG{o}{.}\PYG{n}{legend}\PYG{p}{(}\PYG{p}{[}\PYG{l+s+s1}{\PYGZsq{}}\PYG{l+s+s1}{a}\PYG{l+s+s1}{\PYGZsq{}}\PYG{p}{,} \PYG{l+s+s1}{\PYGZsq{}}\PYG{l+s+s1}{b}\PYG{l+s+s1}{\PYGZsq{}}\PYG{p}{,} \PYG{l+s+s1}{\PYGZsq{}}\PYG{l+s+s1}{c}\PYG{l+s+s1}{\PYGZsq{}}\PYG{p}{]}\PYG{p}{,} \PYG{n}{loc}\PYG{o}{=}\PYG{l+s+s1}{\PYGZsq{}}\PYG{l+s+s1}{lower left}\PYG{l+s+s1}{\PYGZsq{}}\PYG{p}{)}
\end{sphinxVerbatim}


\subsubsection{默认风格}
\label{\detokenize{_u914d_u7f6e_u6587_u4ef6_u548c_u6837_u5f0f_u8868:id5}}
\fvset{hllines={, ,}}%
\begin{sphinxVerbatim}[commandchars=\\\{\}]
\PYG{c+c1}{\PYGZsh{}还原默认风格}
\PYG{n}{plt}\PYG{o}{.}\PYG{n}{rcParams}\PYG{o}{.}\PYG{n}{update}\PYG{p}{(}\PYG{n}{rc\PYGZus{}default}\PYG{p}{)}

\PYG{n}{hist\PYGZus{}and\PYGZus{}lines}\PYG{p}{(}\PYG{p}{)}
\end{sphinxVerbatim}

\sphinxincludegraphics{{output_123_0}.png}


\subsubsection{FiveThirtyEight风格}
\label{\detokenize{_u914d_u7f6e_u6587_u4ef6_u548c_u6837_u5f0f_u8868:fivethirtyeight}}
这个风格是模仿网站FiveThirtyEight。

\sphinxcode{\sphinxupquote{http://fivethirtyeight.com}}

\fvset{hllines={, ,}}%
\begin{sphinxVerbatim}[commandchars=\\\{\}]
\PYG{k}{with} \PYG{n}{plt}\PYG{o}{.}\PYG{n}{style}\PYG{o}{.}\PYG{n}{context}\PYG{p}{(}\PYG{l+s+s1}{\PYGZsq{}}\PYG{l+s+s1}{fivethirtyeight}\PYG{l+s+s1}{\PYGZsq{}}\PYG{p}{)}\PYG{p}{:}
    \PYG{n}{hist\PYGZus{}and\PYGZus{}lines}\PYG{p}{(}\PYG{p}{)}
\end{sphinxVerbatim}

\sphinxincludegraphics{{output_125_0}.png}


\subsubsection{ggplot风格}
\label{\detokenize{_u914d_u7f6e_u6587_u4ef6_u548c_u6837_u5f0f_u8868:ggplot}}
ggplot是R语言非常流行的可视化工具,ggplot风格就是模仿ggplot工具包。

\fvset{hllines={, ,}}%
\begin{sphinxVerbatim}[commandchars=\\\{\}]
\PYG{k}{with} \PYG{n}{plt}\PYG{o}{.}\PYG{n}{style}\PYG{o}{.}\PYG{n}{context}\PYG{p}{(}\PYG{l+s+s1}{\PYGZsq{}}\PYG{l+s+s1}{ggplot}\PYG{l+s+s1}{\PYGZsq{}}\PYG{p}{)}\PYG{p}{:}
    \PYG{n}{hist\PYGZus{}and\PYGZus{}lines}\PYG{p}{(}\PYG{p}{)}
\end{sphinxVerbatim}

\sphinxincludegraphics{{output_127_0}.png}


\subsubsection{bmh风格}
\label{\detokenize{_u914d_u7f6e_u6587_u4ef6_u548c_u6837_u5f0f_u8868:bmh}}
\fvset{hllines={, ,}}%
\begin{sphinxVerbatim}[commandchars=\\\{\}]
\PYG{k}{with} \PYG{n}{plt}\PYG{o}{.}\PYG{n}{style}\PYG{o}{.}\PYG{n}{context}\PYG{p}{(}\PYG{l+s+s1}{\PYGZsq{}}\PYG{l+s+s1}{bmh}\PYG{l+s+s1}{\PYGZsq{}}\PYG{p}{)}\PYG{p}{:}
    \PYG{n}{hist\PYGZus{}and\PYGZus{}lines}\PYG{p}{(}\PYG{p}{)}
\end{sphinxVerbatim}

\sphinxincludegraphics{{output_129_0}.png}


\subsubsection{黑色背景风格}
\label{\detokenize{_u914d_u7f6e_u6587_u4ef6_u548c_u6837_u5f0f_u8868:id6}}
\fvset{hllines={, ,}}%
\begin{sphinxVerbatim}[commandchars=\\\{\}]
\PYG{k}{with} \PYG{n}{plt}\PYG{o}{.}\PYG{n}{style}\PYG{o}{.}\PYG{n}{context}\PYG{p}{(}\PYG{l+s+s2}{\PYGZdq{}}\PYG{l+s+s2}{dark\PYGZus{}background}\PYG{l+s+s2}{\PYGZdq{}}\PYG{p}{)}\PYG{p}{:}
    \PYG{n}{hist\PYGZus{}and\PYGZus{}lines}\PYG{p}{(}\PYG{p}{)}
\end{sphinxVerbatim}

\sphinxincludegraphics{{output_131_0}.png}


\subsubsection{灰度风格}
\label{\detokenize{_u914d_u7f6e_u6587_u4ef6_u548c_u6837_u5f0f_u8868:id7}}
\fvset{hllines={, ,}}%
\begin{sphinxVerbatim}[commandchars=\\\{\}]
\PYG{k}{with} \PYG{n}{plt}\PYG{o}{.}\PYG{n}{style}\PYG{o}{.}\PYG{n}{context}\PYG{p}{(}\PYG{l+s+s2}{\PYGZdq{}}\PYG{l+s+s2}{grayscale}\PYG{l+s+s2}{\PYGZdq{}}\PYG{p}{)}\PYG{p}{:}
    \PYG{n}{hist\PYGZus{}and\PYGZus{}lines}\PYG{p}{(}\PYG{p}{)}
\end{sphinxVerbatim}

\sphinxincludegraphics{{output_133_0}.png}


\subsubsection{Seaborn风格}
\label{\detokenize{_u914d_u7f6e_u6587_u4ef6_u548c_u6837_u5f0f_u8868:seaborn}}
\fvset{hllines={, ,}}%
\begin{sphinxVerbatim}[commandchars=\\\{\}]
\PYG{c+c1}{\PYGZsh{} 导入seaborn库的时候自动导入seaborn风格}
\PYG{k+kn}{import} \PYG{n+nn}{seaborn}
\PYG{n}{hist\PYGZus{}and\PYGZus{}lines}\PYG{p}{(}\PYG{p}{)}
\end{sphinxVerbatim}

\sphinxincludegraphics{{output_135_0}.png}


\section{三维图}
\label{\detokenize{_u4e09_u7ef4_u56fe:id1}}\label{\detokenize{_u4e09_u7ef4_u56fe::doc}}
借助于matplotlib自带的mplot3d包,我们可以实现三维图的绘制。

我们默认使用魔法函数\sphinxcode{\sphinxupquote{matplotlib inline}}来进行绘制,但是,如果使用魔法函数\sphinxcode{\sphinxupquote{\%matplotlib notebook}}绘制,画出的图是交互式的,我们可以通过拖动图形来让图形转动。


\subsection{简单三维坐标的绘制}
\label{\detokenize{_u4e09_u7ef4_u56fe:id2}}
通过导入mplot3d包,在\sphinxcode{\sphinxupquote{plt.axes}}中使用projection参数,我们可以采用默认方式绘制一个三维的坐标系。

\fvset{hllines={, ,}}%
\begin{sphinxVerbatim}[commandchars=\\\{\}]
\PYG{o}{\PYGZpc{}}\PYG{n}{matplotlib} \PYG{n}{inline}

\PYG{k+kn}{import} \PYG{n+nn}{numpy} \PYG{k+kn}{as} \PYG{n+nn}{np}
\PYG{k+kn}{import} \PYG{n+nn}{matplotlib.pyplot} \PYG{k+kn}{as} \PYG{n+nn}{plt}

\PYG{k+kn}{from} \PYG{n+nn}{mpl\PYGZus{}toolkits} \PYG{k+kn}{import} \PYG{n}{mplot3d}

\PYG{n}{fig} \PYG{o}{=} \PYG{n}{plt}\PYG{o}{.}\PYG{n}{figure}\PYG{p}{(}\PYG{p}{)}
\PYG{n}{ax} \PYG{o}{=} \PYG{n}{plt}\PYG{o}{.}\PYG{n}{axes}\PYG{p}{(}\PYG{n}{projection}\PYG{o}{=}\PYG{l+s+s1}{\PYGZsq{}}\PYG{l+s+s1}{3d}\PYG{l+s+s1}{\PYGZsq{}}\PYG{p}{)}

\end{sphinxVerbatim}

\sphinxincludegraphics{{output_138_0}.png}


\subsection{三维数据的点和线}
\label{\detokenize{_u4e09_u7ef4_u56fe:id3}}
三维数据主要研究的是\sphinxcode{\sphinxupquote{z=f(x,y)}}, 如果绘制三维图形,需要有x,y,z,求三个数据然后在相应的点上画点或者连线。

使用的画点或者连线的函数常用的是\sphinxcode{\sphinxupquote{ax.plot3d}}和\sphinxcode{\sphinxupquote{ax.scatter3D}}。

为了呈现更好的三维效果,默认散点图会自动改变透明度。

\fvset{hllines={, ,}}%
\begin{sphinxVerbatim}[commandchars=\\\{\}]
\PYG{c+c1}{\PYGZsh{} 画一个螺旋三维线}
\PYG{o}{\PYGZpc{}}\PYG{n}{matplotlib} \PYG{n}{inline}
\PYG{n}{ax} \PYG{o}{=} \PYG{n}{plt}\PYG{o}{.}\PYG{n}{axes}\PYG{p}{(}\PYG{n}{projection}\PYG{o}{=}\PYG{l+s+s1}{\PYGZsq{}}\PYG{l+s+s1}{3d}\PYG{l+s+s1}{\PYGZsq{}}\PYG{p}{)}

\PYG{c+c1}{\PYGZsh{}数据准备}
\PYG{n}{z} \PYG{o}{=} \PYG{n}{np}\PYG{o}{.}\PYG{n}{linspace}\PYG{p}{(}\PYG{l+m+mi}{0}\PYG{p}{,}\PYG{l+m+mi}{15}\PYG{p}{,} \PYG{l+m+mi}{1000}\PYG{p}{)}
\PYG{n}{x} \PYG{o}{=} \PYG{n}{np}\PYG{o}{.}\PYG{n}{sin}\PYG{p}{(}\PYG{n}{z}\PYG{p}{)}
\PYG{n}{y} \PYG{o}{=} \PYG{n}{np}\PYG{o}{.}\PYG{n}{cos}\PYG{p}{(}\PYG{n}{z}\PYG{p}{)}

\PYG{c+c1}{\PYGZsh{}三维线}
\PYG{n}{ax}\PYG{o}{.}\PYG{n}{plot3D}\PYG{p}{(}\PYG{n}{x}\PYG{p}{,} \PYG{n}{y}\PYG{p}{,} \PYG{n}{z}\PYG{p}{,} \PYG{l+s+s1}{\PYGZsq{}}\PYG{l+s+s1}{red}\PYG{l+s+s1}{\PYGZsq{}}\PYG{p}{)}

\PYG{c+c1}{\PYGZsh{}三维点}
\PYG{n}{zdata} \PYG{o}{=} \PYG{l+m+mi}{15} \PYG{o}{*} \PYG{n}{np}\PYG{o}{.}\PYG{n}{random}\PYG{o}{.}\PYG{n}{random}\PYG{p}{(}\PYG{l+m+mi}{100}\PYG{p}{)}
\PYG{n}{xdata} \PYG{o}{=} \PYG{n}{np}\PYG{o}{.}\PYG{n}{sin}\PYG{p}{(}\PYG{n}{zdata}\PYG{p}{)} \PYG{o}{+} \PYG{l+m+mf}{0.1} \PYG{o}{*} \PYG{n}{np}\PYG{o}{.}\PYG{n}{random}\PYG{o}{.}\PYG{n}{randn}\PYG{p}{(}\PYG{l+m+mi}{100}\PYG{p}{)}
\PYG{n}{ydata} \PYG{o}{=} \PYG{n}{np}\PYG{o}{.}\PYG{n}{cos}\PYG{p}{(}\PYG{n}{zdata}\PYG{p}{)} \PYG{o}{+} \PYG{l+m+mf}{0.1} \PYG{o}{*} \PYG{n}{np}\PYG{o}{.}\PYG{n}{random}\PYG{o}{.}\PYG{n}{randn}\PYG{p}{(}\PYG{l+m+mi}{100}\PYG{p}{)}

\PYG{n}{ax}\PYG{o}{.}\PYG{n}{scatter3D}\PYG{p}{(}\PYG{n}{xdata}\PYG{p}{,} \PYG{n}{ydata}\PYG{p}{,} \PYG{n}{zdata}\PYG{p}{,} \PYG{n}{c}\PYG{o}{=}\PYG{n}{zdata}\PYG{p}{,} \PYG{n}{cmap}\PYG{o}{=}\PYG{l+s+s1}{\PYGZsq{}}\PYG{l+s+s1}{Greens}\PYG{l+s+s1}{\PYGZsq{}}\PYG{p}{)}
\end{sphinxVerbatim}

\fvset{hllines={, ,}}%
\begin{sphinxVerbatim}[commandchars=\\\{\}]
\PYG{o}{\PYGZlt{}}\PYG{n}{mpl\PYGZus{}toolkits}\PYG{o}{.}\PYG{n}{mplot3d}\PYG{o}{.}\PYG{n}{art3d}\PYG{o}{.}\PYG{n}{Path3DCollection} \PYG{n}{at} \PYG{l+m+mh}{0x7f5014234748}\PYG{o}{\PYGZgt{}}
\end{sphinxVerbatim}

\sphinxincludegraphics{{output_140_1}.png}


\subsection{三维等高线}
\label{\detokenize{_u4e09_u7ef4_u56fe:id4}}
三维图默认的观察视角可能不是最后的,我们可以时候用\sphinxcode{\sphinxupquote{ax.view\_init}}来改变观察视角,两个参数:
\begin{itemize}
\item {} 
俯仰角度,即x-y平面的旋转角度

\item {} 
方位角度,即沿着z轴顺时针转动角度

\end{itemize}

\fvset{hllines={, ,}}%
\begin{sphinxVerbatim}[commandchars=\\\{\}]
\PYG{k}{def} \PYG{n+nf}{f}\PYG{p}{(}\PYG{n}{x}\PYG{p}{,} \PYG{n}{y}\PYG{p}{)}\PYG{p}{:}
    \PYG{k}{return} \PYG{n}{np}\PYG{o}{.}\PYG{n}{sin}\PYG{p}{(}\PYG{n}{np}\PYG{o}{.}\PYG{n}{sqrt}\PYG{p}{(}\PYG{n}{x}\PYG{o}{*}\PYG{o}{*}\PYG{l+m+mi}{2} \PYG{o}{+} \PYG{n}{y}\PYG{o}{*}\PYG{o}{*}\PYG{l+m+mi}{2}\PYG{p}{)}\PYG{p}{)}

\PYG{n}{x} \PYG{o}{=} \PYG{n}{np}\PYG{o}{.}\PYG{n}{linspace}\PYG{p}{(}\PYG{o}{\PYGZhy{}}\PYG{l+m+mi}{6}\PYG{p}{,} \PYG{l+m+mi}{6}\PYG{p}{,} \PYG{l+m+mi}{30}\PYG{p}{)}
\PYG{n}{y} \PYG{o}{=} \PYG{n}{np}\PYG{o}{.}\PYG{n}{linspace}\PYG{p}{(}\PYG{o}{\PYGZhy{}}\PYG{l+m+mi}{6}\PYG{p}{,} \PYG{l+m+mi}{6}\PYG{p}{,} \PYG{l+m+mi}{30}\PYG{p}{)}

\PYG{n}{X}\PYG{p}{,} \PYG{n}{Y} \PYG{o}{=} \PYG{n}{np}\PYG{o}{.}\PYG{n}{meshgrid}\PYG{p}{(}\PYG{n}{x}\PYG{p}{,} \PYG{n}{y}\PYG{p}{)}
\PYG{n}{Z} \PYG{o}{=} \PYG{n}{f}\PYG{p}{(}\PYG{n}{X}\PYG{p}{,}\PYG{n}{Y}\PYG{p}{)}

\PYG{n}{fig} \PYG{o}{=} \PYG{n}{plt}\PYG{o}{.}\PYG{n}{figure}\PYG{p}{(}\PYG{p}{)}
\PYG{n}{ax} \PYG{o}{=} \PYG{n}{plt}\PYG{o}{.}\PYG{n}{axes}\PYG{p}{(}\PYG{n}{projection}\PYG{o}{=}\PYG{l+s+s1}{\PYGZsq{}}\PYG{l+s+s1}{3d}\PYG{l+s+s1}{\PYGZsq{}}\PYG{p}{)}
\PYG{n}{ax}\PYG{o}{.}\PYG{n}{contour3D}\PYG{p}{(}\PYG{n}{X}\PYG{p}{,} \PYG{n}{Y}\PYG{p}{,} \PYG{n}{Z}\PYG{p}{,} \PYG{l+m+mi}{50}\PYG{p}{)}

\PYG{n}{ax}\PYG{o}{.}\PYG{n}{set\PYGZus{}xlabel}\PYG{p}{(}\PYG{l+s+s2}{\PYGZdq{}}\PYG{l+s+s2}{X\PYGZhy{}Axis}\PYG{l+s+s2}{\PYGZdq{}}\PYG{p}{)}
\PYG{n}{ax}\PYG{o}{.}\PYG{n}{set\PYGZus{}ylabel}\PYG{p}{(}\PYG{l+s+s2}{\PYGZdq{}}\PYG{l+s+s2}{Y\PYGZhy{}Axis}\PYG{l+s+s2}{\PYGZdq{}}\PYG{p}{)}
\PYG{n}{ax}\PYG{o}{.}\PYG{n}{set\PYGZus{}zlabel}\PYG{p}{(}\PYG{l+s+s2}{\PYGZdq{}}\PYG{l+s+s2}{Z\PYGZhy{}Axis}\PYG{l+s+s2}{\PYGZdq{}}\PYG{p}{)}
\end{sphinxVerbatim}

\fvset{hllines={, ,}}%
\begin{sphinxVerbatim}[commandchars=\\\{\}]
\PYG{n}{Text}\PYG{p}{(}\PYG{l+m+mf}{0.5}\PYG{p}{,}\PYG{l+m+mi}{0}\PYG{p}{,}\PYG{l+s+s1}{\PYGZsq{}}\PYG{l+s+s1}{Z\PYGZhy{}Axis}\PYG{l+s+s1}{\PYGZsq{}}\PYG{p}{)}
\end{sphinxVerbatim}

\sphinxincludegraphics{{output_142_1}.png}

\fvset{hllines={, ,}}%
\begin{sphinxVerbatim}[commandchars=\\\{\}]
\PYG{c+c1}{\PYGZsh{} 上图调整旋转角度为(60, 30)}
\PYG{n}{ax}\PYG{o}{.}\PYG{n}{view\PYGZus{}init}\PYG{p}{(}\PYG{l+m+mi}{60}\PYG{p}{,} \PYG{l+m+mi}{30}\PYG{p}{)}
\PYG{n}{fig}
\end{sphinxVerbatim}

\sphinxincludegraphics{{output_143_0}.png}


\subsection{线框图和曲面图}
\label{\detokenize{_u4e09_u7ef4_u56fe:id5}}
线框图是由网格做成的可视化三维图。线框图显示的是由线条组成的轮廓,相对来讲,曲面图是有多边形构成的多边形,可以更好的显示图形表面的结构。

如果选择合适的坐标系,利用合适的数据区间,可以产生类似切片的可视化效果。

\fvset{hllines={, ,}}%
\begin{sphinxVerbatim}[commandchars=\\\{\}]
\PYG{n}{fig} \PYG{o}{=} \PYG{n}{plt}\PYG{o}{.}\PYG{n}{figure}\PYG{p}{(}\PYG{p}{)}
\PYG{n}{ax} \PYG{o}{=} \PYG{n}{plt}\PYG{o}{.}\PYG{n}{axes}\PYG{p}{(}\PYG{n}{projection}\PYG{o}{=}\PYG{l+s+s1}{\PYGZsq{}}\PYG{l+s+s1}{3d}\PYG{l+s+s1}{\PYGZsq{}}\PYG{p}{)}
\PYG{c+c1}{\PYGZsh{}线框图}
\PYG{n}{ax}\PYG{o}{.}\PYG{n}{plot\PYGZus{}wireframe}\PYG{p}{(}\PYG{n}{X}\PYG{p}{,} \PYG{n}{Y}\PYG{p}{,} \PYG{n}{Z}\PYG{p}{,} \PYG{n}{color}\PYG{o}{=}\PYG{l+s+s1}{\PYGZsq{}}\PYG{l+s+s1}{red}\PYG{l+s+s1}{\PYGZsq{}}\PYG{p}{)}
\PYG{n}{ax}\PYG{o}{.}\PYG{n}{set\PYGZus{}title}\PYG{p}{(}\PYG{l+s+s1}{\PYGZsq{}}\PYG{l+s+s1}{wireframe}\PYG{l+s+s1}{\PYGZsq{}}\PYG{p}{)}
\PYG{n}{ax}\PYG{o}{.}\PYG{n}{view\PYGZus{}init}\PYG{p}{(}\PYG{l+m+mi}{60}\PYG{p}{,} \PYG{l+m+mi}{30}\PYG{p}{)}
\end{sphinxVerbatim}

\sphinxincludegraphics{{output_145_0}.png}

\fvset{hllines={, ,}}%
\begin{sphinxVerbatim}[commandchars=\\\{\}]
\PYG{n}{fig} \PYG{o}{=} \PYG{n}{plt}\PYG{o}{.}\PYG{n}{figure}\PYG{p}{(}\PYG{p}{)}
\PYG{n}{ax} \PYG{o}{=} \PYG{n}{plt}\PYG{o}{.}\PYG{n}{axes}\PYG{p}{(}\PYG{n}{projection}\PYG{o}{=}\PYG{l+s+s1}{\PYGZsq{}}\PYG{l+s+s1}{3d}\PYG{l+s+s1}{\PYGZsq{}}\PYG{p}{)}
\PYG{c+c1}{\PYGZsh{}线框图}
\PYG{n}{ax}\PYG{o}{.}\PYG{n}{plot\PYGZus{}surface}\PYG{p}{(}\PYG{n}{X}\PYG{p}{,} \PYG{n}{Y}\PYG{p}{,} \PYG{n}{Z}\PYG{p}{,} \PYG{n}{color}\PYG{o}{=}\PYG{l+s+s1}{\PYGZsq{}}\PYG{l+s+s1}{green}\PYG{l+s+s1}{\PYGZsq{}}\PYG{p}{)}
\PYG{n}{ax}\PYG{o}{.}\PYG{n}{set\PYGZus{}title}\PYG{p}{(}\PYG{l+s+s1}{\PYGZsq{}}\PYG{l+s+s1}{surface}\PYG{l+s+s1}{\PYGZsq{}}\PYG{p}{)}
\PYG{n}{ax}\PYG{o}{.}\PYG{n}{view\PYGZus{}init}\PYG{p}{(}\PYG{l+m+mi}{60}\PYG{p}{,} \PYG{l+m+mi}{30}\PYG{p}{)}
\end{sphinxVerbatim}

\sphinxincludegraphics{{output_146_0}.png}

\fvset{hllines={, ,}}%
\begin{sphinxVerbatim}[commandchars=\\\{\}]
\PYG{n}{r} \PYG{o}{=} \PYG{n}{np}\PYG{o}{.}\PYG{n}{linspace}\PYG{p}{(}\PYG{l+m+mi}{0}\PYG{p}{,} \PYG{l+m+mi}{6}\PYG{p}{,} \PYG{l+m+mi}{30}\PYG{p}{)}
\PYG{n}{theta} \PYG{o}{=} \PYG{n}{np}\PYG{o}{.}\PYG{n}{linspace}\PYG{p}{(}\PYG{o}{\PYGZhy{}}\PYG{l+m+mf}{0.9}\PYG{o}{*}\PYG{n}{np}\PYG{o}{.}\PYG{n}{pi}\PYG{p}{,} \PYG{l+m+mf}{0.8}\PYG{o}{*}\PYG{n}{np}\PYG{o}{.}\PYG{n}{pi}\PYG{p}{,} \PYG{l+m+mi}{40}\PYG{p}{)}

\PYG{n}{r}\PYG{p}{,} \PYG{n}{theta} \PYG{o}{=} \PYG{n}{np}\PYG{o}{.}\PYG{n}{meshgrid}\PYG{p}{(}\PYG{n}{r}\PYG{p}{,} \PYG{n}{theta}\PYG{p}{)}

\PYG{n}{X} \PYG{o}{=} \PYG{n}{r} \PYG{o}{*} \PYG{n}{np}\PYG{o}{.}\PYG{n}{sin}\PYG{p}{(}\PYG{n}{theta}\PYG{p}{)}
\PYG{n}{Y} \PYG{o}{=} \PYG{n}{r} \PYG{o}{*} \PYG{n}{np}\PYG{o}{.}\PYG{n}{cos}\PYG{p}{(}\PYG{n}{theta}\PYG{p}{)}
\PYG{n}{Z} \PYG{o}{=} \PYG{n}{f}\PYG{p}{(}\PYG{n}{X}\PYG{p}{,} \PYG{n}{Y}\PYG{p}{)}

\PYG{n}{ax} \PYG{o}{=} \PYG{n}{plt}\PYG{o}{.}\PYG{n}{axes}\PYG{p}{(}\PYG{n}{projection}\PYG{o}{=}\PYG{l+s+s1}{\PYGZsq{}}\PYG{l+s+s1}{3d}\PYG{l+s+s1}{\PYGZsq{}}\PYG{p}{)}
\PYG{n}{ax}\PYG{o}{.}\PYG{n}{plot\PYGZus{}surface}\PYG{p}{(}\PYG{n}{X}\PYG{p}{,} \PYG{n}{Y}\PYG{p}{,} \PYG{n}{Z}\PYG{p}{,} \PYG{n}{cmap}\PYG{o}{=}\PYG{l+s+s1}{\PYGZsq{}}\PYG{l+s+s1}{viridis}\PYG{l+s+s1}{\PYGZsq{}}\PYG{p}{)}

\PYG{n}{ax}\PYG{o}{.}\PYG{n}{view\PYGZus{}init}\PYG{p}{(}\PYG{l+m+mi}{60}\PYG{p}{,}\PYG{o}{\PYGZhy{}}\PYG{l+m+mi}{40}\PYG{p}{)}
\end{sphinxVerbatim}

\sphinxincludegraphics{{output_147_0}.png}


\subsection{曲面三角剖分}
\label{\detokenize{_u4e09_u7ef4_u56fe:id6}}
对于一些要求均匀采样的网格数据显得太过严格而不太容易实现的图像,我们可以使用三角剖分(triangulation-based plot)来解决。

\fvset{hllines={, ,}}%
\begin{sphinxVerbatim}[commandchars=\\\{\}]
\PYG{c+c1}{\PYGZsh{} 随机散点数据组成的图形}

\PYG{n}{theta} \PYG{o}{=} \PYG{l+m+mi}{2} \PYG{o}{*} \PYG{n}{np}\PYG{o}{.}\PYG{n}{pi} \PYG{o}{*} \PYG{n}{np}\PYG{o}{.}\PYG{n}{random}\PYG{o}{.}\PYG{n}{random}\PYG{p}{(}\PYG{l+m+mi}{1000}\PYG{p}{)}
\PYG{n}{r} \PYG{o}{=} \PYG{l+m+mi}{6} \PYG{o}{*} \PYG{n}{np}\PYG{o}{.}\PYG{n}{random}\PYG{o}{.}\PYG{n}{random}\PYG{p}{(}\PYG{l+m+mi}{1000}\PYG{p}{)}
\PYG{n}{x} \PYG{o}{=} \PYG{n}{np}\PYG{o}{.}\PYG{n}{ravel}\PYG{p}{(}\PYG{n}{r} \PYG{o}{*} \PYG{n}{np}\PYG{o}{.}\PYG{n}{sin}\PYG{p}{(}\PYG{n}{theta}\PYG{p}{)}\PYG{p}{)}
\PYG{n}{y} \PYG{o}{=} \PYG{n}{np}\PYG{o}{.}\PYG{n}{ravel}\PYG{p}{(}\PYG{n}{r} \PYG{o}{*} \PYG{n}{np}\PYG{o}{.}\PYG{n}{cos}\PYG{p}{(}\PYG{n}{theta}\PYG{p}{)}\PYG{p}{)}
\PYG{n}{z} \PYG{o}{=} \PYG{n}{f}\PYG{p}{(}\PYG{n}{x}\PYG{p}{,} \PYG{n}{y}\PYG{p}{)}

\PYG{n}{ax} \PYG{o}{=} \PYG{n}{plt}\PYG{o}{.}\PYG{n}{axes}\PYG{p}{(}\PYG{n}{projection}\PYG{o}{=}\PYG{l+s+s1}{\PYGZsq{}}\PYG{l+s+s1}{3d}\PYG{l+s+s1}{\PYGZsq{}}\PYG{p}{)}
\PYG{n}{ax}\PYG{o}{.}\PYG{n}{scatter}\PYG{p}{(}\PYG{n}{x}\PYG{p}{,} \PYG{n}{y}\PYG{p}{,} \PYG{n}{z}\PYG{p}{,} \PYG{n}{c}\PYG{o}{=}\PYG{n}{z}\PYG{p}{,} \PYG{n}{cmap}\PYG{o}{=}\PYG{l+s+s1}{\PYGZsq{}}\PYG{l+s+s1}{viridis}\PYG{l+s+s1}{\PYGZsq{}}\PYG{p}{,} \PYG{n}{linewidth}\PYG{o}{=}\PYG{l+m+mi}{1}\PYG{p}{)}

\PYG{n}{ax}\PYG{o}{.}\PYG{n}{view\PYGZus{}init}\PYG{p}{(}\PYG{l+m+mi}{60}\PYG{p}{,} \PYG{l+m+mi}{30}\PYG{p}{)}
\end{sphinxVerbatim}

\sphinxincludegraphics{{output_149_0}.png}

\fvset{hllines={, ,}}%
\begin{sphinxVerbatim}[commandchars=\\\{\}]
\PYG{c+c1}{\PYGZsh{}随机数据构成三角剖分}
\PYG{n}{ax} \PYG{o}{=} \PYG{n}{plt}\PYG{o}{.}\PYG{n}{axes}\PYG{p}{(}\PYG{n}{projection}\PYG{o}{=}\PYG{l+s+s1}{\PYGZsq{}}\PYG{l+s+s1}{3d}\PYG{l+s+s1}{\PYGZsq{}}\PYG{p}{)}
\PYG{n}{ax}\PYG{o}{.}\PYG{n}{plot\PYGZus{}trisurf}\PYG{p}{(}\PYG{n}{x}\PYG{p}{,} \PYG{n}{y}\PYG{p}{,} \PYG{n}{z}\PYG{p}{,} \PYG{n}{cmap}\PYG{o}{=}\PYG{l+s+s1}{\PYGZsq{}}\PYG{l+s+s1}{viridis}\PYG{l+s+s1}{\PYGZsq{}}\PYG{p}{,} \PYG{n}{edgecolor}\PYG{o}{=}\PYG{l+s+s1}{\PYGZsq{}}\PYG{l+s+s1}{none}\PYG{l+s+s1}{\PYGZsq{}}\PYG{p}{)}

\PYG{n}{ax}\PYG{o}{.}\PYG{n}{view\PYGZus{}init}\PYG{p}{(}\PYG{l+m+mi}{60}\PYG{p}{,} \PYG{l+m+mi}{30}\PYG{p}{)}
\end{sphinxVerbatim}

\sphinxincludegraphics{{output_150_0}.png}


\section{Seaborn数据可视化}
\label{\detokenize{Seaborn_u6570_u636e_u53ef_u89c6_u5316:seaborn}}\label{\detokenize{Seaborn_u6570_u636e_u53ef_u89c6_u5316::doc}}
Matplotlib作为数据可视化工具非常强大,但相对来讲,还是有一些缺憾,特别是早期版本,引发问题的根本原因主要是,Matplotlib开发早于Pandas,所以前期版本对Pandas的支持可想而知不会太好,相比较而言,Seaborn作为在Matplotlib基础上发展出来的绘图工具,快速得到使用和的认可。


\subsection{Seaborn和Matplotlib的对比}
\label{\detokenize{Seaborn_u6570_u636e_u53ef_u89c6_u5316:seabornmatplotlib}}
同一组数据我们用Seaborn和Matplotlib两种绘图风格来绘图进行对比。

\fvset{hllines={, ,}}%
\begin{sphinxVerbatim}[commandchars=\\\{\}]
\PYG{c+c1}{\PYGZsh{} 准备环境}

\PYG{o}{\PYGZpc{}}\PYG{n}{matplotlib} \PYG{n}{inline}
\PYG{k+kn}{import} \PYG{n+nn}{matplotlib.pyplot} \PYG{k+kn}{as} \PYG{n+nn}{plt}
\PYG{k+kn}{import} \PYG{n+nn}{numpy} \PYG{k+kn}{as} \PYG{n+nn}{np}
\PYG{k+kn}{import} \PYG{n+nn}{pandas} \PYG{k+kn}{as} \PYG{n+nn}{pd}

\PYG{n}{plt}\PYG{o}{.}\PYG{n}{style}\PYG{o}{.}\PYG{n}{use}\PYG{p}{(}\PYG{l+s+s2}{\PYGZdq{}}\PYG{l+s+s2}{classic}\PYG{l+s+s2}{\PYGZdq{}}\PYG{p}{)}
\end{sphinxVerbatim}

\fvset{hllines={, ,}}%
\begin{sphinxVerbatim}[commandchars=\\\{\}]
\PYG{c+c1}{\PYGZsh{} 准备数据}
\PYG{n}{rng} \PYG{o}{=} \PYG{n}{np}\PYG{o}{.}\PYG{n}{random}\PYG{o}{.}\PYG{n}{RandomState}\PYG{p}{(}\PYG{l+m+mi}{0}\PYG{p}{)}
\PYG{n}{x} \PYG{o}{=} \PYG{n}{np}\PYG{o}{.}\PYG{n}{linspace}\PYG{p}{(}\PYG{l+m+mi}{0}\PYG{p}{,} \PYG{l+m+mi}{10}\PYG{p}{,} \PYG{l+m+mi}{500}\PYG{p}{)}
\PYG{n}{y} \PYG{o}{=} \PYG{n}{np}\PYG{o}{.}\PYG{n}{cumsum}\PYG{p}{(}\PYG{n}{rng}\PYG{o}{.}\PYG{n}{randn}\PYG{p}{(}\PYG{l+m+mi}{500}\PYG{p}{,} \PYG{l+m+mi}{6}\PYG{p}{)}\PYG{p}{,} \PYG{l+m+mi}{0}\PYG{p}{)}
\PYG{c+c1}{\PYGZsh{}画图}
\PYG{n}{plt}\PYG{o}{.}\PYG{n}{plot}\PYG{p}{(}\PYG{n}{x}\PYG{p}{,} \PYG{n}{y}\PYG{p}{)}
\PYG{n}{plt}\PYG{o}{.}\PYG{n}{legend}\PYG{p}{(}\PYG{l+s+s2}{\PYGZdq{}}\PYG{l+s+s2}{ABCDEF}\PYG{l+s+s2}{\PYGZdq{}}\PYG{p}{,} \PYG{n}{ncol}\PYG{o}{=}\PYG{l+m+mi}{2}\PYG{p}{,} \PYG{n}{loc}\PYG{o}{=}\PYG{l+s+s1}{\PYGZsq{}}\PYG{l+s+s1}{upper left}\PYG{l+s+s1}{\PYGZsq{}}\PYG{p}{)}
\end{sphinxVerbatim}

\fvset{hllines={, ,}}%
\begin{sphinxVerbatim}[commandchars=\\\{\}]
\PYG{o}{\PYGZlt{}}\PYG{n}{matplotlib}\PYG{o}{.}\PYG{n}{legend}\PYG{o}{.}\PYG{n}{Legend} \PYG{n}{at} \PYG{l+m+mh}{0x7f4ff8e5f9e8}\PYG{o}{\PYGZgt{}}
\end{sphinxVerbatim}

\sphinxincludegraphics{{output_154_1}.png}

\fvset{hllines={, ,}}%
\begin{sphinxVerbatim}[commandchars=\\\{\}]
\PYG{c+c1}{\PYGZsh{} 上述图用seaborn来实现}

\PYG{k+kn}{import} \PYG{n+nn}{seaborn} \PYG{k+kn}{as} \PYG{n+nn}{sns}
\PYG{n}{sns}\PYG{o}{.}\PYG{n}{set}\PYG{p}{(}\PYG{p}{)}

\PYG{n}{plt}\PYG{o}{.}\PYG{n}{plot}\PYG{p}{(}\PYG{n}{x}\PYG{p}{,} \PYG{n}{y}\PYG{p}{)}
\PYG{n}{plt}\PYG{o}{.}\PYG{n}{legend}\PYG{p}{(}\PYG{l+s+s2}{\PYGZdq{}}\PYG{l+s+s2}{ABCDEF}\PYG{l+s+s2}{\PYGZdq{}}\PYG{p}{,} \PYG{n}{ncol}\PYG{o}{=}\PYG{l+m+mi}{2}\PYG{p}{,} \PYG{n}{loc}\PYG{o}{=}\PYG{l+s+s1}{\PYGZsq{}}\PYG{l+s+s1}{upper left}\PYG{l+s+s1}{\PYGZsq{}}\PYG{p}{)}
\end{sphinxVerbatim}

\fvset{hllines={, ,}}%
\begin{sphinxVerbatim}[commandchars=\\\{\}]
\PYG{o}{\PYGZlt{}}\PYG{n}{matplotlib}\PYG{o}{.}\PYG{n}{legend}\PYG{o}{.}\PYG{n}{Legend} \PYG{n}{at} \PYG{l+m+mh}{0x7f50147832b0}\PYG{o}{\PYGZgt{}}
\end{sphinxVerbatim}

\sphinxincludegraphics{{output_155_1}.png}


\subsection{Seaborn图形介绍}
\label{\detokenize{Seaborn_u6570_u636e_u53ef_u89c6_u5316:id1}}
Seaborn的主要思想是用高级命令为统计数据探索和统计模型拟合创建各种图形。


\subsubsection{频次直方图,KDE和密度图}
\label{\detokenize{Seaborn_u6570_u636e_u53ef_u89c6_u5316:kde}}
\fvset{hllines={, ,}}%
\begin{sphinxVerbatim}[commandchars=\\\{\}]
\PYG{c+c1}{\PYGZsh{} 直方图}
\PYG{n}{data} \PYG{o}{=} \PYG{n}{np}\PYG{o}{.}\PYG{n}{random}\PYG{o}{.}\PYG{n}{multivariate\PYGZus{}normal}\PYG{p}{(}\PYG{p}{[}\PYG{l+m+mi}{0}\PYG{p}{,} \PYG{l+m+mi}{0}\PYG{p}{]}\PYG{p}{,} \PYG{p}{[}\PYG{p}{[}\PYG{l+m+mi}{5}\PYG{p}{,} \PYG{l+m+mi}{2}\PYG{p}{]}\PYG{p}{,} \PYG{p}{[}\PYG{l+m+mi}{2}\PYG{p}{,} \PYG{l+m+mi}{2}\PYG{p}{]}\PYG{p}{]}\PYG{p}{,} \PYG{n}{size}\PYG{o}{=}\PYG{l+m+mi}{2000}\PYG{p}{)}
\PYG{n}{data} \PYG{o}{=} \PYG{n}{pd}\PYG{o}{.}\PYG{n}{DataFrame}\PYG{p}{(}\PYG{n}{data}\PYG{p}{,} \PYG{n}{columns}\PYG{o}{=}\PYG{p}{[}\PYG{l+s+s1}{\PYGZsq{}}\PYG{l+s+s1}{X}\PYG{l+s+s1}{\PYGZsq{}}\PYG{p}{,} \PYG{l+s+s1}{\PYGZsq{}}\PYG{l+s+s1}{Y}\PYG{l+s+s1}{\PYGZsq{}}\PYG{p}{]}\PYG{p}{)}

\PYG{k}{for} \PYG{n}{col} \PYG{o+ow}{in} \PYG{l+s+s1}{\PYGZsq{}}\PYG{l+s+s1}{XY}\PYG{l+s+s1}{\PYGZsq{}}\PYG{p}{:}
    \PYG{n}{plt}\PYG{o}{.}\PYG{n}{hist}\PYG{p}{(}\PYG{n}{data}\PYG{p}{[}\PYG{n}{col}\PYG{p}{]}\PYG{p}{,} \PYG{n}{alpha}\PYG{o}{=}\PYG{l+m+mf}{0.5}\PYG{p}{)}
\end{sphinxVerbatim}

\sphinxincludegraphics{{output_157_0}.png}

\fvset{hllines={, ,}}%
\begin{sphinxVerbatim}[commandchars=\\\{\}]
\PYG{c+c1}{\PYGZsh{} sns.kedplot可以实现KDE变量帆布的平滑估计}

\PYG{k}{for} \PYG{n}{col} \PYG{o+ow}{in} \PYG{l+s+s1}{\PYGZsq{}}\PYG{l+s+s1}{XY}\PYG{l+s+s1}{\PYGZsq{}}\PYG{p}{:}
    \PYG{n}{sns}\PYG{o}{.}\PYG{n}{kdeplot}\PYG{p}{(}\PYG{n}{data}\PYG{p}{[}\PYG{n}{col}\PYG{p}{]}\PYG{p}{,} \PYG{n}{shade}\PYG{o}{=}\PYG{n+nb+bp}{True}\PYG{p}{)}
\end{sphinxVerbatim}

\fvset{hllines={, ,}}%
\begin{sphinxVerbatim}[commandchars=\\\{\}]
/sw/ana/lib/python3.7/site\PYGZhy{}packages/scipy/stats/stats.py:1713: FutureWarning: Using a non\PYGZhy{}tuple sequence for multidimensional indexing is deprecated; use {}`arr[tuple(seq)]{}` instead of {}`arr[seq]{}`. In the future this will be interpreted as an array index, {}`arr[np.array(seq)]{}`, which will result either in an error or a different result.
  return np.add.reduce(sorted[indexer] * weights, axis=axis) / sumval
\end{sphinxVerbatim}

\sphinxincludegraphics{{output_158_1}.png}

\fvset{hllines={, ,}}%
\begin{sphinxVerbatim}[commandchars=\\\{\}]
\PYG{c+c1}{\PYGZsh{} distplot可以让频次直方图和KDE结合起来}

\PYG{n}{sns}\PYG{o}{.}\PYG{n}{distplot}\PYG{p}{(}\PYG{n}{data}\PYG{p}{[}\PYG{l+s+s1}{\PYGZsq{}}\PYG{l+s+s1}{X}\PYG{l+s+s1}{\PYGZsq{}}\PYG{p}{]}\PYG{p}{)}
\PYG{n}{sns}\PYG{o}{.}\PYG{n}{distplot}\PYG{p}{(}\PYG{n}{data}\PYG{p}{[}\PYG{l+s+s1}{\PYGZsq{}}\PYG{l+s+s1}{Y}\PYG{l+s+s1}{\PYGZsq{}}\PYG{p}{]}\PYG{p}{)}
\end{sphinxVerbatim}

\fvset{hllines={, ,}}%
\begin{sphinxVerbatim}[commandchars=\\\{\}]
/sw/ana/lib/python3.7/site\PYGZhy{}packages/scipy/stats/stats.py:1713: FutureWarning: Using a non\PYGZhy{}tuple sequence for multidimensional indexing is deprecated; use {}`arr[tuple(seq)]{}` instead of {}`arr[seq]{}`. In the future this will be interpreted as an array index, {}`arr[np.array(seq)]{}`, which will result either in an error or a different result.
  return np.add.reduce(sorted[indexer] * weights, axis=axis) / sumval





\PYGZlt{}matplotlib.axes.\PYGZus{}subplots.AxesSubplot at 0x7f4ff6c17da0\PYGZgt{}
\end{sphinxVerbatim}

\sphinxincludegraphics{{output_159_2}.png}

\fvset{hllines={, ,}}%
\begin{sphinxVerbatim}[commandchars=\\\{\}]
\PYG{c+c1}{\PYGZsh{} 如果是想kdeplot输入的二维数据,则可以获得二维数据的可视化}
\PYG{n}{sns}\PYG{o}{.}\PYG{n}{kdeplot}\PYG{p}{(}\PYG{n}{data}\PYG{p}{)}
\end{sphinxVerbatim}

\fvset{hllines={, ,}}%
\begin{sphinxVerbatim}[commandchars=\\\{\}]
/sw/ana/lib/python3.7/site\PYGZhy{}packages/seaborn/distributions.py:679: UserWarning: Passing a 2D dataset for a bivariate plot is deprecated in favor of kdeplot(x, y), and it will cause an error in future versions. Please update your code.
  warnings.warn(warn\PYGZus{}msg, UserWarning)
/sw/ana/lib/python3.7/site\PYGZhy{}packages/scipy/stats/stats.py:1713: FutureWarning: Using a non\PYGZhy{}tuple sequence for multidimensional indexing is deprecated; use {}`arr[tuple(seq)]{}` instead of {}`arr[seq]{}`. In the future this will be interpreted as an array index, {}`arr[np.array(seq)]{}`, which will result either in an error or a different result.
  return np.add.reduce(sorted[indexer] * weights, axis=axis) / sumval





\PYGZlt{}matplotlib.axes.\PYGZus{}subplots.AxesSubplot at 0x7f4ff6b99f28\PYGZgt{}
\end{sphinxVerbatim}

\sphinxincludegraphics{{output_160_2}.png}

\fvset{hllines={, ,}}%
\begin{sphinxVerbatim}[commandchars=\\\{\}]
\PYG{c+c1}{\PYGZsh{} jointplot可以同时看到两个变量的联合分布和单变量的独立分布}

\PYG{k}{with} \PYG{n}{sns}\PYG{o}{.}\PYG{n}{axes\PYGZus{}style}\PYG{p}{(}\PYG{l+s+s1}{\PYGZsq{}}\PYG{l+s+s1}{white}\PYG{l+s+s1}{\PYGZsq{}}\PYG{p}{)}\PYG{p}{:}
    \PYG{n}{sns}\PYG{o}{.}\PYG{n}{jointplot}\PYG{p}{(}\PYG{l+s+s1}{\PYGZsq{}}\PYG{l+s+s1}{X}\PYG{l+s+s1}{\PYGZsq{}}\PYG{p}{,} \PYG{l+s+s1}{\PYGZsq{}}\PYG{l+s+s1}{Y}\PYG{l+s+s1}{\PYGZsq{}}\PYG{p}{,} \PYG{n}{data}\PYG{p}{,} \PYG{n}{kind}\PYG{o}{=}\PYG{l+s+s1}{\PYGZsq{}}\PYG{l+s+s1}{kde}\PYG{l+s+s1}{\PYGZsq{}}\PYG{p}{)}
\end{sphinxVerbatim}

\fvset{hllines={, ,}}%
\begin{sphinxVerbatim}[commandchars=\\\{\}]
/sw/ana/lib/python3.7/site\PYGZhy{}packages/scipy/stats/stats.py:1713: FutureWarning: Using a non\PYGZhy{}tuple sequence for multidimensional indexing is deprecated; use {}`arr[tuple(seq)]{}` instead of {}`arr[seq]{}`. In the future this will be interpreted as an array index, {}`arr[np.array(seq)]{}`, which will result either in an error or a different result.
  return np.add.reduce(sorted[indexer] * weights, axis=axis) / sumval
\end{sphinxVerbatim}

\sphinxincludegraphics{{output_161_1}.png}

\fvset{hllines={, ,}}%
\begin{sphinxVerbatim}[commandchars=\\\{\}]
\PYG{c+c1}{\PYGZsh{} 向jointplot函数传递一些参数,可以用六边形块代替频次直方图。}

\PYG{k}{with} \PYG{n}{sns}\PYG{o}{.}\PYG{n}{axes\PYGZus{}style}\PYG{p}{(}\PYG{l+s+s2}{\PYGZdq{}}\PYG{l+s+s2}{white}\PYG{l+s+s2}{\PYGZdq{}}\PYG{p}{)}\PYG{p}{:}
    \PYG{n}{sns}\PYG{o}{.}\PYG{n}{jointplot}\PYG{p}{(}\PYG{l+s+s1}{\PYGZsq{}}\PYG{l+s+s1}{X}\PYG{l+s+s1}{\PYGZsq{}}\PYG{p}{,} \PYG{l+s+s1}{\PYGZsq{}}\PYG{l+s+s1}{Y}\PYG{l+s+s1}{\PYGZsq{}}\PYG{p}{,} \PYG{n}{data}\PYG{p}{,} \PYG{n}{kind}\PYG{o}{=}\PYG{l+s+s1}{\PYGZsq{}}\PYG{l+s+s1}{hex}\PYG{l+s+s1}{\PYGZsq{}}\PYG{p}{)}
\end{sphinxVerbatim}

\fvset{hllines={, ,}}%
\begin{sphinxVerbatim}[commandchars=\\\{\}]
/sw/ana/lib/python3.7/site\PYGZhy{}packages/scipy/stats/stats.py:1713: FutureWarning: Using a non\PYGZhy{}tuple sequence for multidimensional indexing is deprecated; use {}`arr[tuple(seq)]{}` instead of {}`arr[seq]{}`. In the future this will be interpreted as an array index, {}`arr[np.array(seq)]{}`, which will result either in an error or a different result.
  return np.add.reduce(sorted[indexer] * weights, axis=axis) / sumval
\end{sphinxVerbatim}

\sphinxincludegraphics{{output_162_1}.png}


\subsubsection{矩阵图}
\label{\detokenize{Seaborn_u6570_u636e_u53ef_u89c6_u5316:id2}}
对多维数据集进行可视化时,需要用到矩阵图(pair plot)来表示变量中任意两个变量的关系,探索多维数据不同维度的相关性。

\fvset{hllines={, ,}}%
\begin{sphinxVerbatim}[commandchars=\\\{\}]
\PYG{k+kn}{import} \PYG{n+nn}{seaborn} \PYG{k+kn}{as} \PYG{n+nn}{sns}
\PYG{c+c1}{\PYGZsh{} 载入鸢尾花数据集}
\PYG{c+c1}{\PYGZsh{} 鸢尾花数据集研究的是花瓣和花萼的尺寸和鸢尾花品种的关系}
\PYG{c+c1}{\PYGZsh{} 数据从Github下载,可能需要多试几次才能成功}
\PYG{n}{iris} \PYG{o}{=} \PYG{n}{sns}\PYG{o}{.}\PYG{n}{load\PYGZus{}dataset}\PYG{p}{(}\PYG{l+s+s1}{\PYGZsq{}}\PYG{l+s+s1}{iris}\PYG{l+s+s1}{\PYGZsq{}}\PYG{p}{)}
\PYG{n}{iris}\PYG{o}{.}\PYG{n}{head}\PYG{p}{(}\PYG{p}{)}
\end{sphinxVerbatim}



\fvset{hllines={, ,}}%
\begin{sphinxVerbatim}[commandchars=\\\{\}]
\PYG{o}{.}\PYG{n}{dataframe} \PYG{n}{tbody} \PYG{n}{tr} \PYG{n}{th} \PYG{p}{\PYGZob{}}
    \PYG{n}{vertical}\PYG{o}{\PYGZhy{}}\PYG{n}{align}\PYG{p}{:} \PYG{n}{top}\PYG{p}{;}
\PYG{p}{\PYGZcb{}}

\PYG{o}{.}\PYG{n}{dataframe} \PYG{n}{thead} \PYG{n}{th} \PYG{p}{\PYGZob{}}
    \PYG{n}{text}\PYG{o}{\PYGZhy{}}\PYG{n}{align}\PYG{p}{:} \PYG{n}{right}\PYG{p}{;}
\PYG{p}{\PYGZcb{}}
\end{sphinxVerbatim}



\fvset{hllines={, ,}}%
\begin{sphinxVerbatim}[commandchars=\\\{\}]
\PYG{c+c1}{\PYGZsh{}展示四个变量的矩阵}
\PYG{n}{sns}\PYG{o}{.}\PYG{n}{pairplot}\PYG{p}{(}\PYG{n}{iris}\PYG{p}{,} \PYG{n}{hue}\PYG{o}{=}\PYG{l+s+s1}{\PYGZsq{}}\PYG{l+s+s1}{species}\PYG{l+s+s1}{\PYGZsq{}}\PYG{p}{,} \PYG{n}{size}\PYG{o}{=}\PYG{l+m+mf}{2.5}\PYG{p}{)}
\end{sphinxVerbatim}

\fvset{hllines={, ,}}%
\begin{sphinxVerbatim}[commandchars=\\\{\}]
/sw/ana/lib/python3.7/site\PYGZhy{}packages/seaborn/axisgrid.py:2065: UserWarning: The {}`size{}` parameter has been renamed to {}`height{}`; pleaes update your code.
  warnings.warn(msg, UserWarning)
/sw/ana/lib/python3.7/site\PYGZhy{}packages/scipy/stats/stats.py:1713: FutureWarning: Using a non\PYGZhy{}tuple sequence for multidimensional indexing is deprecated; use {}`arr[tuple(seq)]{}` instead of {}`arr[seq]{}`. In the future this will be interpreted as an array index, {}`arr[np.array(seq)]{}`, which will result either in an error or a different result.
  return np.add.reduce(sorted[indexer] * weights, axis=axis) / sumval





\PYGZlt{}seaborn.axisgrid.PairGrid at 0x7f385d5cb748\PYGZgt{}
\end{sphinxVerbatim}

\sphinxincludegraphics{{output_165_2}.png}


\subsubsection{分面频次直方图}
\label{\detokenize{Seaborn_u6570_u636e_u53ef_u89c6_u5316:id3}}
借助数据子集的频次直方图观察数据是一种很好的观察方法,下面案例展示的是服务员收取消费的数据。

\fvset{hllines={, ,}}%
\begin{sphinxVerbatim}[commandchars=\\\{\}]
\PYG{c+c1}{\PYGZsh{} 再如数据}
\PYG{c+c1}{\PYGZsh{} tips数据研究的是服务员小费数量和顾客年龄等之间的关系}
\PYG{n}{tips} \PYG{o}{=} \PYG{n}{sns}\PYG{o}{.}\PYG{n}{load\PYGZus{}dataset}\PYG{p}{(}\PYG{l+s+s1}{\PYGZsq{}}\PYG{l+s+s1}{tips}\PYG{l+s+s1}{\PYGZsq{}}\PYG{p}{)}
\PYG{n}{tips}\PYG{o}{.}\PYG{n}{head}\PYG{p}{(}\PYG{p}{)}
\end{sphinxVerbatim}



\fvset{hllines={, ,}}%
\begin{sphinxVerbatim}[commandchars=\\\{\}]
\PYG{o}{.}\PYG{n}{dataframe} \PYG{n}{tbody} \PYG{n}{tr} \PYG{n}{th} \PYG{p}{\PYGZob{}}
    \PYG{n}{vertical}\PYG{o}{\PYGZhy{}}\PYG{n}{align}\PYG{p}{:} \PYG{n}{top}\PYG{p}{;}
\PYG{p}{\PYGZcb{}}

\PYG{o}{.}\PYG{n}{dataframe} \PYG{n}{thead} \PYG{n}{th} \PYG{p}{\PYGZob{}}
    \PYG{n}{text}\PYG{o}{\PYGZhy{}}\PYG{n}{align}\PYG{p}{:} \PYG{n}{right}\PYG{p}{;}
\PYG{p}{\PYGZcb{}}
\end{sphinxVerbatim}



\fvset{hllines={, ,}}%
\begin{sphinxVerbatim}[commandchars=\\\{\}]
\PYG{c+c1}{\PYGZsh{} 把数量变成百分比}
\PYG{n}{tips}\PYG{p}{[}\PYG{l+s+s1}{\PYGZsq{}}\PYG{l+s+s1}{tip\PYGZus{}pct}\PYG{l+s+s1}{\PYGZsq{}}\PYG{p}{]} \PYG{o}{=} \PYG{l+m+mi}{100} \PYG{o}{*} \PYG{n}{tips}\PYG{p}{[}\PYG{l+s+s1}{\PYGZsq{}}\PYG{l+s+s1}{tip}\PYG{l+s+s1}{\PYGZsq{}}\PYG{p}{]} \PYG{o}{/} \PYG{n}{tips}\PYG{p}{[}\PYG{l+s+s1}{\PYGZsq{}}\PYG{l+s+s1}{total\PYGZus{}bill}\PYG{l+s+s1}{\PYGZsq{}}\PYG{p}{]}
\PYG{n}{grid} \PYG{o}{=} \PYG{n}{sns}\PYG{o}{.}\PYG{n}{FacetGrid}\PYG{p}{(}\PYG{n}{tips}\PYG{p}{,} \PYG{n}{row}\PYG{o}{=}\PYG{l+s+s1}{\PYGZsq{}}\PYG{l+s+s1}{sex}\PYG{l+s+s1}{\PYGZsq{}}\PYG{p}{,} \PYG{n}{col}\PYG{o}{=}\PYG{l+s+s1}{\PYGZsq{}}\PYG{l+s+s1}{time}\PYG{l+s+s1}{\PYGZsq{}}\PYG{p}{,} \PYG{n}{margin\PYGZus{}titles}\PYG{o}{=}\PYG{n+nb+bp}{True}\PYG{p}{)}
\PYG{n}{grid}\PYG{o}{.}\PYG{n}{map}\PYG{p}{(}\PYG{n}{plt}\PYG{o}{.}\PYG{n}{hist}\PYG{p}{,} \PYG{l+s+s1}{\PYGZsq{}}\PYG{l+s+s1}{tip\PYGZus{}pct}\PYG{l+s+s1}{\PYGZsq{}}\PYG{p}{,} \PYG{n}{bins}\PYG{o}{=}\PYG{n}{np}\PYG{o}{.}\PYG{n}{linspace}\PYG{p}{(}\PYG{l+m+mi}{0}\PYG{p}{,} \PYG{l+m+mi}{40}\PYG{p}{,} \PYG{l+m+mi}{15}\PYG{p}{)}\PYG{p}{)}
\end{sphinxVerbatim}

\fvset{hllines={, ,}}%
\begin{sphinxVerbatim}[commandchars=\\\{\}]
\PYG{o}{\PYGZlt{}}\PYG{n}{seaborn}\PYG{o}{.}\PYG{n}{axisgrid}\PYG{o}{.}\PYG{n}{FacetGrid} \PYG{n}{at} \PYG{l+m+mh}{0x7f4ff5bccb38}\PYG{o}{\PYGZgt{}}
\end{sphinxVerbatim}

\sphinxincludegraphics{{output_168_1}.png}


\subsubsection{因子图}
\label{\detokenize{Seaborn_u6570_u636e_u53ef_u89c6_u5316:id4}}
因子图(Factor Plot)也是对数据子集进行可视化的方法,可以用来观察一个参数在另一个参数间隔中的分布情况。

\fvset{hllines={, ,}}%
\begin{sphinxVerbatim}[commandchars=\\\{\}]
\PYG{k}{with} \PYG{n}{sns}\PYG{o}{.}\PYG{n}{axes\PYGZus{}style}\PYG{p}{(}\PYG{n}{style}\PYG{o}{=}\PYG{l+s+s1}{\PYGZsq{}}\PYG{l+s+s1}{ticks}\PYG{l+s+s1}{\PYGZsq{}}\PYG{p}{)}\PYG{p}{:}
    \PYG{n}{g} \PYG{o}{=} \PYG{n}{sns}\PYG{o}{.}\PYG{n}{factorplot}\PYG{p}{(}\PYG{l+s+s1}{\PYGZsq{}}\PYG{l+s+s1}{day}\PYG{l+s+s1}{\PYGZsq{}}\PYG{p}{,} \PYG{l+s+s1}{\PYGZsq{}}\PYG{l+s+s1}{total\PYGZus{}bill}\PYG{l+s+s1}{\PYGZsq{}}\PYG{p}{,} \PYG{l+s+s1}{\PYGZsq{}}\PYG{l+s+s1}{sex}\PYG{l+s+s1}{\PYGZsq{}}\PYG{p}{,} \PYG{n}{data}\PYG{o}{=}\PYG{n}{tips}\PYG{p}{,} \PYG{n}{kind}\PYG{o}{=}\PYG{l+s+s1}{\PYGZsq{}}\PYG{l+s+s1}{box}\PYG{l+s+s1}{\PYGZsq{}}\PYG{p}{)}
    \PYG{n}{g}\PYG{o}{.}\PYG{n}{set\PYGZus{}axis\PYGZus{}labels}\PYG{p}{(}\PYG{l+s+s2}{\PYGZdq{}}\PYG{l+s+s2}{Day}\PYG{l+s+s2}{\PYGZdq{}}\PYG{p}{,} \PYG{l+s+s1}{\PYGZsq{}}\PYG{l+s+s1}{Total\PYGZus{}bill}\PYG{l+s+s1}{\PYGZsq{}}\PYG{p}{)}
\end{sphinxVerbatim}

\fvset{hllines={, ,}}%
\begin{sphinxVerbatim}[commandchars=\\\{\}]
/sw/ana/lib/python3.7/site\PYGZhy{}packages/seaborn/categorical.py:3666: UserWarning: The {}`factorplot{}` function has been renamed to {}`catplot{}`. The original name will be removed in a future release. Please update your code. Note that the default {}`kind{}` in {}`factorplot{}` ({}`\PYGZsq{}point\PYGZsq{}{}`) has changed {}`\PYGZsq{}strip\PYGZsq{}{}` in {}`catplot{}`.
  warnings.warn(msg)
\end{sphinxVerbatim}

\sphinxincludegraphics{{output_170_1}.png}


\subsubsection{联合分布图}
\label{\detokenize{Seaborn_u6570_u636e_u53ef_u89c6_u5316:id5}}
可以用jointplot画出不同数据集的联合分布和各数据样本本身的分布。

\fvset{hllines={, ,}}%
\begin{sphinxVerbatim}[commandchars=\\\{\}]
\PYG{c+c1}{\PYGZsh{}联合分布图}
\PYG{k}{with} \PYG{n}{sns}\PYG{o}{.}\PYG{n}{axes\PYGZus{}style}\PYG{p}{(}\PYG{l+s+s2}{\PYGZdq{}}\PYG{l+s+s2}{white}\PYG{l+s+s2}{\PYGZdq{}}\PYG{p}{)}\PYG{p}{:}
    \PYG{n}{sns}\PYG{o}{.}\PYG{n}{jointplot}\PYG{p}{(}\PYG{l+s+s1}{\PYGZsq{}}\PYG{l+s+s1}{total\PYGZus{}bill}\PYG{l+s+s1}{\PYGZsq{}}\PYG{p}{,} \PYG{l+s+s1}{\PYGZsq{}}\PYG{l+s+s1}{tip}\PYG{l+s+s1}{\PYGZsq{}}\PYG{p}{,} \PYG{n}{data}\PYG{o}{=}\PYG{n}{tips}\PYG{p}{,} \PYG{n}{kind}\PYG{o}{=}\PYG{l+s+s1}{\PYGZsq{}}\PYG{l+s+s1}{hex}\PYG{l+s+s1}{\PYGZsq{}}\PYG{p}{)}
\end{sphinxVerbatim}

\fvset{hllines={, ,}}%
\begin{sphinxVerbatim}[commandchars=\\\{\}]
/sw/ana/lib/python3.7/site\PYGZhy{}packages/scipy/stats/stats.py:1713: FutureWarning: Using a non\PYGZhy{}tuple sequence for multidimensional indexing is deprecated; use {}`arr[tuple(seq)]{}` instead of {}`arr[seq]{}`. In the future this will be interpreted as an array index, {}`arr[np.array(seq)]{}`, which will result either in an error or a different result.
  return np.add.reduce(sorted[indexer] * weights, axis=axis) / sumval
\end{sphinxVerbatim}

\sphinxincludegraphics{{output_172_1}.png}

\fvset{hllines={, ,}}%
\begin{sphinxVerbatim}[commandchars=\\\{\}]
\PYG{c+c1}{\PYGZsh{}带回归拟合的联合分布图}
\PYG{c+c1}{\PYGZsh{}联合分布图自行进行kde和回归}
\PYG{n}{sns}\PYG{o}{.}\PYG{n}{jointplot}\PYG{p}{(}\PYG{l+s+s1}{\PYGZsq{}}\PYG{l+s+s1}{total\PYGZus{}bill}\PYG{l+s+s1}{\PYGZsq{}}\PYG{p}{,} \PYG{l+s+s1}{\PYGZsq{}}\PYG{l+s+s1}{tip}\PYG{l+s+s1}{\PYGZsq{}}\PYG{p}{,} \PYG{n}{data}\PYG{o}{=}\PYG{n}{tips}\PYG{p}{,} \PYG{n}{kind}\PYG{o}{=}\PYG{l+s+s1}{\PYGZsq{}}\PYG{l+s+s1}{reg}\PYG{l+s+s1}{\PYGZsq{}}\PYG{p}{)}
\end{sphinxVerbatim}

\fvset{hllines={, ,}}%
\begin{sphinxVerbatim}[commandchars=\\\{\}]
/sw/ana/lib/python3.7/site\PYGZhy{}packages/scipy/stats/stats.py:1713: FutureWarning: Using a non\PYGZhy{}tuple sequence for multidimensional indexing is deprecated; use {}`arr[tuple(seq)]{}` instead of {}`arr[seq]{}`. In the future this will be interpreted as an array index, {}`arr[np.array(seq)]{}`, which will result either in an error or a different result.
  return np.add.reduce(sorted[indexer] * weights, axis=axis) / sumval





\PYGZlt{}seaborn.axisgrid.JointGrid at 0x7f4ff2f6d5f8\PYGZgt{}
\end{sphinxVerbatim}

\sphinxincludegraphics{{output_173_2}.png}


\subsubsection{条形图}
\label{\detokenize{Seaborn_u6570_u636e_u53ef_u89c6_u5316:id6}}
\fvset{hllines={, ,}}%
\begin{sphinxVerbatim}[commandchars=\\\{\}]
\PYG{c+c1}{\PYGZsh{} 行星观测数据的展示}
\PYG{n}{planets} \PYG{o}{=} \PYG{n}{sns}\PYG{o}{.}\PYG{n}{load\PYGZus{}dataset}\PYG{p}{(}\PYG{l+s+s1}{\PYGZsq{}}\PYG{l+s+s1}{planets}\PYG{l+s+s1}{\PYGZsq{}}\PYG{p}{)}
\PYG{n}{planets}\PYG{o}{.}\PYG{n}{head}\PYG{p}{(}\PYG{p}{)}
\end{sphinxVerbatim}



\fvset{hllines={, ,}}%
\begin{sphinxVerbatim}[commandchars=\\\{\}]
\PYG{o}{.}\PYG{n}{dataframe} \PYG{n}{tbody} \PYG{n}{tr} \PYG{n}{th} \PYG{p}{\PYGZob{}}
    \PYG{n}{vertical}\PYG{o}{\PYGZhy{}}\PYG{n}{align}\PYG{p}{:} \PYG{n}{top}\PYG{p}{;}
\PYG{p}{\PYGZcb{}}

\PYG{o}{.}\PYG{n}{dataframe} \PYG{n}{thead} \PYG{n}{th} \PYG{p}{\PYGZob{}}
    \PYG{n}{text}\PYG{o}{\PYGZhy{}}\PYG{n}{align}\PYG{p}{:} \PYG{n}{right}\PYG{p}{;}
\PYG{p}{\PYGZcb{}}
\end{sphinxVerbatim}



\fvset{hllines={, ,}}%
\begin{sphinxVerbatim}[commandchars=\\\{\}]
\PYG{k}{with} \PYG{n}{sns}\PYG{o}{.}\PYG{n}{axes\PYGZus{}style}\PYG{p}{(}\PYG{l+s+s2}{\PYGZdq{}}\PYG{l+s+s2}{white}\PYG{l+s+s2}{\PYGZdq{}}\PYG{p}{)}\PYG{p}{:}
    \PYG{n}{g} \PYG{o}{=} \PYG{n}{sns}\PYG{o}{.}\PYG{n}{factorplot}\PYG{p}{(}\PYG{l+s+s1}{\PYGZsq{}}\PYG{l+s+s1}{year}\PYG{l+s+s1}{\PYGZsq{}}\PYG{p}{,} \PYG{n}{data}\PYG{o}{=}\PYG{n}{planets}\PYG{p}{,} \PYG{n}{aspect}\PYG{o}{=}\PYG{l+m+mi}{2}\PYG{p}{,} \PYGZbs{}
                      \PYG{n}{kind}\PYG{o}{=}\PYG{l+s+s1}{\PYGZsq{}}\PYG{l+s+s1}{count}\PYG{l+s+s1}{\PYGZsq{}}\PYG{p}{,} \PYG{n}{color}\PYG{o}{=}\PYG{l+s+s1}{\PYGZsq{}}\PYG{l+s+s1}{steelblue}\PYG{l+s+s1}{\PYGZsq{}}\PYG{p}{)}
    \PYG{n}{g}\PYG{o}{.}\PYG{n}{set\PYGZus{}xticklabels}\PYG{p}{(}\PYG{n}{step}\PYG{o}{=}\PYG{l+m+mi}{5}\PYG{p}{)}
\end{sphinxVerbatim}

\fvset{hllines={, ,}}%
\begin{sphinxVerbatim}[commandchars=\\\{\}]
/sw/ana/lib/python3.7/site\PYGZhy{}packages/seaborn/categorical.py:3666: UserWarning: The {}`factorplot{}` function has been renamed to {}`catplot{}`. The original name will be removed in a future release. Please update your code. Note that the default {}`kind{}` in {}`factorplot{}` ({}`\PYGZsq{}point\PYGZsq{}{}`) has changed {}`\PYGZsq{}strip\PYGZsq{}{}` in {}`catplot{}`.
  warnings.warn(msg)
\end{sphinxVerbatim}

\sphinxincludegraphics{{output_176_1}.png}

\fvset{hllines={, ,}}%
\begin{sphinxVerbatim}[commandchars=\\\{\}]
\PYG{c+c1}{\PYGZsh{}对比用不同的方法发现行星数量}
\PYG{k}{with} \PYG{n}{sns}\PYG{o}{.}\PYG{n}{axes\PYGZus{}style}\PYG{p}{(}\PYG{l+s+s2}{\PYGZdq{}}\PYG{l+s+s2}{white}\PYG{l+s+s2}{\PYGZdq{}}\PYG{p}{)}\PYG{p}{:}
    \PYG{n}{g} \PYG{o}{=} \PYG{n}{sns}\PYG{o}{.}\PYG{n}{factorplot}\PYG{p}{(}\PYG{l+s+s1}{\PYGZsq{}}\PYG{l+s+s1}{year}\PYG{l+s+s1}{\PYGZsq{}}\PYG{p}{,} \PYG{n}{data}\PYG{o}{=}\PYG{n}{planets}\PYG{p}{,} \PYG{n}{aspect}\PYG{o}{=}\PYG{l+m+mf}{4.0}\PYG{p}{,} \PYGZbs{}
                      \PYG{n}{kind}\PYG{o}{=}\PYG{l+s+s1}{\PYGZsq{}}\PYG{l+s+s1}{count}\PYG{l+s+s1}{\PYGZsq{}}\PYG{p}{,} \PYG{n}{hue}\PYG{o}{=}\PYG{l+s+s1}{\PYGZsq{}}\PYG{l+s+s1}{method}\PYG{l+s+s1}{\PYGZsq{}}\PYG{p}{,} \PYG{n}{order}\PYG{o}{=}\PYG{n+nb}{range}\PYG{p}{(}\PYG{l+m+mi}{2001}\PYG{p}{,} \PYG{l+m+mi}{2015}\PYG{p}{)}\PYG{p}{)}
    \PYG{n}{g}\PYG{o}{.}\PYG{n}{set\PYGZus{}ylabels}\PYG{p}{(}\PYG{l+s+s1}{\PYGZsq{}}\PYG{l+s+s1}{Number of Planets Discovered}\PYG{l+s+s1}{\PYGZsq{}}\PYG{p}{)}
\end{sphinxVerbatim}

\fvset{hllines={, ,}}%
\begin{sphinxVerbatim}[commandchars=\\\{\}]
/sw/ana/lib/python3.7/site\PYGZhy{}packages/seaborn/categorical.py:3666: UserWarning: The {}`factorplot{}` function has been renamed to {}`catplot{}`. The original name will be removed in a future release. Please update your code. Note that the default {}`kind{}` in {}`factorplot{}` ({}`\PYGZsq{}point\PYGZsq{}{}`) has changed {}`\PYGZsq{}strip\PYGZsq{}{}` in {}`catplot{}`.
  warnings.warn(msg)
\end{sphinxVerbatim}

\sphinxincludegraphics{{output_177_1}.png}



\renewcommand{\indexname}{索引}
\printindex
\end{document}